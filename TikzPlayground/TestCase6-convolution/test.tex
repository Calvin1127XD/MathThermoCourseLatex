\documentclass[tikz,border=10pt]{standalone}
\usepackage{amsmath,amssymb}
\usetikzlibrary{matrix,positioning,backgrounds,fit}

\begin{document}
\begin{tikzpicture}[
    cell/.style={minimum size=8mm, draw, outer sep=0pt, inner sep=0pt},
    highlight red/.style={fill=red!30},
    highlight green/.style={fill=green!30},
    kernel cell/.style={minimum size=8mm, draw, outer sep=0pt, inner sep=0pt, fill=gray!20},
]

% === Input Matrix I (7x7) ===
\def\cellsize{0.8}

% Draw the input matrix
\foreach \row [count=\r from 0] in {
    {0,1,1,1,0,0,0},
    {0,0,1,1,1,0,0},
    {0,0,0,1,1,1,0},
    {0,0,0,1,1,0,0},
    {0,0,1,1,0,0,0},
    {0,1,1,0,0,0,0},
    {1,1,0,0,0,0,0}%
} {
    \foreach \val [count=\c from 0] in \row {
        % Check if in red region (rows 0-2, cols 2-4)
        \pgfmathsetmacro{\inred}{(\r >= 0 && \r <= 2 && \c >= 2 && \c <= 4) ? 1 : 0}
        \ifnum\inred=1
            \node[cell, highlight red] at (\c*\cellsize, -\r*\cellsize) {\val};
        \else
            \node[cell] at (\c*\cellsize, -\r*\cellsize) {\val};
        \fi
    }
}

% Red border around 3x3 region
\draw[red, very thick] (2*\cellsize-0.4, 0.4) rectangle (4*\cellsize+0.4, -2*\cellsize-0.4);

% Small subscript annotations in red region (showing kernel weights)
\node[font=\tiny, blue] at (2*\cellsize+0.2, -0.2) {$\times 1$};
\node[font=\tiny, blue] at (3*\cellsize+0.2, -0.2) {$\times 0$};
\node[font=\tiny, blue] at (4*\cellsize+0.2, -0.2) {$\times 1$};
\node[font=\tiny, blue] at (2*\cellsize+0.2, -\cellsize-0.2) {$\times 0$};
\node[font=\tiny, blue] at (3*\cellsize+0.2, -\cellsize-0.2) {$\times 1$};
\node[font=\tiny, blue] at (4*\cellsize+0.2, -\cellsize-0.2) {$\times 0$};
\node[font=\tiny, blue] at (2*\cellsize+0.2, -2*\cellsize-0.2) {$\times 1$};
\node[font=\tiny, blue] at (3*\cellsize+0.2, -2*\cellsize-0.2) {$\times 0$};
\node[font=\tiny, blue] at (4*\cellsize+0.2, -2*\cellsize-0.2) {$\times 1$};

% Label for I
\node at (3*\cellsize, -7*\cellsize-0.3) {\Large $\mathbf{I}$};

% === Convolution operator ===
\node at (7.5*\cellsize, -3*\cellsize) {\Large $*$};

% === Kernel K (3x3) ===
\def\koffset{9}
\foreach \row [count=\r from 0] in {
    {1,0,1},
    {0,1,0},
    {1,0,1}%
} {
    \foreach \val [count=\c from 0] in \row {
        \node[kernel cell] at (\koffset*\cellsize+\c*\cellsize, -\r*\cellsize-\cellsize) {\val};
    }
}

% Blue border around kernel
\draw[blue, very thick] (\koffset*\cellsize-0.4, -\cellsize+0.4) rectangle ((\koffset+2)*\cellsize+0.4, -3*\cellsize-0.4);

% Label for K
\node at ((\koffset+1)*\cellsize, -4.5*\cellsize) {\Large $\mathbf{K}$};

% === Equals sign ===
\node at (13*\cellsize, -3*\cellsize) {\Large $=$};

% === Output Matrix I*K (5x5) ===
\def\ooffset{14.5}
\foreach \row [count=\r from 0] in {
    {1,4,3,4,1},
    {1,2,4,3,3},
    {1,2,3,4,1},
    {1,3,3,1,1},
    {3,3,1,1,0}%
} {
    \foreach \val [count=\c from 0] in \row {
        % Check if this is the highlighted cell (row 0, col 3)
        \pgfmathsetmacro{\ingreen}{(\r == 0 && \c == 3) ? 1 : 0}
        \ifnum\ingreen=1
            \node[cell, highlight green] at (\ooffset*\cellsize+\c*\cellsize, -\r*\cellsize) {\val};
        \else
            \node[cell] at (\ooffset*\cellsize+\c*\cellsize, -\r*\cellsize) {\val};
        \fi
    }
}

% Green border around highlighted cell
\draw[green!70!black, very thick] ((\ooffset+3)*\cellsize-0.4, 0.4) rectangle ((\ooffset+3)*\cellsize+0.4, -0.4);

% Label for I*K
\node at ((\ooffset+2)*\cellsize, -5.5*\cellsize) {\Large $\mathbf{I} * \mathbf{K}$};

% === Dotted lines connecting red region to kernel and to output ===
% From red region to kernel
\draw[dotted, thick, green!50!black] (4*\cellsize+0.4, -0.4) -- (\koffset*\cellsize-0.4, -\cellsize-0.4);
\draw[dotted, thick, green!50!black] (4*\cellsize+0.4, -2*\cellsize-0.4) -- (\koffset*\cellsize-0.4, -3*\cellsize-0.4);

% From kernel to output
\draw[dotted, thick, green!50!black] ((\koffset+2)*\cellsize+0.4, -\cellsize+0.4) -- ((\ooffset+3)*\cellsize-0.4, 0.4);
\draw[dotted, thick, green!50!black] ((\koffset+2)*\cellsize+0.4, -3*\cellsize-0.4) -- ((\ooffset+3)*\cellsize+0.4, -0.4);

\end{tikzpicture}
\end{document}
