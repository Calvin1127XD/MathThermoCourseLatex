\documentclass[tikz,border=10pt]{standalone}
\usepackage{pgfplots}
\usepgfplotslibrary{fillbetween}
\pgfplotsset{compat=1.18}

\begin{document}
\begin{tikzpicture}
\begin{axis}[
    axis lines=center,
    domain=-3:11,
    samples=300,
    xmin=-4, xmax=11.5,
    ymin=-2.5, ymax=5.5,
    width=16cm, height=9cm,
    xtick=\empty,
    ytick=\empty,
    axis line style={very thick, -{Stealth[length=5mm]}},
    clip=false,
]

% ============================================================
% Revised functions to better match the original diagram:
%
% ORIGINAL ANALYSIS:
% - f(x) blue: smooth increasing, starts low-left (~1 at x=-3),
%   gently rises, becomes steeper but NOT too steep on right.
%   Reaches about 4.5 at x=10. More like x^(1.5) or moderate cubic.
%
% - h(x) red: clearly curved (concave up for decreasing),
%   starts at ~4 on far left, decreases to ~1 on far right.
%   Visible arch shape.
%
% - g(x) green: pronounced U-shape, starts high-left (~4),
%   dips to minimum (~0.5) around x=4.5, rises to ~3 on right.
%
% Intersection constraints:
%   f(a=2) = g(2), f(b=5) = h(5), g(c=8) = h(8)
%
% NEW FUNCTIONS:
%
% f(x): Use a gentler power law / moderate polynomial
%   f(x) = 0.25*x + 0.8  is too linear
%   f(x) = 0.02*x^2 + 0.15*x + 0.7
%   f(2)  = 0.08+0.30+0.70 = 1.08
%   f(5)  = 0.50+0.75+0.70 = 1.95
%   f(8)  = 1.28+1.20+0.70 = 3.18
%   f(10) = 2.00+1.50+0.70 = 4.20
%   f(-3) = 0.18-0.45+0.70 = 0.43
%   This looks good - moderate growth.
%
% g(x): U-shaped with minimum near x=4.5
%   g(x) = 0.1*(x-4.5)^2 + 0.3
%   g(2)  = 0.1*6.25+0.3 = 0.925
%   g(5)  = 0.1*0.25+0.3 = 0.325
%   g(8)  = 0.1*12.25+0.3 = 1.525
%   g(10) = 0.1*30.25+0.3 = 3.325
%   g(-2) = 0.1*42.25+0.3 = 4.525
%
%   Need f(2) = g(2): f(2)=1.08, g(2)=0.925 -- close but not exact
%   Adjust: let me set g(x) = 0.1*(x-4.5)^2 + c and f(2)=g(2):
%   0.02*4+0.15*2+0.7 = 0.1*(2-4.5)^2 + c
%   1.08 = 0.625 + c => c = 0.455
%   g(x) = 0.1*(x-4.5)^2 + 0.455
%   g(2) = 0.625+0.455 = 1.08 YES = f(2)
%   g(5) = 0.025+0.455 = 0.48
%   g(8) = 1.225+0.455 = 1.68
%
% h(x): Need f(5)=h(5) and g(8)=h(8)
%   f(5) = 1.95, so h(5) = 1.95
%   g(8) = 1.68, so h(8) = 1.68
%   h(x) = -a*x + b (linear) or with curvature
%   Linear: slope = (1.68-1.95)/(8-5) = -0.09
%   h(x) = -0.09*x + 2.4
%   h(5) = -0.45+2.4 = 1.95 YES
%   h(8) = -0.72+2.4 = 1.68 YES
%   h(2) = -0.18+2.4 = 2.22
%   h(0) = 2.4
%   h(-2) = 2.58
%   TOO FLAT. In the original, h starts much higher on left.
%
% The issue is that with f moderate and g moderate, h must also be moderate.
% But the original shows h starting quite high (near top of plot at left).
%
% Let me RESCALE everything. The key visual feature is:
%   - At x=-2 to 0: h and g are both high (3-4 range), f is low (~0.5-1)
%   - At x=a=2: f=g (~1.5), h is above (~3)
%   - At x=b=5: f=h (~2.5), g is below (~0.5)
%   - At x=c=8: h=g (~1.8), f is above (~3.5)
%
% FINAL FUNCTIONS:
% f(x) = 0.03*x^2 + 0.1*x + 0.9
%   f(2) = 0.12+0.2+0.9 = 1.22
%   f(5) = 0.75+0.5+0.9 = 2.15
%   f(8) = 1.92+0.8+0.9 = 3.62
%   f(10) = 3.0+1.0+0.9 = 4.9
%   f(-3) = 0.27-0.3+0.9 = 0.87
%
% g(x) = 0.12*(x-4.2)^2 + 0.1
%   g(2) = 0.12*4.84+0.1 = 0.581+0.1 = 0.68  too low
%
% OK let me just set the values I want and solve exactly.
% Target: f(2)=g(2)=1.3, f(5)=h(5)=2.3, g(8)=h(8)=1.6
%
% f(x) = ax^2+bx+c: f(2)=1.3, f(5)=2.3
%   4a+2b+c=1.3, 25a+5b+c=2.3
%   21a+3b=1.0 => 7a+b=1/3
%   Want f(-3)~0.5 and f(10)~4.5
%   f(10)=100a+10b+c, f(-3)=9a-3b+c
%   Choose a=0.025: b=1/3-0.175=0.158, c=1.3-0.1-0.317=0.883
%   f(x)=0.025x^2+0.158x+0.883
%   f(2)=0.1+0.317+0.883=1.3 YES
%   f(5)=0.625+0.792+0.883=2.3 YES
%   f(8)=1.6+1.267+0.883=3.75
%   f(10)=2.5+1.583+0.883=4.97
%   f(-3)=0.225-0.475+0.883=0.633
%   f(0)=0.883
%   Good moderate growth!
%
% g(x)=a(x-x0)^2+g_min: g(2)=1.3, g(8)=1.6
%   a(2-x0)^2+g_min=1.3, a(8-x0)^2+g_min=1.6
%   a[(8-x0)^2-(2-x0)^2]=0.3
%   a[(64-16x0+x0^2)-(4-4x0+x0^2)]=0.3
%   a[60-12x0]=0.3
%   Choose x0=4.5: a[60-54]=0.3 => 6a=0.3 => a=0.05
%   g_min = 1.3-0.05*(2-4.5)^2 = 1.3-0.05*6.25 = 1.3-0.3125 = 0.9875
%   Hmm, g_min=0.99 is too high, the U-shape won't dip enough.
%   Try x0=4: a[60-48]=0.3 => 12a=0.3 => a=0.025
%   g_min = 1.3-0.025*(2-4)^2 = 1.3-0.1 = 1.2  Still too high.
%
%   The problem is the asymmetry. Let me use a general quadratic:
%   g(x)=Ax^2+Bx+C, g(2)=1.3, g(8)=1.6
%   4A+2B+C=1.3, 64A+8B+C=1.6
%   60A+6B=0.3 => 10A+B=0.05
%   Want minimum around x=4.5 => -B/(2A)=4.5 => B=-9A
%   10A-9A=0.05 => A=0.05
%   B=-0.45
%   C=1.3-0.2+0.9=2.0
%   g(x)=0.05x^2-0.45x+2.0
%   g(2)=0.2-0.9+2.0=1.3 YES
%   g(5)=1.25-2.25+2.0=1.0
%   g(8)=3.2-3.6+2.0=1.6 YES
%   g(4.5)=1.0125-2.025+2.0=0.9875 -- minimum ~1.0
%   g(0)=2.0
%   g(-2)=0.2+0.9+2.0=3.1
%   g(-3)=0.45+1.35+2.0=3.8
%   g(10)=5.0-4.5+2.0=2.5
%   This looks good! Pronounced U-shape with minimum at ~1.0
%
% h(x): h(5)=2.3, h(8)=1.6
%   Use quadratic with concave-down curvature:
%   h(x)=-Ax^2+Bx+C
%   Choose A=0.015:
%   -0.375+5B+C=2.3 => 5B+C=2.675
%   -0.96+8B+C=1.6  => 8B+C=2.56
%   3B=-0.115 => B=-0.0383
%   C=2.675+0.192=2.867
%   h(x)=-0.015x^2-0.0383x+2.867
%   h(0)=2.867
%   h(2)=-0.06-0.077+2.867=2.73
%   h(-2)=-0.06+0.077+2.867=2.88
%   h(-3)=-0.135+0.115+2.867=2.85
%   TOO FLAT on left side.
%
%   I need h to be HIGH on the left. Let me increase curvature:
%   A=0.04:
%   -1.0+5B+C=2.3 => 5B+C=3.3
%   -2.56+8B+C=1.6 => 8B+C=4.16
%   3B=0.86 => B=0.287
%   C=3.3-1.433=1.867
%   h(x)=-0.04x^2+0.287x+1.867
%   h(2)=-0.16+0.573+1.867=2.28 (want 2.3, close)
%   h(5)=-1.0+1.433+1.867=2.3 YES
%   h(8)=-2.56+2.293+1.867=1.6 YES
%   h(0)=1.867  -- not very high
%   h(-2)=-0.16-0.573+1.867=1.133
%   This goes DOWN to the left, which is WRONG.
%   The vertex is at x = 0.287/(2*0.04) = 3.59
%   So h increases from left to x=3.59, then decreases.
%   That gives a hump shape which doesn't match.
%
%   For h to be monotonically decreasing AND start high on left:
%   Need h'(x) < 0 for all x in domain. With h=-Ax^2+Bx+C:
%   h'(x) = -2Ax+B < 0 => x > B/(2A)
%   So we need vertex at x < -3 (left of visible domain)
%   B/(2A) < -3 => B < -6A (with A > 0)
%   If A=0.01: B < -0.06
%   h(5)=-0.25+5B+C=2.3 => 5B+C=2.55
%   h(8)=-0.64+8B+C=1.6 => 8B+C=2.24
%   3B=-0.31 => B=-0.103
%   C=2.55+0.517=3.067
%   h(x)=-0.01x^2-0.103x+3.067
%   h(0)=3.067
%   h(-3)=-0.09+0.31+3.067=3.287
%   h(-2)=-0.04+0.207+3.067=3.234
%   vertex at x=-0.103/(0.02) = -5.17 (left of domain, good!)
%   h is decreasing for x > -5.17, which is our whole visible range.
%   h(10)=-1.0-1.033+3.067=1.034
%   Range: from ~3.3 to ~1.0, that's decent curvature.
%
%   But original shows h starting at ~4 on the far left. Let me boost:
%   A=0.02:
%   h(5)=-0.5+5B+C=2.3 => 5B+C=2.8
%   h(8)=-1.28+8B+C=1.6 => 8B+C=2.88
%   3B=0.08 => B=0.0267
%   C=2.8-0.133=2.667
%   h(x)=-0.02x^2+0.0267x+2.667
%   vertex at x=0.0267/0.04=0.667 => h increases to x=0.67 then decreases
%   h(0)=2.667, h(-3)=-0.18-0.08+2.667=2.407
%   Not monotonically decreasing from left. And doesn't go high enough.
%
%   Let me try: h(x) = 3.5 - 0.2*x  (linear, simple)
%   h(5)=2.5, h(8)=1.9 -- not matching targets
%   Adjust targets: f(5)=h(5), g(8)=h(8)
%   If h is linear with slope -0.2: h(x) = -0.2x + k
%   f(5) = h(5) = -1 + k => k = f(5)+1
%   g(8) = h(8) = -1.6 + k => k = g(8)+1.6
%   So f(5)+1 = g(8)+1.6 => f(5) = g(8)+0.6
%   With g(8)=1.6: f(5)=2.2, k=3.2
%   h(x) = -0.2x + 3.2
%   h(2) = 2.8
%   h(0) = 3.2
%   h(-3) = 3.8
%   h(10) = 1.2
%   Range 3.8 to 1.2. That's visible decrease. Not much curvature
%   but the original might just be gently curved.
%
%   Let me recompute f and g with these:
%   f(2)=g(2), f(5)=2.2
%   4a+2b+c=g(2), 25a+5b+c=2.2
%   g(2)=0.05*4-0.45*2+2.0=0.2-0.9+2.0=1.3
%   So f(2)=1.3, f(5)=2.2
%   21a+3b=0.9 => 7a+b=0.3
%   a=0.025: b=0.3-0.175=0.125, c=1.3-0.1-0.25=0.95
%   f(x)=0.025x^2+0.125x+0.95
%   f(2)=0.1+0.25+0.95=1.3 YES
%   f(5)=0.625+0.625+0.95=2.2 YES
%   f(8)=1.6+1.0+0.95=3.55
%   f(10)=2.5+1.25+0.95=4.7
%   f(-3)=0.225-0.375+0.95=0.8
%   Good!
%
%   BUT g(8) must equal h(8):
%   g(8) = 1.6, h(8) = -0.2*8+3.2 = 1.6 YES!
%
%   FINAL: add slight curvature to h:
%   h(x) = -0.005x^2 - 0.15x + 3.05
%   h(5) = -0.125-0.75+3.05 = 2.175  want 2.2, close
%   h(8) = -0.32-1.2+3.05 = 1.53    want 1.6, off
%   Let me solve exactly:
%   -0.005*25+5b+c = 2.2 => 5b+c=2.325
%   -0.005*64+8b+c = 1.6 => 8b+c=1.92
%   3b=-0.405 => b=-0.135, c=2.325+0.675=3.0
%   h(x) = -0.005x^2-0.135x+3.0
%   h(0)=3.0, h(-3)=-0.045+0.405+3.0=3.36
%   h(2)=-0.02-0.27+3.0=2.71
%   vertex at x=-0.135/0.01=-13.5 (far left, good, monotone decreasing)
%   h(10)=-0.5-1.35+3.0=1.15
%   This is gently curved. Good.
% ============================================================

% Redefine clean functions:
% f(x) = 0.025*x^2 + 0.125*x + 0.95
% g(x) = 0.05*x^2 - 0.45*x + 2.0
% h(x) = -0.005*x^2 - 0.135*x + 3.0
%
% Intersections:
% f(2)=g(2)=1.3  [x=a]
% f(5)=h(5)=2.2  [x=b] -- let me verify: f(5)=0.625+0.625+0.95=2.2,
%   h(5)=-0.125-0.675+3.0=2.2 YES
% g(8)=h(8): g(8)=3.2-3.6+2.0=1.6,
%   h(8)=-0.32-1.08+3.0=1.6 YES

% --- Shaded region FIRST (behind curves) ---

% From x=2 to x=5: upper = f(x), lower = g(x)
\addplot[name path=f_shade, draw=none, domain=2:5, samples=100]
  {0.025*x^2 + 0.125*x + 0.95};
\addplot[name path=g_shade_left, draw=none, domain=2:5, samples=100]
  {0.05*x^2 - 0.45*x + 2.0};
\addplot[gray!30] fill between[of=f_shade and g_shade_left];

% From x=5 to x=8: upper = h(x), lower = g(x)
\addplot[name path=h_shade, draw=none, domain=5:8, samples=100]
  {-0.005*x^2 - 0.135*x + 3.0};
\addplot[name path=g_shade_right, draw=none, domain=5:8, samples=100]
  {0.05*x^2 - 0.45*x + 2.0};
\addplot[gray!30] fill between[of=h_shade and g_shade_right];

% --- Curves with arrow tips ---

% f(x) blue: increasing quadratic
\addplot[blue, very thick, domain=-3.5:9.8, samples=200]
  {0.025*x^2 + 0.125*x + 0.95};
\draw[-{Stealth[length=3.5mm,width=2.5mm]}, blue, very thick]
  (axis cs:9.6,{0.025*9.6^2+0.125*9.6+0.95})
  -- (axis cs:10.2,{0.025*10.2^2+0.125*10.2+0.95});

% g(x) dark green: U-shaped quadratic
\addplot[green!50!black, very thick, domain=-2.5:10, samples=200]
  {0.05*x^2 - 0.45*x + 2.0};
% Arrow at left end
\draw[-{Stealth[length=3.5mm,width=2.5mm]}, green!50!black, very thick]
  (axis cs:-2,{0.05*4+0.9+2.0})
  -- (axis cs:-2.5,{0.05*6.25+1.125+2.0});
% Arrow at right end
\draw[-{Stealth[length=3.5mm,width=2.5mm]}, green!50!black, very thick]
  (axis cs:9.5,{0.05*90.25-0.45*9.5+2.0})
  -- (axis cs:10,{0.05*100-0.45*10+2.0});

% h(x) red: gently decreasing with curvature
\addplot[red, very thick, domain=-2.5:10.5, samples=200]
  {-0.005*x^2 - 0.135*x + 3.0};
% Arrow at left end
\draw[-{Stealth[length=3.5mm,width=2.5mm]}, red, very thick]
  (axis cs:-2,{-0.005*4+0.27+3.0})
  -- (axis cs:-2.5,{-0.005*6.25+0.3375+3.0});
% Arrow at right end
\draw[-{Stealth[length=3.5mm,width=2.5mm]}, red, very thick]
  (axis cs:10,{-0.005*100-1.35+3.0})
  -- (axis cs:10.5,{-0.005*110.25-0.135*10.5+3.0});

% Arrow at left end of f
\draw[-{Stealth[length=3.5mm,width=2.5mm]}, blue, very thick]
  (axis cs:-3,{0.025*9-0.375+0.95})
  -- (axis cs:-3.5,{0.025*12.25-0.4375+0.95});

% --- Vertical lines at x=a, x=b, x=c ---
\draw[gray, thin] (axis cs:2,0) -- (axis cs:2,1.3);
\draw[gray, thin] (axis cs:5,0) -- (axis cs:5,2.2);
\draw[gray, thin] (axis cs:8,0) -- (axis cs:8,1.6);

% --- x-axis labels ---
\node[below, font=\large] at (axis cs:2,-0.15) {$x{=}a$};
\node[below, font=\large] at (axis cs:5,-0.15) {$x{=}b$};
\node[below, font=\large] at (axis cs:8,-0.15) {$x{=}c$};

% --- Function labels ---
\node[blue, font=\Large\bfseries, anchor=west] at (axis cs:10.5,4.8) {$f(x)$};
\node[green!50!black, font=\Large\bfseries, anchor=west] at (axis cs:10.5,3.3) {$g(x)$};
\node[red, font=\Large\bfseries, anchor=west] at (axis cs:10.5,1.8) {$h(x)$};

\end{axis}
\end{tikzpicture}
\end{document}
