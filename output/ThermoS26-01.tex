\documentclass[12pt]{article}

% Import macros
\usepackage{cmbright}
\usepackage{mathtools}
\usepackage{amsmath}
\usepackage{amssymb}
\usepackage{amsfonts}
\usepackage{amsthm}
\usepackage{graphicx}
\usepackage{tikz}
\usepackage{pgfplots}
\pgfplotsset{compat=1.18}
\usepackage{hyperref}
\usepackage[shortlabels]{enumitem}

% Page setup
\usepackage{fancyhdr}
\pagestyle{fancy}
\setlength{\headheight}{15pt}
\lhead{Math Thermo - Class 01}
\chead{}
\rhead{01/20/2026}
\cfoot{\thepage}

% Theorem environments
\theoremstyle{plain}
\newtheorem{theorem}{Theorem}[section]
\newtheorem{lemma}[theorem]{Lemma}
\newtheorem{proposition}[theorem]{Proposition}
\newtheorem{corollary}[theorem]{Corollary}
\newtheorem{postulate}{Postulate}

\theoremstyle{definition}
\newtheorem{definition}[theorem]{Definition}
\newtheorem{example}[theorem]{Example}

\theoremstyle{remark}
\newtheorem{remark}[theorem]{Remark}

% Thermodynamic symbols (consistent with thermoSymbols.tex)
\newcommand{\Int}{U}           % Internal energy
\newcommand{\Ent}{S}           % Entropy
\newcommand{\Vol}{V}           % Volume
\newcommand{\Enth}{H}          % Enthalpy
\newcommand{\Helm}{A}          % Helmholtz free energy
\newcommand{\Gibbs}{G}         % Gibbs free energy
\newcommand{\Temp}{T}          % Temperature
\newcommand{\Press}{P}         % Pressure
\newcommand{\ChemPot}{\mu}     % Chemical potential

% Partial derivatives
\newcommand{\pder}[2]{\frac{\partial #1}{\partial #2}}
\newcommand{\pderT}[3]{\left(\frac{\partial #1}{\partial #2}\right)_{#3}}

% Sets
\newcommand{\R}{\mathbb{R}}
\newcommand{\N}{\mathbb{N}}

% Differential operator
\newcommand{\dd}{\mathrm{d}}

% Common abbreviations
\newcommand{\ie}{i.e.\ }
\newcommand{\eg}{e.g.\ }

\title{Mathematical Thermodynamics\\
       Class 01: Non-Equilibrium Thermodynamics}
\author{Course Notes}
\date{January 20, 2026}

\begin{document}

\maketitle

\section{Course Plan}

This course covers the following topics:

\begin{enumerate}
\item Equilibrium Thermodynamics
\item Statistical Mechanics
\item Kinetic Theory of Gases
\item Conservation Laws in Continuous Systems
\item Entropy Production
\item Onsager's Principle
\item Applications
\end{enumerate}

\section{Equilibrium Thermodynamics}

\subsection{Isolated Systems and Equilibrium States}

\begin{definition}[Isolated System]
\label{def:isolated-system}
An \textbf{isolated system} is a collection of matter that shares no information with the outside world.
\end{definition}

\begin{figure}[h]
\centering
\includegraphics[width=0.6\textwidth]{figures/ThermoS26-01-fig-1.pdf}
\caption{Isolated system with rigid, adiabatic walls. The system boundary prevents exchange of energy, matter, and volume with the surroundings.}
\label{fig:isolated-system}
\end{figure}

\begin{remark}
The rigid walls that isolate the system from the rest of the universe allow no change in volume nor exchange of matter, and no exchange of energy.
\end{remark}

\begin{postulate}[Equilibrium States]
\label{post:equilibrium-states}
There are particular states, called \textbf{equilibrium states} of an isolated system that, macroscopically, are characterized by:
\begin{itemize}
\item $\Int$, the internal energy of the system, $\Int > 0$,
\item $\Vol$, the volume of the system, $\Vol \gg 0$,
\item $N_1, \ldots, N_r$, the mole numbers of the $r$ chemical components of the system, $N_i > 0$.
\end{itemize}
\end{postulate}

\subsection{Composite Systems}

\begin{definition}[Composite System]
\label{def:composite-system}
A \textbf{composite system} is the union of two isolated systems (chemical components) that can exchange volume, matter (chemical components) and/or energy, however no volume, matter, or energy is exchanged with the outside.
\end{definition}

\begin{figure}[h]
\centering
\includegraphics[width=0.8\textwidth]{figures/ThermoS26-01-fig-2.pdf}
\caption{Composite system with diathermal wall separating subsystems $\alpha$ and $\beta$. The diathermal wall allows energy exchange but prevents matter and volume exchange between subsystems.}
\label{fig:composite-system}
\end{figure}

\begin{definition}[Diathermal Wall]
\label{def:diathermal-wall}
A \textbf{diathermal wall} in a composite system is one which separates two otherwise isolated systems and allows for the exchange of energy, but not matter or volume.
\end{definition}

\subsection{The Entropy Postulates}

\begin{postulate}[Existence of Entropy]
\label{post:entropy-existence}
There exists a function $\Ent$ for an isolated system called its \textbf{entropy}, defined for the system at equilibrium and depending on $(\Int, \Vol, N_1, \ldots, N_r)$, \ie
\[
\widetilde{\Ent} = \widetilde{\Ent}(\Int, \Vol, N_1, \ldots, N_r).
\]

The entropy for a composite system is the sum of the entropy functions, $\widetilde{\Ent}^{\alpha}$ and $\widetilde{\Ent}^{\beta}$, for the respective subsystems, \ie
\[
\widetilde{\Ent} = \widetilde{\Ent}^{\alpha} + \widetilde{\Ent}^{\beta} = \widetilde{\Ent}^{\alpha}(\Int^{\alpha}, \Vol^{\alpha}, \widetilde{N}^{\alpha}) + \widetilde{\Ent}^{\beta}(\Int^{\beta}, \Vol^{\beta}, \widetilde{N}^{\beta}).
\]

The entropy of an isolated system is \textbf{homogeneous of degree 1}, meaning:
\[
\widetilde{\Ent}(\lambda \Int, \lambda \Vol, \lambda N_1, \ldots, \lambda N_r) = \lambda \widetilde{\Ent}(\Int, \Vol, N_1, \ldots, N_r)
\]
for any $\lambda > 0$.
\end{postulate}

\begin{postulate}[Properties of Entropy]
\label{post:entropy-properties}
The entropy of an isolated system is a \textbf{concave}, \textbf{twice continuously differentiable}, \textbf{positive function} over its (convex) domain:
\[
\Sigma_{\Int} \subseteq \underbrace{[0, \infty)}_{(\Int)} \times \underbrace{[0, \infty)}_{(\Vol)} \times \underbrace{[0, \infty)}_{(N_1)} \times \cdots \times \underbrace{[0, \infty)}_{(N_r)}.
\]

The entropy is a \textbf{monotonically increasing function} of $\Int$. In particular:
\[
\pderT{\widetilde{\Ent}}{\Int}{\Vol, N_1, \ldots, N_r} > 0, \quad \forall (\Int, \Vol, N) \in \Sigma_{\Int}.
\]
\end{postulate}

\subsection{Fundamental Relation in the Energy Representation}

\begin{theorem}[Energy Representation]
\label{thm:energy-representation}
Suppose that Postulates \ref{post:equilibrium-states}--\ref{post:entropy-properties} hold for an isolated system. Then, there is a function:
\[
\widetilde{\Int} = \widetilde{\Int}(\Ent, \Vol, N_1, \ldots, N_r)
\]
and a convex domain of definition:
\[
\Sigma_{\Ent} \subseteq \underbrace{[0, \infty)}_{(\Ent)} \times \underbrace{[0, \infty)}_{(\Vol)} \times \underbrace{[0, \infty)}_{(N_1)} \times \cdots \times \underbrace{[0, \infty)}_{(N_r)}
\]
that satisfies:
\begin{enumerate}[(i)]
\item $\widetilde{\Int}\bigl(\widetilde{\Ent}(\Int, \Vol, N), \Vol, N\bigr) = \Int$, $\forall (\Int, \Vol, N) \in \Sigma_{\Int}$
\item $\widetilde{\Ent}\bigl(\widetilde{\Int}(\Ent, \Vol, N), \Vol, N\bigr) = \Ent$, $\forall (\Ent, \Vol, N) \in \Sigma_{\Ent}$
\end{enumerate}

Moreover, $\widetilde{\Int}$ is a \textbf{twice continuously differentiable convex function} with the property:
\[
\pderT{\widetilde{\Int}}{\Ent}{\Vol, N_1, \ldots, N_r} > 0, \quad \forall (\Ent, \Vol, N) \in \Sigma_{\Ent}.
\]
\end{theorem}

\begin{example}[Unary Material]
\label{ex:unary-material}
For a unary material ($r = 1$), consider the function:
\[
\widetilde{\Ent}(\Int, \Vol, N) = \left(\frac{N\Vol\Int R^2}{v_0 \theta}\right)^{1/3}, \quad \Sigma_{\Vol} = [0, \infty]^3
\]
where $R$, $v_0$, $\theta > 0$ are constants.

The corresponding energy representation is:
\[
\widetilde{\Int}(\Ent, \Vol, N) = \frac{\Ent^3 v_0 \theta}{N\Vol R^2}.
\]
\end{example}

\subsection{Temperature}

\begin{definition}[Temperature]
\label{def:temperature}
The \textbf{temperature} of an equilibrium isolated system is defined as:
\[
\Temp_0(\Ent, \Vol, N_1, \ldots, N_r) \equiv \pderT{\widetilde{\Int}}{\Ent}{\Vol, N_1, \ldots, N_r} = \left(\pderT{\widetilde{\Ent}}{\Int}{\Vol, N_1, \ldots, N_r}\right)^{-1}.
\]
\end{definition}

\subsection{Equilibrium Condition}

\begin{postulate}[Maximum Entropy Principle]
\label{post:maximum-entropy}
Equilibrium of a composite system is a state where subsystems are in equilibrium such that:
\[
\Ent = \Ent^{\alpha} + \Ent^{\beta}
\]
is at its maximum possible value.
\end{postulate}

\begin{remark}
This postulate forms the basis for deriving equilibrium conditions in composite systems. Further mathematical derivations and proofs will follow in subsequent lectures.
\end{remark}

\section{Notes for Future Lectures}

The material presented here establishes the foundational postulates of equilibrium thermodynamics. Future lectures will explore:

\begin{itemize}
\item Derivation of equilibrium conditions from the maximum entropy principle
\item Thermodynamic potentials and Legendre transforms
\item Maxwell relations and thermodynamic identities
\item Connection to statistical mechanics
\item Applications to specific systems
\end{itemize}

\end{document}
