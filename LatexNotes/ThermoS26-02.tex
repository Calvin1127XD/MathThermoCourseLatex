\ifdefined\THERMONOTESMAIN
    \providecommand{\THERMOEND}{}
\else
\documentclass[11pt]{book}

\usepackage{cmbright}
\usepackage{mathtools}

\usepackage{pgfplots}
\usepgfplotslibrary{patchplots}
\pgfplotsset{compat=1.15}

\usepackage{color}
\usepackage{xcolor}
\usepackage{subfigure}

\usepackage{amssymb}
\usepackage{amsmath}
\usepackage{amsfonts}
\usepackage{amsthm} 
\usepackage{mdframed}
\usepackage{bbm}
\usepackage{listings}

\usepackage[shortlabels]{enumitem}

\usepackage{tikz}
\usepackage{tikz-3dplot}
\usepackage{xkeyval,tkz-base}

\usetikzlibrary{patterns}
\usetikzlibrary{arrows,arrows.meta,shapes,decorations,automata}
\usetikzlibrary{backgrounds,petri,matrix}
\usetikzlibrary{intersections}
\usetikzlibrary{calc}
\usetikzlibrary{positioning}
\usetikzlibrary{shapes.geometric}
\usetikzlibrary{shapes.arrows}
\usetikzlibrary{matrix}
\usetikzlibrary{positioning}
\usetikzlibrary{3d,calc}

\usepackage{multirow}
\usepackage{array}
\usepackage{algorithm}
\usepackage{algpseudocode}

\usepackage{graphicx}
\usepackage{latexsym}
\usepackage{booktabs}
\usepackage{longtable}

%\usepackage{aligned-overset}

\numberwithin{equation}{section}

\usepackage{lastpage}
\usepackage{fancyhdr}
\pagestyle{fancy}
\setlength{\headheight}{14pt}
\lhead{Mathematical Thermodynamics}
\chead{}
\rhead{Page \thepage}
\cfoot{Page\ \thepage\ of\ \protect\pageref{LastPage}}
\setcounter{section}{-1}

% Theorem environments
\theoremstyle{plain}
\newtheorem{theorem}{Theorem}[section]
\newtheorem{lemma}[theorem]{Lemma}
\newtheorem{proposition}[theorem]{Proposition}
\newtheorem{corollary}[theorem]{Corollary}

\theoremstyle{definition}
\newtheorem{definition}[theorem]{Definition}
\newtheorem{example}[theorem]{Example}
\newtheorem{exercise}[theorem]{Exercise}

\theoremstyle{remark}
\newtheorem{remark}[theorem]{Remark}
\newtheorem{note}[theorem]{Note}

% Hyperref setup
\usepackage{hyperref}
\usepackage{nameref}
\hypersetup{
    hypertexnames=true,
    breaklinks=false,
    colorlinks=true,
    linkcolor=blue,
    citecolor=blue,
    urlcolor=blue
}

\makeatletter
  \newenvironment{dedication}
  {
     \cleardoublepage
     \thispagestyle{empty}
     \vspace*{\stretch{1}}
     \hfill\begin{minipage}[t]{0.66\textwidth}
     \raggedright
  }%
  {
     \end{minipage}
     \vspace*{\stretch{3}}
     \clearpage
  }
  \newenvironment{varsubequations}[1]
  {%
    \addtocounter{equation}{-1}%
    \begin{subequations}
    \renewcommand{\theparentequation}{#1}%
    \def\@currentlabel{#1}%
  }
  {%
    \end{subequations}\ignorespacesafterend
  }
\makeatother

% ====================================
% THERMODYNAMIC STATE VARIABLES
% ====================================

% Extensive properties
\newcommand{\Int}{U}           % Internal energy
\newcommand{\Ent}{S}           % Entropy
\newcommand{\Vol}{V}           % Volume
\newcommand{\Enth}{H}          % Enthalpy H = U + PV
\newcommand{\Helm}{A}          % Helmholtz free energy A = U - TS
\newcommand{\Gibbs}{G}         % Gibbs free energy G = H - TS
\newcommand{\GrandPot}{\Omega} % Grand potential

% Intensive properties
\newcommand{\Temp}{T}          % Temperature
\newcommand{\Press}{P}         % Pressure
\newcommand{\ChemPot}{\mu}     % Chemical potential
\newcommand{\Mass}{m}          % Mass
\newcommand{\Moles}{n}         % Number of moles

% Specific properties (per unit mass)
\newcommand{\sint}{u}          % Specific internal energy
\newcommand{\sent}{s}          % Specific entropy
\newcommand{\svol}{v}          % Specific volume
\newcommand{\senth}{h}         % Specific enthalpy

% Heat and work
\newcommand{\Heat}{Q}          % Heat transfer
\newcommand{\Work}{W}          % Work
\newcommand{\dQ}{\delta Q}     % Inexact differential of heat
\newcommand{\dW}{\delta W}     % Inexact differential of work

% Thermodynamic processes
\newcommand{\iso}{\text{iso}}
\newcommand{\rev}{\text{rev}}
\newcommand{\irrev}{\text{irrev}}

% ====================================
% MATHEMATICAL OPERATORS
% ====================================

\DeclareMathOperator{\grad}{grad}
\DeclareMathOperator{\dive}{div}
\DeclareMathOperator{\curl}{curl}
\DeclareMathOperator{\Span}{span}
\DeclareMathOperator{\rank}{rank}
\DeclareMathOperator{\trace}{tr}
\DeclareMathOperator{\sgn}{sgn}
\DeclareMathOperator{\diam}{diam}

% Partial derivatives
\newcommand{\pder}[2]{\frac{\partial #1}{\partial #2}}
\newcommand{\pdertwo}[2]{\frac{\partial^2 #1}{\partial #2^2}}
\newcommand{\pdermix}[3]{\frac{\partial^2 #1}{\partial #2 \partial #3}}

% Thermodynamic partial derivatives
\newcommand{\pderT}[3]{\left(\frac{\partial #1}{\partial #2}\right)_{#3}}

% Total derivatives
\newcommand{\dder}[2]{\frac{d #1}{d #2}}
\newcommand{\ddertwo}[2]{\frac{d^2 #1}{d #2^2}}

% Differential operators
\newcommand{\dd}{\mathrm{d}}
\newcommand{\dif}{\mathrm{d}}

% ====================================
% VECTOR AND MATRIX NOTATION
% ====================================

% Bold vectors
\newcommand{\bfv}[1]{\boldsymbol{#1}}
\newcommand{\bfx}{\boldsymbol{x}}
\newcommand{\bfy}{\boldsymbol{y}}
\newcommand{\bfz}{\boldsymbol{z}}
\newcommand{\bfu}{\boldsymbol{u}}
\newcommand{\bfv}{\boldsymbol{v}}
\newcommand{\bfw}{\boldsymbol{w}}
\newcommand{\bff}{\boldsymbol{f}}
\newcommand{\bfg}{\boldsymbol{g}}
\newcommand{\bfF}{\boldsymbol{F}}
\newcommand{\bfn}{\boldsymbol{n}}
\newcommand{\bfe}{\boldsymbol{e}}
\newcommand{\bfzero}{\boldsymbol{0}}

% Matrix spaces
\newcommand{\Rn}{\mathbb{R}^n}
\newcommand{\Rm}{\mathbb{R}^m}
\newcommand{\Rmn}{\mathbb{R}^{m \times n}}

% ====================================
% SETS AND SPACES
% ====================================

\newcommand{\R}{\mathbb{R}}    % Real numbers
\newcommand{\C}{\mathbb{C}}    % Complex numbers
\newcommand{\N}{\mathbb{N}}    % Natural numbers
\newcommand{\Z}{\mathbb{Z}}    % Integers
\newcommand{\Q}{\mathbb{Q}}    % Rationals

% Function spaces
\newcommand{\Lp}[1]{L^{#1}}
\newcommand{\Hone}{H^1}
\newcommand{\Ck}[1]{C^{#1}}

% ====================================
% SPECIAL SYMBOLS
% ====================================

% Indicator/characteristic function
\newcommand{\ind}{\mathbbm{1}}

% Inner product
\newcommand{\inner}[2]{\langle #1, #2 \rangle}

% Norm
\newcommand{\norm}[1]{\left\| #1 \right\|}
\newcommand{\abs}[1]{\left| #1 \right|}

% Set notation
\newcommand{\set}[1]{\left\{ #1 \right\}}
\newcommand{\setdef}[2]{\left\{ #1 \,:\, #2 \right\}}

% Ceiling and floor
\newcommand{\ceil}[1]{\left\lceil #1 \right\rceil}
\newcommand{\floor}[1]{\left\lfloor #1 \right\rfloor}

% ====================================
% MATHCAL AND MATHSF LETTERS
% ====================================

% Calligraphic letters for sets and spaces
\newcommand{\calA}{\mathcal{A}}
\newcommand{\calB}{\mathcal{B}}
\newcommand{\calC}{\mathcal{C}}
\newcommand{\calD}{\mathcal{D}}
\newcommand{\calE}{\mathcal{E}}
\newcommand{\calF}{\mathcal{F}}
\newcommand{\calG}{\mathcal{G}}
\newcommand{\calH}{\mathcal{H}}
\newcommand{\calI}{\mathcal{I}}
\newcommand{\calJ}{\mathcal{J}}
\newcommand{\calK}{\mathcal{K}}
\newcommand{\calL}{\mathcal{L}}
\newcommand{\calM}{\mathcal{M}}
\newcommand{\calN}{\mathcal{N}}
\newcommand{\calO}{\mathcal{O}}
\newcommand{\calP}{\mathcal{P}}
\newcommand{\calQ}{\mathcal{Q}}
\newcommand{\calR}{\mathcal{R}}
\newcommand{\calS}{\mathcal{S}}
\newcommand{\calT}{\mathcal{T}}
\newcommand{\calU}{\mathcal{U}}
\newcommand{\calV}{\mathcal{V}}
\newcommand{\calW}{\mathcal{W}}
\newcommand{\calX}{\mathcal{X}}
\newcommand{\calY}{\mathcal{Y}}
\newcommand{\calZ}{\mathcal{Z}}

% Sans-serif letters for matrices and operators
\newcommand{\msfA}{\mathsf{A}}
\newcommand{\msfB}{\mathsf{B}}
\newcommand{\msfC}{\mathsf{C}}
\newcommand{\msfD}{\mathsf{D}}
\newcommand{\msfE}{\mathsf{E}}
\newcommand{\msfF}{\mathsf{F}}
\newcommand{\msfG}{\mathsf{G}}
\newcommand{\msfH}{\mathsf{H}}
\newcommand{\msfI}{\mathsf{I}}
\newcommand{\msfJ}{\mathsf{J}}
\newcommand{\msfK}{\mathsf{K}}
\newcommand{\msfL}{\mathsf{L}}
\newcommand{\msfM}{\mathsf{M}}
\newcommand{\msfN}{\mathsf{N}}
\newcommand{\msfO}{\mathsf{O}}
\newcommand{\msfP}{\mathsf{P}}
\newcommand{\msfQ}{\mathsf{Q}}
\newcommand{\msfR}{\mathsf{R}}
\newcommand{\msfS}{\mathsf{S}}
\newcommand{\msfT}{\mathsf{T}}
\newcommand{\msfU}{\mathsf{U}}
\newcommand{\msfV}{\mathsf{V}}
\newcommand{\msfW}{\mathsf{W}}
\newcommand{\msfX}{\mathsf{X}}
\newcommand{\msfY}{\mathsf{Y}}
\newcommand{\msfZ}{\mathsf{Z}}

% ====================================
% COMMON ABBREVIATIONS
% ====================================

\newcommand{\ie}{i.e.\ }
\newcommand{\eg}{e.g.\ }
\newcommand{\cf}{cf.\ }
\newcommand{\etc}{etc.\ }
\newcommand{\vs}{vs.\ }
\newcommand{\resp}{resp.\ }
\newcommand{\etal}{et al.\ }


    \providecommand{\THERMOEND}{\end{document}}
    \begin{document}
\fi

\chapter{Math Thermo}
\label{chap:thermo-s26-02}

\begin{center}
Math Thermo\\
Class 02\\
01/22/2026
\end{center}

\graphicspath{{Figures/}}

\begin{figure}[h]
    \centering
    \includegraphics[width=0.5\textwidth]{ThermoS26-02-fig-1.pdf}
    \caption{Coordinate axes labeled $N$, $V$, and $S$.}
\end{figure}

\begin{example}
\textbf{(2.1)}\label{ex:thermo-s26-02-2-1} Suppose that, for an isolated unary fluid
\[
\tilde{U} = \left(\frac{v_0 \theta}{R^2}\right) \frac{\tilde{S}^3}{N V}, \quad \Sigma_s \subset [0,\infty)^3.
\]
A explicit expression of the form $\tilde{U} = \tilde{U}(S,V,N)$ is called a fundamental relation.
Here $v_0$, $\theta$ and $R$ are positive constants.
The units of $U$ and $\tilde{U}$ are
\[
[U] = \text{Joules}.
\]
The units of entropy, $S$ and $\tilde{S}$, are
\[
[S] = \frac{\text{Joules}}{\text{degree Kelvin}} = \frac{J}{K}.
\]
The units of volume, $V$, are
\[
[V] = \text{meters}^3.
\]
The units of $N$ are
\[
[N] = \text{moles}.
\]
Of course, it is easy to see that
\[
\tilde{S} = \left(\frac{N V U R^2}{v_0 \theta}\right)^{1/3}, \quad \Sigma_U = [0,\infty)^3.
\]
\end{example}

An explicit function of the form
\[
\tilde{S} = \tilde{S}(U,V,N)
\]
is also called a fundamental relation.

In any case, the temperature which has units
\[
[T] = \text{degrees Kelvin} = K,
\]
is
\[
T_U(S,V,N) = \frac{\partial \tilde{U}}{\partial S} = \frac{3 v_0 \theta}{R^2} \frac{S^2}{N V}.
\]

This expression is called an equation of state.

Observe that
\[
T_U(\lambda S,\lambda V,\lambda N) = T_U(S,V,N),
\]
that is $T_U$ is homogeneous of order $0$.
This property is true of every equation of state.

Now,
\[
\left(\frac{\partial \tilde{S}}{\partial U}\right)^{-1} =
\left[\frac{1}{3} \left(\frac{N V U R^2}{v_0 \theta}\right)^{-2/3} \frac{N V R^2}{v_0 \theta}\right]
\]
\[
= \frac{3 v_0 \theta}{N V R^2} \left(\frac{N V U R^2}{v_0 \theta}\right)^{2/3}
\]
\[
= \frac{3 v_0 \theta}{N V R^2} \tilde{S}(U,V,N)
\]
\[
= T_S(U,V,N),
\]
Clearly
\[
T_U(\tilde{S}(U,V,N),V,N) = T_S(U,V,N),
\]
as claimed in Theorem (1.7).

The pressure satisfies the equation of state
\[
P_S(S,V,N) = - \frac{v_0 \theta}{R^2} \frac{S^3}{N} \frac{1}{V^2}
\]
\[
= \frac{v_0 \theta}{R^2} \frac{S^3}{N V^2},
\]
which is also homogeneous of order zero.
We leave it as an exercise for the reader to show that
\[
P_S(U,V,N) = P_S(\tilde{S}(U,V,N),V,N).
\]
Finally, the chemical potential is
\[
\mu_S(S,V,N) = - \frac{v_0 \theta}{R^2} \frac{S^3}{V N^2},
\]
which is clearly homogeneous of degree zero.
The reader can show that
\[
\mu_S(U,V,N) = \mu_S(\tilde{S}(U,V,N),V,N).
\]

Recall we have assumed with the Postulate II that $\tilde{S}$ is homogeneous of degree one. We also
must have the following, as suggested by the example.

\begin{theorem}
\textbf{(2.2)}\label{thm:thermo-s26-02-2-2} $\tilde{U}$ is homogeneous of degree one when written as
\[
\tilde{U} = \tilde{U}(S,V,\vec{N}).
\]
Further, $T_U(S,V,\vec{N})$, $P_U(S,V,\vec{N})$ and $\mu_{U_i}(S,V,\vec{N})$
the equations of state, are homogeneous of degree zero. I mean:
\[
T_U(\lambda S,\lambda V,\lambda \vec{N}) = T_U(S,V,\vec{N})
\]
for any $\lambda > 0$, and similarly for $P$ and $\mu_{U_i}$, $i=1,\ldots,r$.
Likewise $T_S(U,V,\vec{N})$, $P_S(U,V,\vec{N})$, and $\mu_{S_i}(U,V,\vec{N})$ are
homogeneous of degree zero.
\end{theorem}

\begin{proof}
Fix $V \in [0,\infty)$ and $\vec{N} \in [0,\infty)^r$. $\tilde{S}$ is
a monotonically increasing function of $U \in [0,\infty)$.
For each $S \in [0,\infty)$ there exists a unique $U \in [0,\infty)$
such that
\begin{equation}
S = \tilde{S}(U,V,\vec{N}),
\tag{2.1}
\label{eq:thermo-s26-02-2-1}
\end{equation}
where we assume, for simplicity, that $\Sigma_U = [0,\infty)^{r+2}$.
Then,
\begin{equation}
\tilde{U}(S,V,\vec{N}) = \tilde{U}(\tilde{S}(U,V,\vec{N}),V,\vec{N}), \quad \forall (S,V,\vec{N}) \in \Sigma_s.
\tag{2.2}
\label{eq:thermo-s26-02-2-2}
\end{equation}
Let $\lambda > 0$ be arbitrary; then (2.1) and (2.2) imply
\begin{equation}
\tilde{U}(\lambda S,\lambda V,\lambda \vec{N}) =
\tilde{U}(\lambda \tilde{S}(U,V,\vec{N}),\lambda V,\lambda \vec{N})
\tag{2.3}
\label{eq:thermo-s26-02-2-3}
\end{equation}
Since $\tilde{S}$ is homogeneous of degree $1$, it follows that
\begin{equation}
\lambda \tilde{S}(U,V,\vec{N}) = \tilde{S}(\lambda U,\lambda V,\lambda \vec{N})
\tag{2.4}
\label{eq:thermo-s26-02-2-4}
\end{equation}
Also, recall that, generically,
\begin{equation}
\tilde{U}(\tilde{S}(U,V,\vec{N}),V,\vec{N}) = U
\tag{2.5}
\label{eq:thermo-s26-02-2-5}
\end{equation}
because of inverse relations. Combining (2.3) -- (2.5) we have
\[
\tilde{U}(\lambda S,\lambda V,\lambda \vec{N}) = \lambda U
\]
\[
= \lambda \tilde{U}(\tilde{S}(U,V,\vec{N}),V,\vec{N})
\]
\[
= \lambda \tilde{U}(S,V,\vec{N}).
\]
\end{proof}

\section{Path (Contour) Integrals}

\begin{definition}
\textbf{(2.4)}\label{def:thermo-s26-02-2-4} Suppose that $D \subset \mathbb{R}^n$ is open.
A function $\vec{\gamma}:[a,b] \to D$ is called a path
(or contour) iff $\vec{\gamma}$ is continuous and
piecewise smooth. The domain $D$ is called path-connected
iff for any two distinct points $\vec{a},\vec{b} \in D$
there is a path $\vec{\gamma}:[a,b] \to D$ such that
$\vec{\gamma}(a) = \vec{a}$ and $\vec{\gamma}(b) = \vec{b}$.
\end{definition}

$D$ is called simply-connected iff it is
(1) path-connected and (2) a path can
be continuously deformed to a point, i.e.,
there are no holes. $D$ is called convex iff
for every pair $\vec{a},\vec{b} \in D$ the point
\[
\vec{x}(t) = \vec{a}(1-t) + \vec{b} t \in D
\]
for all $t \in [0,1]$.

\begin{definition}
\textbf{(2.5)}\label{def:thermo-s26-02-2-5} Let $\vec{F}: D \to \mathbb{R}^n$ be a $C^1$
function, i.e., $\vec{F} \in C^1(D;\mathbb{R}^n)$. Let
$\vec{\gamma}:[a,b] \to D$ be a path in $D$ which is assumed to be simple-connected.
Then the path integral $\int_{\vec{\gamma}} \vec{F}(\vec{x})\cdot d\vec{x}$
is defined via
\begin{equation}
\int_{\vec{\gamma}} \vec{F}(\vec{x})\cdot d\vec{x} :=
\int_a^b \vec{F}(\vec{\gamma}(\tau)) \cdot \vec{\gamma}'(\tau)\, d\tau.
\tag{2.6}
\label{eq:thermo-s26-02-2-6}
\end{equation}
We will also use the notation
\[
\int_{\vec{\gamma}} \vec{F}(\vec{x})\cdot d\vec{x} =
\int_{\vec{\gamma}} F_1(\vec{x})\,dx_1 + \cdots + F_n(\vec{x})\,dx_n
\]
\end{definition}

\begin{definition}
\textbf{(2.6)}\label{def:thermo-s26-02-2-6} Let $D \subset \mathbb{R}^n$ be a simply connected
open set. A path $\vec{\gamma}:[a,b] \to D$ is called
closed iff
\[
\vec{\gamma}(a) = \vec{\gamma}(b).
\]
A closed path is called simple iff it does
not intersect itself except at $\tau=a$ and $\tau=b$,
i.e., for every $c \in (a,b)$
\[
\vec{\gamma}(c) \ne \vec{\gamma}(\tau), \quad \tau \in [a,c) \cup (c,b]
\]
\end{definition}

\begin{theorem}
\textbf{(2.7)}\label{thm:thermo-s26-02-2-7} Let $D \subset \mathbb{R}^n$ be an open, simply-
connected set. Assume $\vec{\gamma}:[a,b] \to D$ is a simple
path. If $\vec{x}:[c,d] \to D$ is a path in $D$,
with the property that
\[
\vec{x}(c) = \vec{\gamma}(a),\quad \vec{x}(d) = \vec{\gamma}(b),
\]
and
\[
\operatorname{Range}(\vec{x}) = \operatorname{Range}(\vec{\gamma}),
\]
then
\[
\int_{\vec{\gamma}} \vec{F}(\vec{x})\cdot d\vec{x} =
\int_{\vec{x}} \vec{F}(\vec{x})\cdot d\vec{x}.
\]
This result guarantees that the path integrals
are parametrization independent.
If $C = \operatorname{Range}(\vec{\gamma}) = \operatorname{Range}(\vec{x})$, then we write
\[
\int_C \vec{F}(\vec{x})\cdot d\vec{x} =
\int_{\vec{\gamma}} \vec{F}(\vec{x})\cdot d\vec{x}.
\]
\end{theorem}

\begin{definition}
\textbf{(2.7)}\label{def:thermo-s26-02-2-7} Let $D \subset \mathbb{R}^n$ be an open, simply
connected set. Suppose that
\[
\int_{\vec{\gamma}_1} \vec{F}(\vec{x})\cdot d\vec{x} =
\int_{\vec{\gamma}_2} \vec{F}(\vec{x})\cdot d\vec{x}
\]
for any two paths $\vec{\gamma}_1:[a,b] \to D$, $\vec{\gamma}_2:[c,d] \to D$ with
\[
\vec{\gamma}_1(a) = \vec{\gamma}_2(a) \quad \text{and} \quad
\vec{\gamma}_1(b) = \vec{\gamma}_2(b).
\]
Then we say that the integral is path-
independent. Note that we are not assuming that
\[
\operatorname{Range}(\vec{\gamma}_1) = \operatorname{Range}(\vec{\gamma}_2).
\]
\end{definition}

\begin{definition}
\textbf{(2.8)}\label{def:thermo-s26-02-2-8} Let $D$ be an open set and
$\vec{F} \in C^1(D;\mathbb{R}^n)$. We say that $\vec{F}$ is conservative
iff there is a function $f \in C^1(D;\mathbb{R})$ such
that
\[
\vec{F}(\vec{x}) = \nabla f(\vec{x}), \quad \forall \vec{x} \in D.
\]
\end{definition}

\begin{theorem}
\textbf{(2.9)}\label{thm:thermo-s26-02-2-9} Let $D$ be an open simply
connected set in $\mathbb{R}^n$. If
$\vec{F} \in C^1(D;\mathbb{R}^n)$ is
conservative, then the integral
\[
\int_{\vec{\gamma}} \vec{F}(\vec{x})\cdot d\vec{x}
\]
is path-independent.
\end{theorem}

\begin{proof}
let $\vec{\gamma}_1:[a_1,b_1] \to D$ and
$\vec{\gamma}_2:[a_2,b_2] \to D$
be paths in $D$ with the same end points, i.e.,
\[
\vec{a} := \vec{\gamma}_1(a_1) = \vec{\gamma}_2(a_2), \quad
\vec{b} := \vec{\gamma}_1(b_1) = \vec{\gamma}_2(b_2).
\]
By the chain rule, for $i=1,2$,
\[
\frac{d}{d\tau} f(\vec{\gamma}_i(\tau)) =
\nabla f(\vec{\gamma}_i(\tau)) \cdot \vec{\gamma}_i'(\tau)
\]
\[
= \vec{F}(\vec{\gamma}_i(\tau)) \cdot \vec{\gamma}_i'(\tau).
\]
Thus,
\[
\int_{\vec{\gamma}_i} \vec{F}(\vec{x})\cdot d\vec{x} =
\int_{a_i}^{b_i} \vec{F}(\vec{\gamma}_i(\tau)) \cdot \vec{\gamma}_i'(\tau)\, d\tau
\]
\[
= \int_{a_i}^{b_i} \frac{d}{d\tau} f(\vec{\gamma}_i(\tau))\, d\tau
\]
\[
= f(\vec{b}) - f(\vec{a}),
\]
for $i=1,2$.
\end{proof}

We have the following well-known results.

\begin{theorem}
\textbf{(2.10)}\label{thm:thermo-s26-02-2-10} Let $D \subset \mathbb{R}^n$ be a simply-
connected set and suppose that $\vec{F} \in C^1(D;\mathbb{R}^n)$.
The following are equivalent
\begin{enumerate}
    \item $\vec{F}$ is conservative
    \item $\int_{\vec{\gamma}} \vec{F}(\vec{x})\cdot d\vec{x}$ is path independent
    \item $\oint_{\vec{\gamma}} \vec{F}(\vec{x})\cdot d\vec{x} = 0$ for any closed path.
\end{enumerate}
\end{theorem}

Recall, we write, as a shorthand
\[
d\tilde{U} = T_U\,dS - P_U\,dV + \sum_{i=1}^r \mu_{U_i}\,dN_i.
\]
This has the form
\[
F_1\,dx_1 + F_2\,dx_2 + \cdots + F_n\,dx_n = \vec{F}\cdot d\vec{x}
\]
where $F_1 = T_U$, $F_2 = -P_U$, $\ldots$
We say that $\vec{F}\cdot d\vec{x}$ is an exact differential
if $\vec{F}$ is conservative.

Clearly
\[
d\tilde{U} = \nabla \tilde{U} \cdot d\vec{\sigma} \quad (\vec{\sigma} \in \Sigma_s)
\]
is an exact differential, because $\nabla \tilde{U}$ is
conservative, trivially. Thus, the integral
\[
\int_{\vec{\gamma}} d\tilde{U} = \int_{\vec{\gamma}} \nabla \tilde{U} \cdot d\vec{\sigma}
\]
is path independent. If $\vec{\gamma}:[a,b] \to \Sigma_s$
is the path in question, with
\[
\vec{\gamma}(a) = \vec{\sigma}_a, \quad \vec{\gamma}(b) = \vec{\sigma}_b,
\]
then
\[
\int_{\vec{\gamma}} d\tilde{U} = \int_{\vec{\gamma}} \nabla \tilde{U} \cdot d\vec{\sigma} =
\tilde{U}(\vec{\sigma}_b) - \tilde{U}(\vec{\sigma}_a).
\]
We can use any path we want in state
space $\Sigma_s$.

The same is true for
\[
d\tilde{S} = \frac{1}{T_s} dU + \frac{P_s}{T_s} dV - \sum_{i=1}^r \frac{\mu_{s_i}}{T_s} dN_i,
\]
that is
\[
d\tilde{S} = \nabla \tilde{S}\cdot d\vec{\sigma} \quad (\vec{\sigma} \in \Sigma_U)
\]
is an exact differential.

\THERMOEND
