\newif\ifthermosubfile
\ifdefined\THERMONOTESMAIN
  \thermosubfilefalse
\else
  \thermosubfiletrue
\fi

\providecommand{\THERMOEND}{} % safe in both modes

\ifthermosubfile
  \documentclass[11pt]{book}
  \usepackage{cmbright}
\usepackage{mathtools}

\usepackage{pgfplots}
\usepgfplotslibrary{patchplots}
\pgfplotsset{compat=1.15}

\usepackage{color}
\usepackage{xcolor}
\usepackage{subfigure}

\usepackage{amssymb}
\usepackage{amsmath}
\usepackage{amsfonts}
\usepackage{amsthm} 
\usepackage{mdframed}
\usepackage{bbm}
\usepackage{listings}

\usepackage[shortlabels]{enumitem}

\usepackage{tikz}
\usepackage{tikz-3dplot}
\usepackage{xkeyval,tkz-base}

\usetikzlibrary{patterns}
\usetikzlibrary{arrows,arrows.meta,shapes,decorations,automata}
\usetikzlibrary{backgrounds,petri,matrix}
\usetikzlibrary{intersections}
\usetikzlibrary{calc}
\usetikzlibrary{positioning}
\usetikzlibrary{shapes.geometric}
\usetikzlibrary{shapes.arrows}
\usetikzlibrary{matrix}
\usetikzlibrary{positioning}
\usetikzlibrary{3d,calc}

\usepackage{multirow}
\usepackage{array}
\usepackage{algorithm}
\usepackage{algpseudocode}

\usepackage{graphicx}
\usepackage{latexsym}
\usepackage{booktabs}
\usepackage{longtable}

%\usepackage{aligned-overset}

\numberwithin{equation}{section}

\usepackage{lastpage}
\usepackage{fancyhdr}
\pagestyle{fancy}
\setlength{\headheight}{14pt}
\lhead{Mathematical Thermodynamics}
\chead{}
\rhead{Page \thepage}
\cfoot{Page\ \thepage\ of\ \protect\pageref{LastPage}}
\setcounter{section}{-1}

% Theorem environments
\theoremstyle{plain}
\newtheorem{theorem}{Theorem}[section]
\newtheorem{lemma}[theorem]{Lemma}
\newtheorem{proposition}[theorem]{Proposition}
\newtheorem{corollary}[theorem]{Corollary}

\theoremstyle{definition}
\newtheorem{definition}[theorem]{Definition}
\newtheorem{example}[theorem]{Example}
\newtheorem{exercise}[theorem]{Exercise}

\theoremstyle{remark}
\newtheorem{remark}[theorem]{Remark}
\newtheorem{note}[theorem]{Note}

% Hyperref setup
\usepackage{hyperref}
\usepackage{nameref}
\hypersetup{
    hypertexnames=true,
    breaklinks=false,
    colorlinks=true,
    linkcolor=blue,
    citecolor=blue,
    urlcolor=blue
}

\makeatletter
  \newenvironment{dedication}
  {
     \cleardoublepage
     \thispagestyle{empty}
     \vspace*{\stretch{1}}
     \hfill\begin{minipage}[t]{0.66\textwidth}
     \raggedright
  }%
  {
     \end{minipage}
     \vspace*{\stretch{3}}
     \clearpage
  }
  \newenvironment{varsubequations}[1]
  {%
    \addtocounter{equation}{-1}%
    \begin{subequations}
    \renewcommand{\theparentequation}{#1}%
    \def\@currentlabel{#1}%
  }
  {%
    \end{subequations}\ignorespacesafterend
  }
\makeatother

  % ====================================
% THERMODYNAMIC STATE VARIABLES
% ====================================

% Extensive properties
\newcommand{\Int}{U}           % Internal energy
\newcommand{\Ent}{S}           % Entropy
\newcommand{\Vol}{V}           % Volume
\newcommand{\Enth}{H}          % Enthalpy H = U + PV
\newcommand{\Helm}{A}          % Helmholtz free energy A = U - TS
\newcommand{\Gibbs}{G}         % Gibbs free energy G = H - TS
\newcommand{\GrandPot}{\Omega} % Grand potential

% Intensive properties
\newcommand{\Temp}{T}          % Temperature
\newcommand{\Press}{P}         % Pressure
\newcommand{\ChemPot}{\mu}     % Chemical potential
\newcommand{\Mass}{m}          % Mass
\newcommand{\Moles}{n}         % Number of moles

% Specific properties (per unit mass)
\newcommand{\sint}{u}          % Specific internal energy
\newcommand{\sent}{s}          % Specific entropy
\newcommand{\svol}{v}          % Specific volume
\newcommand{\senth}{h}         % Specific enthalpy

% Heat and work
\newcommand{\Heat}{Q}          % Heat transfer
\newcommand{\Work}{W}          % Work
\newcommand{\dQ}{\delta Q}     % Inexact differential of heat
\newcommand{\dW}{\delta W}     % Inexact differential of work

% Thermodynamic processes
\newcommand{\iso}{\text{iso}}
\newcommand{\rev}{\text{rev}}
\newcommand{\irrev}{\text{irrev}}

% ====================================
% MATHEMATICAL OPERATORS
% ====================================

\DeclareMathOperator{\grad}{grad}
\DeclareMathOperator{\dive}{div}
\DeclareMathOperator{\curl}{curl}
\DeclareMathOperator{\Span}{span}
\DeclareMathOperator{\rank}{rank}
\DeclareMathOperator{\trace}{tr}
\DeclareMathOperator{\sgn}{sgn}
\DeclareMathOperator{\diam}{diam}

% Partial derivatives
\newcommand{\pder}[2]{\frac{\partial #1}{\partial #2}}
\newcommand{\pdertwo}[2]{\frac{\partial^2 #1}{\partial #2^2}}
\newcommand{\pdermix}[3]{\frac{\partial^2 #1}{\partial #2 \partial #3}}

% Thermodynamic partial derivatives
\newcommand{\pderT}[3]{\left(\frac{\partial #1}{\partial #2}\right)_{#3}}

% Total derivatives
\newcommand{\dder}[2]{\frac{d #1}{d #2}}
\newcommand{\ddertwo}[2]{\frac{d^2 #1}{d #2^2}}

% Differential operators
\newcommand{\dd}{\mathrm{d}}
\newcommand{\dif}{\mathrm{d}}

% ====================================
% VECTOR AND MATRIX NOTATION
% ====================================

% Bold vectors
\newcommand{\bfv}[1]{\boldsymbol{#1}}
\newcommand{\bfx}{\boldsymbol{x}}
\newcommand{\bfy}{\boldsymbol{y}}
\newcommand{\bfz}{\boldsymbol{z}}
\newcommand{\bfu}{\boldsymbol{u}}
\newcommand{\bfv}{\boldsymbol{v}}
\newcommand{\bfw}{\boldsymbol{w}}
\newcommand{\bff}{\boldsymbol{f}}
\newcommand{\bfg}{\boldsymbol{g}}
\newcommand{\bfF}{\boldsymbol{F}}
\newcommand{\bfn}{\boldsymbol{n}}
\newcommand{\bfe}{\boldsymbol{e}}
\newcommand{\bfzero}{\boldsymbol{0}}

% Matrix spaces
\newcommand{\Rn}{\mathbb{R}^n}
\newcommand{\Rm}{\mathbb{R}^m}
\newcommand{\Rmn}{\mathbb{R}^{m \times n}}

% ====================================
% SETS AND SPACES
% ====================================

\newcommand{\R}{\mathbb{R}}    % Real numbers
\newcommand{\C}{\mathbb{C}}    % Complex numbers
\newcommand{\N}{\mathbb{N}}    % Natural numbers
\newcommand{\Z}{\mathbb{Z}}    % Integers
\newcommand{\Q}{\mathbb{Q}}    % Rationals

% Function spaces
\newcommand{\Lp}[1]{L^{#1}}
\newcommand{\Hone}{H^1}
\newcommand{\Ck}[1]{C^{#1}}

% ====================================
% SPECIAL SYMBOLS
% ====================================

% Indicator/characteristic function
\newcommand{\ind}{\mathbbm{1}}

% Inner product
\newcommand{\inner}[2]{\langle #1, #2 \rangle}

% Norm
\newcommand{\norm}[1]{\left\| #1 \right\|}
\newcommand{\abs}[1]{\left| #1 \right|}

% Set notation
\newcommand{\set}[1]{\left\{ #1 \right\}}
\newcommand{\setdef}[2]{\left\{ #1 \,:\, #2 \right\}}

% Ceiling and floor
\newcommand{\ceil}[1]{\left\lceil #1 \right\rceil}
\newcommand{\floor}[1]{\left\lfloor #1 \right\rfloor}

% ====================================
% MATHCAL AND MATHSF LETTERS
% ====================================

% Calligraphic letters for sets and spaces
\newcommand{\calA}{\mathcal{A}}
\newcommand{\calB}{\mathcal{B}}
\newcommand{\calC}{\mathcal{C}}
\newcommand{\calD}{\mathcal{D}}
\newcommand{\calE}{\mathcal{E}}
\newcommand{\calF}{\mathcal{F}}
\newcommand{\calG}{\mathcal{G}}
\newcommand{\calH}{\mathcal{H}}
\newcommand{\calI}{\mathcal{I}}
\newcommand{\calJ}{\mathcal{J}}
\newcommand{\calK}{\mathcal{K}}
\newcommand{\calL}{\mathcal{L}}
\newcommand{\calM}{\mathcal{M}}
\newcommand{\calN}{\mathcal{N}}
\newcommand{\calO}{\mathcal{O}}
\newcommand{\calP}{\mathcal{P}}
\newcommand{\calQ}{\mathcal{Q}}
\newcommand{\calR}{\mathcal{R}}
\newcommand{\calS}{\mathcal{S}}
\newcommand{\calT}{\mathcal{T}}
\newcommand{\calU}{\mathcal{U}}
\newcommand{\calV}{\mathcal{V}}
\newcommand{\calW}{\mathcal{W}}
\newcommand{\calX}{\mathcal{X}}
\newcommand{\calY}{\mathcal{Y}}
\newcommand{\calZ}{\mathcal{Z}}

% Sans-serif letters for matrices and operators
\newcommand{\msfA}{\mathsf{A}}
\newcommand{\msfB}{\mathsf{B}}
\newcommand{\msfC}{\mathsf{C}}
\newcommand{\msfD}{\mathsf{D}}
\newcommand{\msfE}{\mathsf{E}}
\newcommand{\msfF}{\mathsf{F}}
\newcommand{\msfG}{\mathsf{G}}
\newcommand{\msfH}{\mathsf{H}}
\newcommand{\msfI}{\mathsf{I}}
\newcommand{\msfJ}{\mathsf{J}}
\newcommand{\msfK}{\mathsf{K}}
\newcommand{\msfL}{\mathsf{L}}
\newcommand{\msfM}{\mathsf{M}}
\newcommand{\msfN}{\mathsf{N}}
\newcommand{\msfO}{\mathsf{O}}
\newcommand{\msfP}{\mathsf{P}}
\newcommand{\msfQ}{\mathsf{Q}}
\newcommand{\msfR}{\mathsf{R}}
\newcommand{\msfS}{\mathsf{S}}
\newcommand{\msfT}{\mathsf{T}}
\newcommand{\msfU}{\mathsf{U}}
\newcommand{\msfV}{\mathsf{V}}
\newcommand{\msfW}{\mathsf{W}}
\newcommand{\msfX}{\mathsf{X}}
\newcommand{\msfY}{\mathsf{Y}}
\newcommand{\msfZ}{\mathsf{Z}}

% ====================================
% COMMON ABBREVIATIONS
% ====================================

\newcommand{\ie}{i.e.\ }
\newcommand{\eg}{e.g.\ }
\newcommand{\cf}{cf.\ }
\newcommand{\etc}{etc.\ }
\newcommand{\vs}{vs.\ }
\newcommand{\resp}{resp.\ }
\newcommand{\etal}{et al.\ }

  \graphicspath{{./Figures/}}
  \begin{document}
  \renewcommand{\THERMOEND}{\end{document}}
\fi


\chapter{ThermoS26-02}
\label{chap:ThermoS26-02}

\begin{remark}
From this point forward, we will assume that all postulates hold.
\end{remark}

\begin{theorem}[Thermal Equilibrium]
\label{thm:thermal-equilibrium}
Suppose that in a composite system $\alpha$ and $\beta$ are separated by a diathermal wall. Then, the equilibrium of the composite system may be characterized by
\begin{equation}
\label{eq:energy-conservation}
U^\alpha + U^\beta = U_o
\end{equation}
and
\begin{equation}
\label{eq:thermal-equilibrium}
T^\alpha(U^\alpha, V^\alpha, \vec{N}^\alpha) = T^\beta(U^\beta, V^\beta, \vec{N}^\beta).
\end{equation}
\end{theorem}

\begin{proof}
Internal energy may be exchanged between systems $\alpha$ and $\beta$ but cannot be exchanged with the outside world. Thus \eqref{eq:energy-conservation} must hold because of energy conservation. At equilibrium we must have
\begin{equation}
\label{eq:total-entropy-diathermal}
S(U^\alpha) = \tilde{S}^\alpha(U^\alpha, V^\alpha, \vec{N}^\alpha) + \tilde{S}^\beta(U_o - U^\alpha, V^\beta, \vec{N}^\beta)
\end{equation}
and
\begin{equation}
\label{eq:entropy-extremum}
\frac{\partial S}{\partial U^\alpha} = 0.
\end{equation}

Note, all other variables besides $U^\alpha$ are fixed.
\begin{align}
\label{eq:entropy-derivative-diathermal}
0 &= \frac{\partial S}{\partial U^\alpha} = \frac{\partial S^\alpha}{\partial U^\alpha} + \frac{\partial S^\beta}{\partial U^\beta} \frac{\partial}{\partial U^\alpha}(U_o - U^\alpha) \notag \\
&= [T^\alpha(U^\alpha, V^\alpha, \vec{N}^\alpha)]^{-1} + [T^\beta(U_o - U^\alpha, V^\beta, \vec{N}^\beta)]^{-1}(-1).
\end{align}

Thus,
\begin{equation}
T^\alpha(U^\alpha, V^\alpha, \vec{N}^\alpha) = T^\beta(U^\beta, V^\beta, \vec{N}^\beta)
\end{equation}
with
\begin{equation}
U^\beta = U_o - U^\alpha.
\end{equation}

How do we know that solutions exist and are unique?

\begin{figure}[h]
\centering
\includegraphics[width=\textwidth]{Figures/ThermoS26-01-fig-3}
\caption{Entropy functions vs $U^\alpha$ for subsystems $\alpha$ and $\beta$. The curves show $\tilde{S}^\alpha(U^\alpha)$ (convex, increasing) and $\tilde{S}^\beta(U_o - U^\alpha)$ (decreasing). The temperature conditions at the boundaries are indicated.}
\label{fig:entropy-vs-energy}
\end{figure}

\begin{figure}[h]
\centering
\includegraphics[width=0.8\textwidth]{Figures/ThermoS26-01-fig-4}
\caption{Total entropy $\tilde{S}(U^\alpha) = \tilde{S}^\alpha(U^\alpha) + \tilde{S}^\beta(U_o - U^\alpha)$ vs $U^\alpha$. The concave function achieves its maximum at $U_*$.}
\label{fig:total-entropy}
\end{figure}

This proof can also be carried out by using Lagrange multipliers. Set
\begin{equation}
\label{eq:lagrangian-diathermal}
J := S^\alpha(U^\alpha, V^\alpha, \vec{N}^\alpha) + S^\beta(U^\beta, V^\beta, \vec{N}^\beta) + \lambda(U^\alpha + U^\beta - U_o).
\end{equation}

Then, equilibrium is characterized by
\begin{align}
0 &= \frac{\partial J}{\partial U^\alpha} = [T^\alpha(U^\alpha, V^\alpha, \vec{N}^\alpha)]^{-1} + \lambda, \label{eq:lagrange-condition-1} \\
0 &= \frac{\partial J}{\partial U^\beta} = [T^\beta(U^\beta, V^\beta, \vec{N}^\beta)]^{-1} + \lambda, \label{eq:lagrange-condition-2} \\
0 &= \frac{\partial J}{\partial \lambda} = U^\alpha + U^\beta - U_o, \label{eq:lagrange-condition-3}
\end{align}
which yields the same result.
\end{proof}

The Lagrange Multiplier technique can be visualized as follows:

\begin{figure}[h]
\centering
\includegraphics[width=0.8\textwidth]{Figures/ThermoS26-01-fig-5}
\caption{Lagrange multiplier visualization: finding the maximum in $\tilde{S}$ along the constraint $U^\alpha + U^\beta = U_o$ (red line). The contours show increasing $\tilde{S}$.}
\label{fig:lagrange-visualization}
\end{figure}

\begin{theorem}[Thermal and Mechanical Equilibrium]
\label{thm:thermal-mechanical-equilibrium}
Suppose that in a composite system $\alpha$ and $\beta$ are separated by a diathermal piston. Then, the equilibrium of the composite system may be characterized by
\begin{equation}
\label{eq:energy-conservation-piston}
U^\alpha + U^\beta = U_o,
\end{equation}
\begin{equation}
\label{eq:volume-conservation}
V^\alpha + V^\beta = V_o,
\end{equation}
and
\begin{align}
T^\alpha(U^\alpha, V^\alpha, \vec{N}^\alpha) &= T^\beta(U^\beta, V^\beta, \vec{N}^\beta), \quad (\text{thermal equilibrium}) \label{eq:thermal-equil-piston} \\
P^\alpha(U^\alpha, V^\alpha, \vec{N}^\alpha) &= P^\beta(U^\beta, V^\beta, \vec{N}^\beta). \quad (\text{mechanical equilibrium}) \label{eq:mechanical-equil}
\end{align}
\end{theorem}

\begin{proof}
For this let us use the method of Lagrange multipliers. Note that $\vec{N}^\alpha$ and $\vec{N}^\beta$ are fixed. Define
\begin{equation}
\label{eq:lagrangian-piston}
J := S^\alpha(U^\alpha, V^\alpha, \vec{N}^\alpha) + S^\beta(U^\beta, V^\beta, \vec{N}^\beta) + \lambda_u(U^\alpha + U^\beta - U_o) + \lambda_v(V^\alpha + V^\beta - V_o).
\end{equation}

The conditions for equilibrium are
\begin{align}
0 &= \frac{\partial J}{\partial U^\alpha} = [T^\alpha(U^\alpha, V^\alpha, \vec{N}^\alpha)]^{-1} + \lambda_u, \label{eq:piston-condition-1} \\
0 &= \frac{\partial J}{\partial U^\beta} = [T^\beta(U^\beta, V^\beta, \vec{N}^\beta)]^{-1} + \lambda_u, \label{eq:piston-condition-2} \\
0 &= \frac{\partial J}{\partial V^\alpha} = \frac{P^\alpha(U^\alpha, V^\alpha, \vec{N}^\alpha)}{T^\alpha(U^\alpha, V^\alpha, \vec{N}^\alpha)} + \lambda_v, \label{eq:piston-condition-3} \\
0 &= \frac{\partial J}{\partial V^\beta} = \frac{P^\beta(U^\beta, V^\beta, \vec{N}^\beta)}{T^\beta(U^\beta, V^\beta, \vec{N}^\beta)} + \lambda_v, \label{eq:piston-condition-4}
\end{align}
and
\begin{align}
0 &= \frac{\partial J}{\partial \lambda_u} = U^\alpha + U^\beta - U_o, \label{eq:piston-condition-5} \\
0 &= \frac{\partial J}{\partial \lambda_v} = V^\alpha + V^\beta - V_o. \label{eq:piston-condition-6}
\end{align}

The result is clear.
\end{proof}

Here, we have used the fact that
\begin{align}
\pder{\tilde{S}^\alpha}{U^\alpha} &= \frac{1}{T^\alpha}, \label{eq:entropy-partial-identity-1} \\
\pder{S^\alpha}{V^\alpha} &= \frac{P^\alpha}{T^\alpha}, \label{eq:entropy-partial-identity-2} \\
\pder{S^\alpha}{N_i^\alpha} &= -\frac{\mu_i^\alpha}{T^\alpha}. \label{eq:entropy-partial-identity-3}
\end{align}

As a short hand, we will write
\begin{equation}
\label{eq:energy-differential}
d\tilde{U}^\alpha = T_u^\alpha dS^\alpha - P_u^\alpha dV^\alpha + \sum_{i=1}^r \mu_{u,i}^\alpha dN_i^\alpha
\end{equation}
and
\begin{equation}
\label{eq:entropy-differential}
d\tilde{S}^\alpha = \frac{1}{T_s^\alpha}dU^\alpha + \frac{P_s^\alpha}{T_s^\alpha} dV^\alpha - \sum_{i=1}^r \frac{\mu_{s,i}^\alpha}{T_s^\alpha} dN_i^\alpha.
\end{equation}

Using our abusive notations, we have
\begin{equation}
\label{eq:entropy-differential-abusive}
d\tilde{S}^\alpha = \frac{1}{T^\alpha}dU^\alpha + \frac{P^\alpha}{T^\alpha} dV^\alpha - \sum_{i=1}^r \frac{\mu_i^\alpha}{T^\alpha} dN_i^\alpha,
\end{equation}
for example.

\begin{theorem}[Full Equilibrium]
\label{thm:full-equilibrium}
Suppose that a composite system is comprised of two otherwise isolated systems with no barrier between the systems. Suppose that
\begin{equation}
S^\alpha = S^\alpha(U^\alpha, V^\alpha, \vec{N}^\alpha)
\end{equation}
and
\begin{equation}
S^\beta = S^\beta(U^\beta, V^\beta, \vec{N}^\beta)
\end{equation}
are the fundamental entropy relations for the two systems. Then the equilibrium state is defined by the relations
\begin{align}
U^\alpha + U^\beta &= U_o, \label{eq:full-equil-energy} \\
V^\alpha + V^\beta &= V_o, \label{eq:full-equil-volume} \\
N_i^\alpha + N_i^\beta &= N_{o,i}, \label{eq:full-equil-moles}
\end{align}
where $U_o, V_o, N_{o,i} > 0$, and
\begin{align}
T^\alpha(U^\alpha, V^\alpha, \vec{N}^\alpha) &= T^\beta(U^\beta, V^\beta, \vec{N}^\beta), \quad (\text{thermal equil.}) \label{eq:full-thermal} \\
P^\alpha(U^\alpha, V^\alpha, \vec{N}^\alpha) &= P^\beta(U^\beta, V^\beta, \vec{N}^\beta), \quad (\text{mech. equil.}) \label{eq:full-mechanical} \\
\mu_i^\alpha(U^\alpha, V^\alpha, \vec{N}^\alpha) &= \mu_i^\beta(U^\beta, V^\beta, \vec{N}^\beta). \quad (\text{chem. equil.}) \label{eq:full-chemical}
\end{align}
\end{theorem}

\begin{proof}
The procedure is the same. One can use the method of Lagrange multipliers to do the calculation. In particular, define
\begin{multline}
\label{eq:lagrangian-full}
J := S^\alpha(U^\alpha, V^\alpha, \vec{N}^\alpha) + S^\beta(U^\beta, V^\beta, \vec{N}^\beta) + \lambda_u(U^\alpha + U^\beta - U_o) \\
+ \lambda_v(V^\alpha + V^\beta - V_o) + \sum_{i=1}^r \lambda_i(N_i^\alpha + N_i^\beta - N_{o,i}).
\end{multline}
\end{proof}

\section{Fundamental Relations and Equations of State}

\begin{figure}[h]
\centering
\includegraphics[width=0.4\textwidth]{Figures/ThermoS26-02-fig-1.pdf}
\caption{Coordinate system for state space with axes $N$ (vertical), $V$ (horizontal right), and $S$ (downward).}
\label{fig:state-space-coords}
\end{figure}

\begin{example}
\label{ex:fundamental-relation-unary}
Suppose that, for an isolated unary fluid,
\begin{equation}
\label{eq:fundamental-u-example}
\tilde{U} = \left(\frac{v_0 e}{R^2}\right) \frac{S^3}{NV}, \quad \Sigma_s \subset [0,\infty)^3.
\end{equation}
\end{example}

An explicit expression of the form $\tilde{U} = \tilde{U}(S,V,N)$ is called a \textbf{fundamental relation}.

Here $v_0$, $e$, and $R$ are positive constants.

The units of $U$ and $\tilde{U}$ are
\begin{equation}
\label{eq:units-energy}
[U] = \text{Joules}.
\end{equation}

The units of entropy, $S$ and $\tilde{S}$, are
\begin{equation}
\label{eq:units-entropy}
[S] = \frac{\text{Joules}}{\text{degree Kelvin}} = \frac{J}{K}.
\end{equation}

The units of volume, $V$, are
\begin{equation}
\label{eq:units-volume}
[V] = \text{meters}^3.
\end{equation}

The units of $N$ are
\begin{equation}
\label{eq:units-moles}
[N] = \text{moles}.
\end{equation}

Of course, it is easy to see that
\begin{equation}
\label{eq:fundamental-s-example}
\tilde{S} = \left(\frac{NVUR^2}{v_0 e}\right)^{1/3}, \quad \Sigma_u = [0,\infty)^3.
\end{equation}

An explicit function of the form
\begin{equation}
\label{eq:fundamental-s-form}
S = \tilde{S}(U,V,N)
\end{equation}
is also called a \textbf{fundamental relation}.

In any case, the temperature which has units
\begin{equation}
\label{eq:units-temperature}
[T] = \text{degrees Kelvin} = K,
\end{equation}
is
\begin{equation}
\label{eq:temperature-equation-state}
T_U(S,V,N) = \frac{\partial \tilde{U}}{\partial S} = \frac{3v_0 e}{R^2} \frac{S^2}{NV}.
\end{equation}

This expression is called an \textbf{equation of state}.

Observe that
\begin{equation}
\label{eq:temperature-homogeneous-zero}
T_U(\lambda S, \lambda V, \lambda N) = T_U(S,V,N),
\end{equation}
that is $T_U$ is homogeneous of order 0.

This property is true of every equation of state.

Now,
\begin{align}
\label{eq:temperature-inverse-1}
\left(\frac{\partial \tilde{S}}{\partial U}\right)^{-1} &= \left[\frac{1}{3}\left(\frac{NVUR^2}{v_0 e}\right)^{-2/3} \frac{NVR^2}{v_0 e}\right]^{-1} \notag \\
&= \frac{3v_0 e}{NVR^2}\left(\frac{NVUR^2}{v_0 e}\right)^{2/3} \notag \\
&= \frac{3v_0 e}{NVR^2} \tilde{S}(U,V,N)^2 \notag \\
&= T_S(U,V,N).
\end{align}

Clearly
\begin{equation}
\label{eq:temperature-consistency}
T_U\left(\tilde{S}(U,V,N), V, N\right) = T_S(U,V,N),
\end{equation}
as claimed in Theorem (1.7).

The pressure satisfies the equation of state
\begin{align}
\label{eq:pressure-equation-state}
P_U(S,V,N) &= -\frac{\partial \tilde{U}}{\partial V} = -\frac{v_0 e}{R^2} \frac{S^3}{N} \left(-\frac{1}{V^2}\right) \notag \\
&= \frac{v_0 e}{R^2} \frac{S^3}{NV^2},
\end{align}
which is also homogeneous of order zero.

We leave it as an exercise for the reader to show that
\begin{equation}
\label{eq:pressure-consistency}
P_S(U,V,N) = P_U\left(\tilde{S}(U,V,N), V, N\right).
\end{equation}

Finally, the chemical potential is
\begin{equation}
\label{eq:chemical-potential-equation-state}
\mu_U(S,V,N) = \frac{\partial \tilde{U}}{\partial N} = -\frac{v_0 e}{R^2} \frac{S^3}{V N^2},
\end{equation}
which is clearly homogeneous of degree zero.

The reader can show that
\begin{equation}
\label{eq:chemical-potential-consistency}
\mu_S(U,V,N) = \mu_U\left(\tilde{S}(U,V,N), V, N\right).
\end{equation}

\section{Homogeneity of the Fundamental Relations}

Recall, we have assumed with Postulate II that $\tilde{S}$ is homogeneous of degree one. We also must have the following, as suggested by the example.

\begin{theorem}
\label{thm:u-homogeneous}
$\tilde{U}$ is homogeneous of degree one when written as
\begin{equation}
\label{eq:u-homogeneous-one}
\tilde{U} = \tilde{U}(S,V,\vec{N}).
\end{equation}

Further, $T_U(S,V,\vec{N})$, $P_U(S,V,\vec{N})$, and $\mu_{U;i}(S,V,\vec{N})$, the equations of state, are homogeneous of degree zero, meaning
\begin{equation}
\label{eq:equations-state-homogeneous-zero}
T_U(\lambda S, \lambda V, \lambda \vec{N}) = T_U(S,V,\vec{N})
\end{equation}
for any $\lambda > 0$, and similarly for $P_U$ and $\mu_{U;i}$; $i=1,\ldots,r$.

Likewise $T_S(U,V,\vec{N})$, $P_S(U,V,\vec{N})$, and $\mu_{S;i}(U,V,\vec{N})$ are homogeneous of degree zero.
\end{theorem}

\begin{proof}
Fix $V \in [0,\infty)$ and $\vec{N} \in [0,\infty)^r$. $\tilde{S}$ is a monotonically increasing function of $U \in [0,\infty)$. For each $S \in [0,\infty)$ there exists a unique $U \in [0,\infty)$ such that
\begin{equation}
\label{eq:s-u-relation}
S = \tilde{S}(U,V,\vec{N}),
\end{equation}
where we assume, for simplicity, that $\Sigma_U = [0,\infty)^{r+2}$. Then,
\begin{equation}
\label{eq:u-inverse-relation}
\tilde{U}(S,V,\vec{N}) = \tilde{U}\left(\tilde{S}(U,V,\vec{N}), V, \vec{N}\right) = U, \quad \forall (S,V,\vec{N}) \in \Sigma_S.
\end{equation}

Let $\lambda > 0$ be arbitrary; then \eqref{eq:s-u-relation} and \eqref{eq:u-inverse-relation} imply
\begin{equation}
\label{eq:u-scaled}
\tilde{U}(\lambda S, \lambda V, \lambda \vec{N}) = \tilde{U}\left(\lambda \widetilde{S}(U,V,\vec{N}), \lambda V, \lambda \vec{N}\right).
\end{equation}

Since $\tilde{S}$ is homogeneous of degree 1, it follows that
\begin{equation}
\label{eq:s-homogeneous-one}
\lambda \tilde{S}(U,V,\vec{N}) = \tilde{S}(\lambda U, \lambda V, \lambda \vec{N}).
\end{equation}

Also, recall that, generically,
\begin{equation}
\label{eq:u-generic-inverse}
\tilde{U}\left(\tilde{S}(\hat{U},\hat{V},\hat{N}), \hat{V}, \hat{N}\right) = \hat{U}
\end{equation}
because of inverse relations. Combining \eqref{eq:u-scaled}--\eqref{eq:u-generic-inverse} we have
\begin{align}
\label{eq:u-homogeneous-derivation}
\tilde{U}(\lambda S, \lambda V, \lambda \vec{N}) &= \tilde{U}\left(\tilde{S}(\lambda U, \lambda V, \lambda \vec{N}), \lambda V, \lambda \vec{N}\right) \notag \\
&= \lambda U \notag \\
&= \lambda \tilde{U}\left(\tilde{S}(U,V,\vec{N}), V, \vec{N}\right) \notag \\
&= \lambda \tilde{U}(S,V,\vec{N}).
\end{align}
This completes the proof.
\end{proof}

\section{Path (Contour) Integrals}

\begin{definition}[Path, Path-connected, Simply-connected, Convex]
\label{defn:path}
Suppose that $\mathcal{D} \subseteq \mathbb{R}^n$ is open. A function $\vec{Y}: [a,b] \to \mathcal{D}$ is called a \textbf{path} (or \textbf{contour}) iff $\vec{Y}$ is continuous and piecewise smooth. $\mathcal{D}$ is called \textbf{path-connected} iff for every two distinct points $\vec{a}, \vec{b} \in \mathcal{D}$ there is a path $\vec{Y}: [0,1] \to \mathcal{D}$ such that
\begin{equation}
\label{eq:path-endpoints}
\vec{Y}(a) = \vec{a} \quad \text{and} \quad \vec{Y}(b) = \vec{b}.
\end{equation}

$\mathcal{D}$ is called \textbf{simply-connected} iff it is (1) path connected and (2) paths can be continuously deformed to a point, \ie, there are no holes. $\mathcal{D}$ is called \textbf{convex} iff: for every pair $\vec{a}, \vec{b} \in \mathcal{D}$, the point
\begin{equation}
\label{eq:convex-definition}
\vec{x}(t) = \vec{a}(1-t) + \vec{b}t \in \mathcal{D}
\end{equation}
for all $t \in [0,1]$.
\end{definition}

\begin{definition}[Path Integral]
\label{defn:path-integral}
Let $\vec{F}: \mathcal{D} \to \mathbb{R}^n$ be a $C^1$ function, \ie, $\vec{F} \in C^1(\mathcal{D}; \mathbb{R}^n)$. Let $\vec{Y}: [a,b] \to \mathcal{D}$ be a path in $\mathcal{D}$, which is assumed to be simply connected. Then the path integral $\int_{\vec{Y}} \vec{F}(\vec{x}) \cdot d\vec{x}$ is defined via
\begin{equation}
\label{eq:path-integral-definition}
\int_{\vec{Y}} \vec{F}(\vec{x}) \cdot d\vec{x} := \int_a^b \vec{F}(\vec{Y}(x)) \cdot \vec{Y}'(x) \, dx.
\end{equation}

We will also use the notation
\begin{equation}
\label{eq:path-integral-notation}
\int_{\vec{Y}} \vec{F}(\vec{x}) \cdot d\vec{x} = \int_{\vec{Y}} F_1(\vec{x})dx_1 + \cdots + F_n(\vec{x})dx_n.
\end{equation}
\end{definition}

\begin{definition}[Closed Path, Simple Path]
\label{defn:closed-path}
Let $\mathcal{D} \subseteq \mathbb{R}^n$ be a simply connected open set. A path $\vec{Y}: [a,b] \to \mathcal{D}$ is called \textbf{closed} iff
\begin{equation}
\label{eq:closed-path}
\vec{Y}(a) = \vec{Y}(b).
\end{equation}

A closed path is called \textbf{simple} iff it does not intersect itself except at $t=a$ and $t=b$, \ie, for every $c \in (a,b)$
\begin{equation}
\label{eq:simple-path}
\vec{Y}(c) \neq \vec{Y}(x) \quad \forall x \in [a,c) \cup (c,b].
\end{equation}
\end{definition}

\begin{theorem}[Parametric Independence]
\label{thm:parametric-independence}
Let $\mathcal{D} \subseteq \mathbb{R}^n$ be an open, simply-connected set. Assume $\vec{Y}: [a,b] \to \mathcal{D}$ is a path. If $\vec{x}: [c,d] \to \mathcal{D}$ is a path in $\mathcal{D}$, with the property that
\begin{equation}
\label{eq:same-endpoints}
\vec{x}(c) = \vec{Y}(a), \quad \vec{x}(d) = \vec{Y}(b),
\end{equation}
and
\begin{equation}
\label{eq:same-range}
\text{Range}(\vec{x}) = \text{Range}(\vec{Y}),
\end{equation}
then
\begin{equation}
\label{eq:parametric-independence}
\int_{\vec{Y}} \vec{F}(\vec{x}) \cdot d\vec{x} = \int_{\vec{x}} \vec{F}(\vec{x}) \cdot d\vec{x}.
\end{equation}
\end{theorem}

This result guarantees that the path integrals are parametrically independent.

If $c = \text{Range}(\vec{Y}) = \text{Range}(\vec{x})$, then we write
\begin{equation}
\label{eq:contour-notation}
\int_c \vec{F}(\vec{x}) \cdot d\vec{x} = \int_{\vec{Y}} \vec{F}(\vec{x}) \cdot d\vec{x}.
\end{equation}

\begin{definition}[Path Independence]
\label{defn:path-independence}
Let $\mathcal{D} \subseteq \mathbb{R}^n$ be an open, simply connected set. Suppose that
\begin{equation}
\label{eq:path-independence-definition}
\int_{\vec{Y}_1} \vec{F}(\vec{x}) \cdot d\vec{x} = \int_{\vec{Y}_2} \vec{F}(\vec{x}) \cdot d\vec{x}
\end{equation}
for any two paths $\vec{Y}_1: [a,b] \to \mathcal{D}$, $\vec{Y}_2: [a,b] \to \mathcal{D}$ with
\begin{equation}
\label{eq:same-endpoints-independence}
\vec{Y}_1(a) = \vec{Y}_2(a) \quad \text{and} \quad \vec{Y}_1(b) = \vec{Y}_2(b).
\end{equation}

Then we say that the integral is \textbf{path independent}. Note that we are not assuming that
\begin{equation}
\label{eq:different-ranges}
\text{Range}(\vec{Y}_1) = \text{Range}(\vec{Y}_2).
\end{equation}
\end{definition}

\begin{definition}[Conservative Vector Field]
\label{defn:conservative}
Let $\mathcal{D}$ be an open set and $\vec{F} \in C^1(\mathcal{D}; \mathbb{R}^n)$. We say that $\vec{F}$ is \textbf{conservative} iff there is a function $f \in C^1(\mathcal{D}; \mathbb{R})$ such that
\begin{equation}
\label{eq:conservative-definition}
\vec{F}(\vec{x}) = \nabla f(\vec{x}), \quad \forall \vec{x} \in \mathcal{D}.
\end{equation}
\end{definition}

\begin{theorem}
\label{thm:conservative-path-independent}
Let $\mathcal{D}$ be an open simply connected set in $\mathbb{R}^n$. If $\vec{F} \in C^1(\mathcal{D}; \mathbb{R}^n)$ is conservative, then the integral
\begin{equation}
\label{eq:conservative-integral}
\int_{\vec{Y}} \vec{F}(\vec{x}) \cdot d\vec{x}
\end{equation}
is path-independent.
\end{theorem}

\begin{proof}
Let $\vec{Y}_1: [a_1, b_1] \to \mathcal{D}$ and $\vec{Y}_2: [a_2, b_2] \to \mathcal{D}$ be paths in $\mathcal{D}$ with the same end points, \ie,
\begin{equation}
\label{eq:proof-endpoints}
\vec{a} := \vec{Y}_1(a_1) = \vec{Y}_2(a_2), \quad \vec{Y}_1(b_1) = \vec{Y}_2(b_2) =: \vec{b}.
\end{equation}

By the chain rule, for $i=1,2$,
\begin{align}
\label{eq:chain-rule-path}
\frac{d}{dx} f(\vec{Y}_i(x)) &= \nabla f(\vec{Y}_i(x)) \cdot \vec{Y}_i'(x) \notag \\
&= \vec{F}(\vec{Y}_i(x)) \cdot \vec{Y}_i'(x).
\end{align}

Thus,
\begin{align}
\label{eq:path-integral-ftc}
\int_{\vec{Y}_1} \vec{F}(\vec{x}) \cdot d\vec{x} &= \int_{a_1}^{b_1} \vec{F}(\vec{Y}_1(x)) \cdot \vec{Y}_1'(x) \, dx \notag \\
&= \int_{a_1}^{b_1} \frac{d}{dx}\left[f(\vec{Y}_1(x))\right] dx \notag \\
&\overset{\text{FTC}}{=} f(\vec{b}) - f(\vec{a}),
\end{align}
for $i=1,2$. This completes the proof.
\end{proof}

We have the following well-known results.

\begin{theorem}
\label{thm:conservative-equivalence}
Let $\mathcal{D} \subseteq \mathbb{R}^n$ be a simply-connected set and suppose that $\vec{F} \in C^1(\mathcal{D}; \mathbb{R}^n)$. The following are equivalent:
\begin{enumerate}
\item $\vec{F}$ is conservative
\item $\int_{\vec{Y}} \vec{F}(\vec{x}) \, d\vec{x}$ is path independent
\item $\oint_{\vec{Y}} \vec{F}(\vec{x}) \, d\vec{x} = 0$ for any closed path.
\end{enumerate}
\end{theorem}

\THERMOEND
