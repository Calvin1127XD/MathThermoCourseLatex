\newif\ifthermosubfile
\ifdefined\THERMONOTESMAIN
  \thermosubfilefalse
\else
  \thermosubfiletrue
\fi

\providecommand{\THERMOEND}{} % safe in both modes

\ifthermosubfile
  \documentclass[11pt]{book}
  \usepackage{cmbright}
\usepackage{mathtools}

\usepackage{pgfplots}
\usepgfplotslibrary{patchplots}
\pgfplotsset{compat=1.15}

\usepackage{color}
\usepackage{xcolor}
\usepackage{subfigure}

\usepackage{amssymb}
\usepackage{amsmath}
\usepackage{amsfonts}
\usepackage{amsthm} 
\usepackage{mdframed}
\usepackage{bbm}
\usepackage{listings}

\usepackage[shortlabels]{enumitem}

\usepackage{tikz}
\usepackage{tikz-3dplot}
\usepackage{xkeyval,tkz-base}

\usetikzlibrary{patterns}
\usetikzlibrary{arrows,arrows.meta,shapes,decorations,automata}
\usetikzlibrary{backgrounds,petri,matrix}
\usetikzlibrary{intersections}
\usetikzlibrary{calc}
\usetikzlibrary{positioning}
\usetikzlibrary{shapes.geometric}
\usetikzlibrary{shapes.arrows}
\usetikzlibrary{matrix}
\usetikzlibrary{positioning}
\usetikzlibrary{3d,calc}

\usepackage{multirow}
\usepackage{array}
\usepackage{algorithm}
\usepackage{algpseudocode}

\usepackage{graphicx}
\usepackage{latexsym}
\usepackage{booktabs}
\usepackage{longtable}

%\usepackage{aligned-overset}

\numberwithin{equation}{section}

\usepackage{lastpage}
\usepackage{fancyhdr}
\pagestyle{fancy}
\setlength{\headheight}{14pt}
\lhead{Mathematical Thermodynamics}
\chead{}
\rhead{Page \thepage}
\cfoot{Page\ \thepage\ of\ \protect\pageref{LastPage}}
\setcounter{section}{-1}

% Theorem environments
\theoremstyle{plain}
\newtheorem{theorem}{Theorem}[section]
\newtheorem{lemma}[theorem]{Lemma}
\newtheorem{proposition}[theorem]{Proposition}
\newtheorem{corollary}[theorem]{Corollary}

\theoremstyle{definition}
\newtheorem{definition}[theorem]{Definition}
\newtheorem{example}[theorem]{Example}
\newtheorem{exercise}[theorem]{Exercise}

\theoremstyle{remark}
\newtheorem{remark}[theorem]{Remark}
\newtheorem{note}[theorem]{Note}

% Hyperref setup
\usepackage{hyperref}
\usepackage{nameref}
\hypersetup{
    hypertexnames=true,
    breaklinks=false,
    colorlinks=true,
    linkcolor=blue,
    citecolor=blue,
    urlcolor=blue
}

\makeatletter
  \newenvironment{dedication}
  {
     \cleardoublepage
     \thispagestyle{empty}
     \vspace*{\stretch{1}}
     \hfill\begin{minipage}[t]{0.66\textwidth}
     \raggedright
  }%
  {
     \end{minipage}
     \vspace*{\stretch{3}}
     \clearpage
  }
  \newenvironment{varsubequations}[1]
  {%
    \addtocounter{equation}{-1}%
    \begin{subequations}
    \renewcommand{\theparentequation}{#1}%
    \def\@currentlabel{#1}%
  }
  {%
    \end{subequations}\ignorespacesafterend
  }
\makeatother

  % ====================================
% THERMODYNAMIC STATE VARIABLES
% ====================================

% Extensive properties
\newcommand{\Int}{U}           % Internal energy
\newcommand{\Ent}{S}           % Entropy
\newcommand{\Vol}{V}           % Volume
\newcommand{\Enth}{H}          % Enthalpy H = U + PV
\newcommand{\Helm}{A}          % Helmholtz free energy A = U - TS
\newcommand{\Gibbs}{G}         % Gibbs free energy G = H - TS
\newcommand{\GrandPot}{\Omega} % Grand potential

% Intensive properties
\newcommand{\Temp}{T}          % Temperature
\newcommand{\Press}{P}         % Pressure
\newcommand{\ChemPot}{\mu}     % Chemical potential
\newcommand{\Mass}{m}          % Mass
\newcommand{\Moles}{n}         % Number of moles

% Specific properties (per unit mass)
\newcommand{\sint}{u}          % Specific internal energy
\newcommand{\sent}{s}          % Specific entropy
\newcommand{\svol}{v}          % Specific volume
\newcommand{\senth}{h}         % Specific enthalpy

% Heat and work
\newcommand{\Heat}{Q}          % Heat transfer
\newcommand{\Work}{W}          % Work
\newcommand{\dQ}{\delta Q}     % Inexact differential of heat
\newcommand{\dW}{\delta W}     % Inexact differential of work

% Thermodynamic processes
\newcommand{\iso}{\text{iso}}
\newcommand{\rev}{\text{rev}}
\newcommand{\irrev}{\text{irrev}}

% ====================================
% MATHEMATICAL OPERATORS
% ====================================

\DeclareMathOperator{\grad}{grad}
\DeclareMathOperator{\dive}{div}
\DeclareMathOperator{\curl}{curl}
\DeclareMathOperator{\Span}{span}
\DeclareMathOperator{\rank}{rank}
\DeclareMathOperator{\trace}{tr}
\DeclareMathOperator{\sgn}{sgn}
\DeclareMathOperator{\diam}{diam}

% Partial derivatives
\newcommand{\pder}[2]{\frac{\partial #1}{\partial #2}}
\newcommand{\pdertwo}[2]{\frac{\partial^2 #1}{\partial #2^2}}
\newcommand{\pdermix}[3]{\frac{\partial^2 #1}{\partial #2 \partial #3}}

% Thermodynamic partial derivatives
\newcommand{\pderT}[3]{\left(\frac{\partial #1}{\partial #2}\right)_{#3}}

% Total derivatives
\newcommand{\dder}[2]{\frac{d #1}{d #2}}
\newcommand{\ddertwo}[2]{\frac{d^2 #1}{d #2^2}}

% Differential operators
\newcommand{\dd}{\mathrm{d}}
\newcommand{\dif}{\mathrm{d}}

% ====================================
% VECTOR AND MATRIX NOTATION
% ====================================

% Bold vectors
\newcommand{\bfv}[1]{\boldsymbol{#1}}
\newcommand{\bfx}{\boldsymbol{x}}
\newcommand{\bfy}{\boldsymbol{y}}
\newcommand{\bfz}{\boldsymbol{z}}
\newcommand{\bfu}{\boldsymbol{u}}
\newcommand{\bfv}{\boldsymbol{v}}
\newcommand{\bfw}{\boldsymbol{w}}
\newcommand{\bff}{\boldsymbol{f}}
\newcommand{\bfg}{\boldsymbol{g}}
\newcommand{\bfF}{\boldsymbol{F}}
\newcommand{\bfn}{\boldsymbol{n}}
\newcommand{\bfe}{\boldsymbol{e}}
\newcommand{\bfzero}{\boldsymbol{0}}

% Matrix spaces
\newcommand{\Rn}{\mathbb{R}^n}
\newcommand{\Rm}{\mathbb{R}^m}
\newcommand{\Rmn}{\mathbb{R}^{m \times n}}

% ====================================
% SETS AND SPACES
% ====================================

\newcommand{\R}{\mathbb{R}}    % Real numbers
\newcommand{\C}{\mathbb{C}}    % Complex numbers
\newcommand{\N}{\mathbb{N}}    % Natural numbers
\newcommand{\Z}{\mathbb{Z}}    % Integers
\newcommand{\Q}{\mathbb{Q}}    % Rationals

% Function spaces
\newcommand{\Lp}[1]{L^{#1}}
\newcommand{\Hone}{H^1}
\newcommand{\Ck}[1]{C^{#1}}

% ====================================
% SPECIAL SYMBOLS
% ====================================

% Indicator/characteristic function
\newcommand{\ind}{\mathbbm{1}}

% Inner product
\newcommand{\inner}[2]{\langle #1, #2 \rangle}

% Norm
\newcommand{\norm}[1]{\left\| #1 \right\|}
\newcommand{\abs}[1]{\left| #1 \right|}

% Set notation
\newcommand{\set}[1]{\left\{ #1 \right\}}
\newcommand{\setdef}[2]{\left\{ #1 \,:\, #2 \right\}}

% Ceiling and floor
\newcommand{\ceil}[1]{\left\lceil #1 \right\rceil}
\newcommand{\floor}[1]{\left\lfloor #1 \right\rfloor}

% ====================================
% MATHCAL AND MATHSF LETTERS
% ====================================

% Calligraphic letters for sets and spaces
\newcommand{\calA}{\mathcal{A}}
\newcommand{\calB}{\mathcal{B}}
\newcommand{\calC}{\mathcal{C}}
\newcommand{\calD}{\mathcal{D}}
\newcommand{\calE}{\mathcal{E}}
\newcommand{\calF}{\mathcal{F}}
\newcommand{\calG}{\mathcal{G}}
\newcommand{\calH}{\mathcal{H}}
\newcommand{\calI}{\mathcal{I}}
\newcommand{\calJ}{\mathcal{J}}
\newcommand{\calK}{\mathcal{K}}
\newcommand{\calL}{\mathcal{L}}
\newcommand{\calM}{\mathcal{M}}
\newcommand{\calN}{\mathcal{N}}
\newcommand{\calO}{\mathcal{O}}
\newcommand{\calP}{\mathcal{P}}
\newcommand{\calQ}{\mathcal{Q}}
\newcommand{\calR}{\mathcal{R}}
\newcommand{\calS}{\mathcal{S}}
\newcommand{\calT}{\mathcal{T}}
\newcommand{\calU}{\mathcal{U}}
\newcommand{\calV}{\mathcal{V}}
\newcommand{\calW}{\mathcal{W}}
\newcommand{\calX}{\mathcal{X}}
\newcommand{\calY}{\mathcal{Y}}
\newcommand{\calZ}{\mathcal{Z}}

% Sans-serif letters for matrices and operators
\newcommand{\msfA}{\mathsf{A}}
\newcommand{\msfB}{\mathsf{B}}
\newcommand{\msfC}{\mathsf{C}}
\newcommand{\msfD}{\mathsf{D}}
\newcommand{\msfE}{\mathsf{E}}
\newcommand{\msfF}{\mathsf{F}}
\newcommand{\msfG}{\mathsf{G}}
\newcommand{\msfH}{\mathsf{H}}
\newcommand{\msfI}{\mathsf{I}}
\newcommand{\msfJ}{\mathsf{J}}
\newcommand{\msfK}{\mathsf{K}}
\newcommand{\msfL}{\mathsf{L}}
\newcommand{\msfM}{\mathsf{M}}
\newcommand{\msfN}{\mathsf{N}}
\newcommand{\msfO}{\mathsf{O}}
\newcommand{\msfP}{\mathsf{P}}
\newcommand{\msfQ}{\mathsf{Q}}
\newcommand{\msfR}{\mathsf{R}}
\newcommand{\msfS}{\mathsf{S}}
\newcommand{\msfT}{\mathsf{T}}
\newcommand{\msfU}{\mathsf{U}}
\newcommand{\msfV}{\mathsf{V}}
\newcommand{\msfW}{\mathsf{W}}
\newcommand{\msfX}{\mathsf{X}}
\newcommand{\msfY}{\mathsf{Y}}
\newcommand{\msfZ}{\mathsf{Z}}

% ====================================
% COMMON ABBREVIATIONS
% ====================================

\newcommand{\ie}{i.e.\ }
\newcommand{\eg}{e.g.\ }
\newcommand{\cf}{cf.\ }
\newcommand{\etc}{etc.\ }
\newcommand{\vs}{vs.\ }
\newcommand{\resp}{resp.\ }
\newcommand{\etal}{et al.\ }

  \graphicspath{{./Figures/}}
  \begin{document}
  \renewcommand{\THERMOEND}{\end{document}}
\fi


\chapter{ThermoS26-05}
\label{chap:ThermoS26-05}

\noindent Math Thermo\\
Class \#05\\
02/03/2026

\section{Legendre Transforms and Thermodynamic Potentials}

\begin{definition}
\label{def:legendre}
\textup{(5.1)} Suppose that $f:[a,b]\to\mathbb{R}$ is
continuous. For every $p\in[c,d]$, define
\begin{equation}
f^\ast(p):=\sup_{x\in[a,b]}\{xp-f(x)\}
\tag{5.1}\label{eq:legendre-def}
\end{equation}
The function $f^\ast:[c,d]\to\mathbb{R}$ is called the
Legendre transform of $f$.
\end{definition}

\begin{theorem}
\label{thm:legendre-form}
\textup{(5.2)} Suppose that $f\in C^2([0,\infty);\mathbb{R})$
with
\[
f''(x)>0,\qquad\forall x\in[0,\infty).
\]
Then, for all $p\in\mathrm{Range}(f')$,
\begin{equation}
f^\ast(p)=x_p\,p-f(x_p),
\tag{5.2}\label{eq:legendre-form}
\end{equation}
where $x_p\in[0,\infty)$ is the unique solution to
\[
f'(x_p)=p.
\]
\end{theorem}

\begin{proof}
let $p\in\mathrm{Range}(f')$.
\[
f':[0,\infty)\to\mathrm{Range}(f')
\]
For every $p\in\mathrm{Range}(f')\ \exists!\ x_p\in[0,\infty)$ and that,
\[
f'(x_p)=p.
\]

Since $f':[0,\infty)$ is strictly increasing,

Fix $p\in\mathrm{Range}(f')$. By Taylor's Theorem,
\[
f(x)=f(x_p)+p(x-x_p)+\frac{1}{2}f''(\xi_p)(x-x_p)^2
\]
for some $\xi_p$ between $x$ and $x_p$.

\[
\sup_{0\le x<\infty}\{x\cdot p-f(x)\}
=\sup_{0\le x<\infty}\{x_p\cdot p-f(x_p)-\tfrac{1}{2}f''(\xi_p)(x-x_p)^2\}
\]
\[
= x_p\cdot p-f(x_p). ///
\]
\end{proof}

\begin{theorem}
\label{thm:legendre-derivs}
\textup{(5.3)} Suppose that $f\in C^2([0,\infty);\mathbb{R})$
with
\[
f''(x)>0,\qquad\forall x\in[0,\infty).
\]
Then,
\[
[f^\ast]'(p)=[f']^{-1}(p),\qquad [f^\ast]''(p)>0,
\]
for all $p\in\mathrm{Range}(f')$.
\end{theorem}

\begin{proof}
Recall
\[
f':[0,\infty)\to\mathrm{Range}(f')
\]

Then
\[
f'\big([f']^{-1}(p)\big)=p,\qquad\forall\ p\in\mathrm{Range}(f').
\]

Also, observe that $\frac{d}{dp}[f']^{-1}\in C(\mathrm{Range}(f');\mathbb{R}).$

Now,
\begin{align*}
\frac{d}{dp}[f^\ast(p)]&=\frac{d}{dp}\{x_p\cdot p-f(x_p)\}
=\frac{d}{dp}\{[f']^{-1}(p)\cdot p-f([f']^{-1}(p))\}\\
&=p\cdot\frac{d}{dp}[f']^{-1}(p)+[f']^{-1}(p)
-\frac{df}{dx}([f']^{-1}(p))\cdot\frac{d}{dp}[f']^{-1}(p)\\
&=[f']^{-1}(p).
\end{align*}

Now that this is established, we have
\[
\frac{d^2}{dp^2}[f^\ast(p)]=\frac{d}{dp}[f']^{-1}(p)
\]

for all $p\in\mathrm{Range}(f')$. Since $f''\in C([0,\infty))$ and
$f''(x)>0,\ \forall x\in[0,\infty)$,
\[
\frac{d}{dp}[f']^{-1}(p)>0,\qquad\forall p\in\mathrm{Range}(f'). ///
\]
\end{proof}

Finally, we observe the following:

\begin{theorem}
\label{thm:legendre-involution}
\textup{(5.4)} Suppose that $f\in C^2([0,\infty);\mathbb{R})$
with
\[
f''(x)>0,\qquad\forall x\in[0,\infty).
\]

Then $f^\ast\in C^2(\mathrm{Range}(f');\mathbb{R})$ with
\[
[f^\ast]''(p)>0,\qquad\forall p\in\mathrm{Range}(f')
\]

Furthermore
\begin{equation}
[f^\ast]^\ast(x)=f(x),\qquad\forall x\in[0,\infty).
\tag{5.3}\label{eq:legendre-involution}
\end{equation}

In other words, the Legendre transform is involutive
and in fact an isomorphism.
\end{theorem}

\begin{proof}
It follows that, for all $x\in[0,\infty)$,
\[
[f^\ast]^\ast(x)=x\,p_x-f^\ast(p_x),
\]

where $p_x\in\mathrm{Range}(f')$ is the unique solution to
\[
f^\ast{}'(p_x)=x.
\]

Recall
\[
f^\ast(p_x)=p_x\cdot x_{p_x}-f(x_{p_x}),
\]

where $x_{p_x}\in[0,\infty)$ is the unique solution to
\[
f'(x_{p_x})=p_x.
\]

Of course, by uniqueness,
\[
x_{p_x}=x.
\]

So,
\[
[f^\ast]^\ast(x)=x\cdot p_x-\{p_x\cdot x_{p_x}-f(x_{p_x})\}
=f(x). ///
\]
\end{proof}

Why is the Legendre transform useful in
Thermodynamics?

This transform allows us to introduce new
thermodynamics coordinates/variables.

Recall
\[
\tilde{U}=\tilde{U}(S,V,N)
\]

or
\[
 d\tilde{U}=T\,dS-p\,dV+\mu\,dN
\]

The latter means
\[
T=\frac{\partial \tilde{U}}{\partial S}\quad p=-\frac{\partial \tilde{U}}{\partial V}\quad \mu=\frac{\partial \tilde{U}}{\partial N}
\]

in short hand
\begin{table}[h]
\centering
\begin{tabular}{lll}
\toprule
Extensive Quantity & Conjugate Variable & Variable \\
\midrule
$\tilde{U}$ & $T,\,-p,\,\mu$ & $S,\,V,\,N$ \\
$\tilde{S}$ & $\frac{1}{T},\,\frac{p}{T},\,-\frac{\mu}{T}$ & $U,\,V,\,N$ \\
\bottomrule
\end{tabular}
\caption{Conjugate variables for $\tilde{U}$ and $\tilde{S}$.}
\label{tab:thermos26-05-conjugate-us}
\end{table}

\begin{definition}
\label{def:helmholtz}
\textup{(5.5)} Suppose that all of the
Thermodynamic Postulates hold. Assume that
the material in question is unary ($r=1$). Define
\begin{equation}
\tilde{F}(T,V,N)=\tilde{U}(S_T,V,N)-TS_T
\tag{5.4}\label{eq:helmholtz-def}
\end{equation}
where $S_T=S_T(V,N)$ is the unique solution of the
equation
\begin{equation}
T_U(S_T(V,N),V,N)=T
\tag{5.5}\label{eq:helmholtz-st}
\end{equation}
for fixed values of $N$ and $V$.

$\tilde{F}$ is called the Helmholtz free energy.

Formally, we have
\begin{align*}
 d\tilde{F}&=d\tilde{U}-S\,dT-T\,dS\\
&=T\,dS-p\,dV+\mu\,dN-S\,dT-T\,dS\\
&=-S\,dT-p\,dV+\mu\,dN.
\end{align*}

\begin{table}[h]
\centering
\begin{tabular}{lll}
\toprule
Extensive Quantity & Conjugate Variable & Variable \\
\midrule
$\tilde{U}$ & $T,\,-p,\,\mu$ & $S,\,V,\,N$ \\
$\tilde{F}$ & $-S,\,-p,\,\mu$ & $T,\,V,\,N$ \\
\bottomrule
\end{tabular}
\caption{Conjugate variables for $\tilde{U}$ and $\tilde{F}$.}
\label{tab:thermos26-05-conjugate-uf}
\end{table}
\end{definition}

\begin{example}
\label{ex:helmholtz-example}
\textup{(5.6)} Suppose that
\[
\tilde{S}(U,V,N)=\left(\frac{NVUR^2}{v_0\theta}\right)^{1/3}.
\]

It follows that
\[
\tilde{U}(S,V,N)=\frac{S^3 v_0\theta}{NVR^2}.
\]

The temperature function is
\[
T_U(S,V,N)=\frac{3S^2 v_0\theta}{NVR^2}.
\]

Or
\[
\frac{1}{T_S(U,V,N)}=\frac{1}{3}\left(\frac{NVUR^2}{v_0\theta}\right)^{-2/3}\frac{NV R^2}{v_0\theta}
=\frac{1}{3}\left(\frac{NV R^2}{v_0\theta}\right)^{1/3}U^{-2/3}.
\]

Thus
\[
T_S(U,V,N)=3\left(\frac{v_0\theta}{NVR^2}\right)^{1/3}U^{2/3}.
\]

Recall that
\[
T_U(S,V,\vec{N})=T_S(\tilde{U}(S,V,\vec{N}),V,\vec{N})
\]

\[
T_S(U,V,\vec{N})=T_U(\tilde{S}(U,V,\vec{N}),V,\vec{N}).
\]

By definition
\[
\tilde{F}(T,V,N)=\tilde{U}(S_T,V,N)-TS_T
\]

where

\[
T_U(S_T(V,N),V,N)=T.
\]

For our example,
\[
T_U(S,V,N)=\frac{3S^2 v_0\theta}{NVR^2}.
\]

Hence,
\[
S_T=\sqrt{\frac{TNV R^2}{3v_0\theta}}.
\]

Thus,
\begin{align*}
\tilde{F}(T,V,N)&=\tilde{U}\left(\sqrt{\frac{TNV R^2}{3v_0\theta}},V,N\right)
-T\sqrt{\frac{TNV R^2}{3v_0\theta}}\\
&=\left(\frac{TNV R^2}{3v_0\theta}\right)^{3/2}\frac{v_0\theta}{NVR^2}
-T\sqrt{\frac{TNV R^2}{3v_0\theta}}.
\end{align*}

But
\[
\tilde{U}(S_T,V,N)=\frac{S_T^3 v_0\theta}{NVR^2}
=\left(\frac{TNV R^2}{3v_0\theta}\right)^{3/2}\frac{v_0\theta}{NVR^2}
\]

so that
\[
\tilde{F}(T,N,V)=\frac{1}{3}T^{3/2}\frac{1}{\sqrt{3}}\sqrt{\frac{NV R^2}{v_0\theta}}
-\frac{T^{3/2}}{\sqrt{3}}\sqrt{\frac{NV R^2}{v_0\theta}}.
\]
\end{example}

Finally,
\begin{equation}
\tilde{F}(T,N,V)=-\frac{2T^{3/2}}{3}\sqrt{\frac{NV R^2}{3v_0\theta}}.
\tag{5.6}\label{eq:helmholtz-final}
\end{equation}

Now,
\[
\frac{\partial \tilde{F}}{\partial T}=-\frac{2}{3}\frac{3}{2}\sqrt{T}\,\sqrt{\frac{NV R^2}{3v_0\theta}}.
\]

But
\[
T_U(S,V,N)=\frac{3S^2 v_0\theta}{NVR^2}
\]

Thus,
\[
\sqrt{T}=\sqrt{\frac{3v_0\theta}{NVR^2}}\,S_T
\]

and
\begin{equation}
\frac{\partial \tilde{F}}{\partial T}=-S_T.
\tag{5.7}\label{eq:helmholtz-deriv}
\end{equation}

\begin{theorem}
\label{thm:helmholtz-proc}
\textup{(5.7)} Suppose that all of the
Thermodynamic Postulates hold. Assume that
the material in question is unary ($r=1$). The
following procedure is equivalent means for finding
the Helmholtz free energy:
\begin{equation}
\tilde{F}(T,V,N)=U_T-T\tilde{S}(U_T,V,N)
\tag{5.8}\label{eq:helmholtz-proc}
\end{equation}
\end{theorem}

where $U_T=U_T(V,N)$ is the unique solution of the
equation
\[
T_S(U_T(V,N),V,N)=T
\]

for fixed values of $N$ and $V$.

Proof: Exercise ///

\begin{example}
\label{ex:ideal-gas-helmholtz}
\textup{(5.8)} Let us use the second version
to calculate the free energy for the monatomic
ideal gas. Recall,
\[
\tilde{S}(U,V,N)=Ns_0+NR\ln\left(\left(\frac{U}{U_0}\right)^{3/2}\left(\frac{V}{V_0}\right)\left(\frac{N}{N_0}\right)^{-5/2}\right),
\]
where $s_0,R,U_0,V_0,N_0$ are positive
constants.

Then
\[
\frac{\partial \tilde{S}}{\partial U}
=\frac{NR}{\left(\frac{U}{U_0}\right)^{3/2}}\cdot \frac{3}{2}\left(\frac{U}{U_0}\right)^{1/2}\frac{1}{U_0}
\]

\[
=\frac{3NR}{2U_0}\cdot \frac{U_0}{U}
=\frac{3NR}{2U}
\]

Thus
\[
T_S(U,V,N)=\frac{2}{3}\frac{U}{NR}
\]

and
\[
U_T=\frac{3NRT}{2}
\]

Therefore,
\begin{align*}
\tilde{F}(T,V,N)&=\frac{3NRT}{2}-TN s_0\\
&\phantom{=} -NR T\ln\left(\left(\frac{3NRT}{2U_0}\right)^{3/2}\left(\frac{V}{V_0}\right)\left(\frac{N}{N_0}\right)^{-5/2}\right)
\end{align*}

Define
\[
T_0:=\frac{2U_0}{3N_0R}.
\]

Then,
\begin{equation}
\tilde{F}(T,V,N)=\frac{3NRT}{2}-TN s_0
-NRT\ln\left(\left(\frac{T}{T_0}\right)^{3/2}\left(\frac{V}{V_0}\right)\left(\frac{N}{N_0}\right)^{-1}\right)
\tag{5.9}\label{eq:ideal-helmholtz}
\end{equation}
\end{example}

\section{Equilibrium with a heat Bath}

Consider two isolated systems that are
initially isolated from each other.
\begin{figure}[h]
\centering
\includegraphics[width=0.85\textwidth]{Figures/ThermoS26-05-fig-3.pdf}
\caption{Two isolated systems initially separated by insulating walls.}
\label{fig:thermos26-05-isolated-systems}
\end{figure}

Now, let us remove the isolating walls so
that energy, volume, and mass may be
exchanged.
\begin{figure}[h]
\centering
\includegraphics[width=0.85\textwidth]{Figures/ThermoS26-05-fig-4.pdf}
\caption{Isolating wall removed so energy, volume, and mass can be exchanged.}
\label{fig:thermos26-05-wall-removed}
\end{figure}

The entropy will attain its maximum at equilibrium and
\[
T^\alpha=T^\beta
\]
\[
P^\alpha=P^\beta
\]
\[
\mu_i^\alpha=\mu_i^\beta,\qquad i=1,\ldots,r.
\]

How can we characterize equilibrium?

Now, suppose that instead of isolating from
the universe, our composite system is put
into contact with a heat bath.

\begin{definition}
\label{def:heat-bath}
\textup{(5.9)} A heat bath is an otherwise-
isolated thermodynamic system that is so large
that, when it exchanges a finite amount of
energy with an otherwise-isolated,
composite system, its temperature, pressure, and
chemical potential changes are so small as to be
negligible.
\end{definition}

Similarly, we have

heat-pressure bath: finite exchanges of energy and volume
lead to negligible changes in $T$, $P$, and $\mu$ in baths.

A heat-pressure-chemical bath is defined analogously.

Let us consider a picture for the case of the
heat bath.
\begin{figure}[h]
\centering
\includegraphics[width=0.9\textwidth]{Figures/ThermoS26-05-fig-5.pdf}
\caption{Composite system in contact with a heat bath.}
\label{fig:thermos26-05-heat-bath}
\end{figure}

Let us recall the "integrated form" of the
entropy function.
\[
\tilde{S}^B=\tilde{S}^B(U^B,V^B,N^B)
=\frac{1}{T^0}U^B+\frac{P^0}{T^0}V^B-\frac{\mu^0}{T^0}\cdot N^B
\]

where
\[
T^0=T^0(U^0,V^0,N^0)
\]

et cetera. We will write
\[
S^{tot}=\tilde{S}^B+\tilde{S}^\alpha+\tilde{S}^\beta
\]
\[
=\tilde{S}^B+\tilde{S}^{\alpha+\beta}
\]

Now, our system will not exchange volume or
mass with the bath, only energy, because
we will replace the horizontal isolations
wall with a diathermal wall.
\begin{figure}[h]
\centering
\includegraphics[width=0.9\textwidth]{Figures/ThermoS26-05-fig-6.pdf}
\caption{Diathermal wall configuration: only energy exchange with the bath.}
\label{fig:thermos26-05-diathermal-wall}
\end{figure}

We know that, at equilibrium
\[
T^\alpha=T^\beta=T^B=T_0.
\]

Recall the deviations of $T^B$ from $T_0$ at equilibrium
is assumed negligible.

System $\alpha$ and $\beta$ can exchange energy.
System $\alpha$ and $\beta$ can exchange volume and mass
with each other but not with the bath.

Thus,
\[
S^{tot,eq}=\max_{C}\{\tilde{S}^B+\tilde{S}^{\alpha+\beta}\}
\]

Now
\[
\tilde{S}^B=\frac{1}{T_0}U^B+\frac{P_0}{T_0}V^B-\frac{\mu_0}{T_0}\cdot N^B
\]

The changes in $T^B$, $P^B$, $\mu^B$ from $T_0$, $P_0$, $\mu_0$ is
negligible because the bath is so large.

Writing
\[
U^B=U_0-U^\alpha-U^\beta
\]

\[
\tilde{S}^B=-\frac{U^\alpha}{T_0}-\frac{U^\beta}{T_0}+\frac{U_0}{T_0}+C_1
\]

\[
=-\frac{U^\alpha}{T_0}-\frac{U^\beta}{T_0}+C_2
\]

Thus,
\begin{align*}
S^{tot,eq}&=\max_{C}\left\{-\frac{U^\alpha}{T_0}-\frac{U^\beta}{T_0}+C_2+\tilde{S}^{\alpha+\beta}\right\}\\
&=C_2+\max_{C}\left\{-\frac{U^\alpha+U^\beta}{T_0}+\tilde{S}^\alpha+\tilde{S}^\beta\right\}\\
&=C_2+\max_{C}\left\{-\frac{F^\alpha+F^\beta}{T_0}\right\}\\
&=C_2-\frac{1}{T_0}\min_{C}\left\{F^\alpha(T_0,V^\alpha,N^\alpha)+F^\beta(T_0,V^\beta,N^\beta)\right\}
\end{align*}

Postulate VI: Consider a compound, otherwise-
isolated thermodynamic system in contact with
an otherwise-isolated heat bath at temperature
$T_0$. Equilibrium of the compound system is the
state satisfying the isothermal condition
\[
T^\alpha=T^\beta=T_0
\]

which minimizes
\[
F^\alpha(T_0,V^\alpha,N^\alpha)+F^\beta(T_0,V^\beta,N^\beta)
\]

subject to the constraints
\[
V^\alpha+V^\beta=V_0,\qquad\text{(volume conservation)}
\]
\[
N_i^\alpha+N_i^\beta=N_{i,0},\qquad\text{(mass conservation)}
\]

Remark: Since the system $\alpha+\beta$ exchanges
energy with the heat bath, energy is not
conserved in $\alpha+\beta$.

Energy is conserved in the system $B+\alpha+\beta$.

\THERMOEND
