\newif\ifthermosubfile
\ifdefined\THERMONOTESMAIN
  \thermosubfilefalse
\else
  \thermosubfiletrue
\fi

\providecommand{\THERMOEND}{} % safe in both modes

\ifthermosubfile
  \documentclass[11pt]{book}
  \usepackage{cmbright}
\usepackage{mathtools}

\usepackage{pgfplots}
\usepgfplotslibrary{patchplots}
\pgfplotsset{compat=1.15}

\usepackage{color}
\usepackage{xcolor}
\usepackage{subfigure}

\usepackage{amssymb}
\usepackage{amsmath}
\usepackage{amsfonts}
\usepackage{amsthm} 
\usepackage{mdframed}
\usepackage{bbm}
\usepackage{listings}

\usepackage[shortlabels]{enumitem}

\usepackage{tikz}
\usepackage{tikz-3dplot}
\usepackage{xkeyval,tkz-base}

\usetikzlibrary{patterns}
\usetikzlibrary{arrows,arrows.meta,shapes,decorations,automata}
\usetikzlibrary{backgrounds,petri,matrix}
\usetikzlibrary{intersections}
\usetikzlibrary{calc}
\usetikzlibrary{positioning}
\usetikzlibrary{shapes.geometric}
\usetikzlibrary{shapes.arrows}
\usetikzlibrary{matrix}
\usetikzlibrary{positioning}
\usetikzlibrary{3d,calc}

\usepackage{multirow}
\usepackage{array}
\usepackage{algorithm}
\usepackage{algpseudocode}

\usepackage{graphicx}
\usepackage{latexsym}
\usepackage{booktabs}
\usepackage{longtable}

%\usepackage{aligned-overset}

\numberwithin{equation}{section}

\usepackage{lastpage}
\usepackage{fancyhdr}
\pagestyle{fancy}
\setlength{\headheight}{14pt}
\lhead{Mathematical Thermodynamics}
\chead{}
\rhead{Page \thepage}
\cfoot{Page\ \thepage\ of\ \protect\pageref{LastPage}}
\setcounter{section}{-1}

% Theorem environments
\theoremstyle{plain}
\newtheorem{theorem}{Theorem}[section]
\newtheorem{lemma}[theorem]{Lemma}
\newtheorem{proposition}[theorem]{Proposition}
\newtheorem{corollary}[theorem]{Corollary}

\theoremstyle{definition}
\newtheorem{definition}[theorem]{Definition}
\newtheorem{example}[theorem]{Example}
\newtheorem{exercise}[theorem]{Exercise}

\theoremstyle{remark}
\newtheorem{remark}[theorem]{Remark}
\newtheorem{note}[theorem]{Note}

% Hyperref setup
\usepackage{hyperref}
\usepackage{nameref}
\hypersetup{
    hypertexnames=true,
    breaklinks=false,
    colorlinks=true,
    linkcolor=blue,
    citecolor=blue,
    urlcolor=blue
}

\makeatletter
  \newenvironment{dedication}
  {
     \cleardoublepage
     \thispagestyle{empty}
     \vspace*{\stretch{1}}
     \hfill\begin{minipage}[t]{0.66\textwidth}
     \raggedright
  }%
  {
     \end{minipage}
     \vspace*{\stretch{3}}
     \clearpage
  }
  \newenvironment{varsubequations}[1]
  {%
    \addtocounter{equation}{-1}%
    \begin{subequations}
    \renewcommand{\theparentequation}{#1}%
    \def\@currentlabel{#1}%
  }
  {%
    \end{subequations}\ignorespacesafterend
  }
\makeatother

  % ====================================
% THERMODYNAMIC STATE VARIABLES
% ====================================

% Extensive properties
\newcommand{\Int}{U}           % Internal energy
\newcommand{\Ent}{S}           % Entropy
\newcommand{\Vol}{V}           % Volume
\newcommand{\Enth}{H}          % Enthalpy H = U + PV
\newcommand{\Helm}{A}          % Helmholtz free energy A = U - TS
\newcommand{\Gibbs}{G}         % Gibbs free energy G = H - TS
\newcommand{\GrandPot}{\Omega} % Grand potential

% Intensive properties
\newcommand{\Temp}{T}          % Temperature
\newcommand{\Press}{P}         % Pressure
\newcommand{\ChemPot}{\mu}     % Chemical potential
\newcommand{\Mass}{m}          % Mass
\newcommand{\Moles}{n}         % Number of moles

% Specific properties (per unit mass)
\newcommand{\sint}{u}          % Specific internal energy
\newcommand{\sent}{s}          % Specific entropy
\newcommand{\svol}{v}          % Specific volume
\newcommand{\senth}{h}         % Specific enthalpy

% Heat and work
\newcommand{\Heat}{Q}          % Heat transfer
\newcommand{\Work}{W}          % Work
\newcommand{\dQ}{\delta Q}     % Inexact differential of heat
\newcommand{\dW}{\delta W}     % Inexact differential of work

% Thermodynamic processes
\newcommand{\iso}{\text{iso}}
\newcommand{\rev}{\text{rev}}
\newcommand{\irrev}{\text{irrev}}

% ====================================
% MATHEMATICAL OPERATORS
% ====================================

\DeclareMathOperator{\grad}{grad}
\DeclareMathOperator{\dive}{div}
\DeclareMathOperator{\curl}{curl}
\DeclareMathOperator{\Span}{span}
\DeclareMathOperator{\rank}{rank}
\DeclareMathOperator{\trace}{tr}
\DeclareMathOperator{\sgn}{sgn}
\DeclareMathOperator{\diam}{diam}

% Partial derivatives
\newcommand{\pder}[2]{\frac{\partial #1}{\partial #2}}
\newcommand{\pdertwo}[2]{\frac{\partial^2 #1}{\partial #2^2}}
\newcommand{\pdermix}[3]{\frac{\partial^2 #1}{\partial #2 \partial #3}}

% Thermodynamic partial derivatives
\newcommand{\pderT}[3]{\left(\frac{\partial #1}{\partial #2}\right)_{#3}}

% Total derivatives
\newcommand{\dder}[2]{\frac{d #1}{d #2}}
\newcommand{\ddertwo}[2]{\frac{d^2 #1}{d #2^2}}

% Differential operators
\newcommand{\dd}{\mathrm{d}}
\newcommand{\dif}{\mathrm{d}}

% ====================================
% VECTOR AND MATRIX NOTATION
% ====================================

% Bold vectors
\newcommand{\bfv}[1]{\boldsymbol{#1}}
\newcommand{\bfx}{\boldsymbol{x}}
\newcommand{\bfy}{\boldsymbol{y}}
\newcommand{\bfz}{\boldsymbol{z}}
\newcommand{\bfu}{\boldsymbol{u}}
\newcommand{\bfv}{\boldsymbol{v}}
\newcommand{\bfw}{\boldsymbol{w}}
\newcommand{\bff}{\boldsymbol{f}}
\newcommand{\bfg}{\boldsymbol{g}}
\newcommand{\bfF}{\boldsymbol{F}}
\newcommand{\bfn}{\boldsymbol{n}}
\newcommand{\bfe}{\boldsymbol{e}}
\newcommand{\bfzero}{\boldsymbol{0}}

% Matrix spaces
\newcommand{\Rn}{\mathbb{R}^n}
\newcommand{\Rm}{\mathbb{R}^m}
\newcommand{\Rmn}{\mathbb{R}^{m \times n}}

% ====================================
% SETS AND SPACES
% ====================================

\newcommand{\R}{\mathbb{R}}    % Real numbers
\newcommand{\C}{\mathbb{C}}    % Complex numbers
\newcommand{\N}{\mathbb{N}}    % Natural numbers
\newcommand{\Z}{\mathbb{Z}}    % Integers
\newcommand{\Q}{\mathbb{Q}}    % Rationals

% Function spaces
\newcommand{\Lp}[1]{L^{#1}}
\newcommand{\Hone}{H^1}
\newcommand{\Ck}[1]{C^{#1}}

% ====================================
% SPECIAL SYMBOLS
% ====================================

% Indicator/characteristic function
\newcommand{\ind}{\mathbbm{1}}

% Inner product
\newcommand{\inner}[2]{\langle #1, #2 \rangle}

% Norm
\newcommand{\norm}[1]{\left\| #1 \right\|}
\newcommand{\abs}[1]{\left| #1 \right|}

% Set notation
\newcommand{\set}[1]{\left\{ #1 \right\}}
\newcommand{\setdef}[2]{\left\{ #1 \,:\, #2 \right\}}

% Ceiling and floor
\newcommand{\ceil}[1]{\left\lceil #1 \right\rceil}
\newcommand{\floor}[1]{\left\lfloor #1 \right\rfloor}

% ====================================
% MATHCAL AND MATHSF LETTERS
% ====================================

% Calligraphic letters for sets and spaces
\newcommand{\calA}{\mathcal{A}}
\newcommand{\calB}{\mathcal{B}}
\newcommand{\calC}{\mathcal{C}}
\newcommand{\calD}{\mathcal{D}}
\newcommand{\calE}{\mathcal{E}}
\newcommand{\calF}{\mathcal{F}}
\newcommand{\calG}{\mathcal{G}}
\newcommand{\calH}{\mathcal{H}}
\newcommand{\calI}{\mathcal{I}}
\newcommand{\calJ}{\mathcal{J}}
\newcommand{\calK}{\mathcal{K}}
\newcommand{\calL}{\mathcal{L}}
\newcommand{\calM}{\mathcal{M}}
\newcommand{\calN}{\mathcal{N}}
\newcommand{\calO}{\mathcal{O}}
\newcommand{\calP}{\mathcal{P}}
\newcommand{\calQ}{\mathcal{Q}}
\newcommand{\calR}{\mathcal{R}}
\newcommand{\calS}{\mathcal{S}}
\newcommand{\calT}{\mathcal{T}}
\newcommand{\calU}{\mathcal{U}}
\newcommand{\calV}{\mathcal{V}}
\newcommand{\calW}{\mathcal{W}}
\newcommand{\calX}{\mathcal{X}}
\newcommand{\calY}{\mathcal{Y}}
\newcommand{\calZ}{\mathcal{Z}}

% Sans-serif letters for matrices and operators
\newcommand{\msfA}{\mathsf{A}}
\newcommand{\msfB}{\mathsf{B}}
\newcommand{\msfC}{\mathsf{C}}
\newcommand{\msfD}{\mathsf{D}}
\newcommand{\msfE}{\mathsf{E}}
\newcommand{\msfF}{\mathsf{F}}
\newcommand{\msfG}{\mathsf{G}}
\newcommand{\msfH}{\mathsf{H}}
\newcommand{\msfI}{\mathsf{I}}
\newcommand{\msfJ}{\mathsf{J}}
\newcommand{\msfK}{\mathsf{K}}
\newcommand{\msfL}{\mathsf{L}}
\newcommand{\msfM}{\mathsf{M}}
\newcommand{\msfN}{\mathsf{N}}
\newcommand{\msfO}{\mathsf{O}}
\newcommand{\msfP}{\mathsf{P}}
\newcommand{\msfQ}{\mathsf{Q}}
\newcommand{\msfR}{\mathsf{R}}
\newcommand{\msfS}{\mathsf{S}}
\newcommand{\msfT}{\mathsf{T}}
\newcommand{\msfU}{\mathsf{U}}
\newcommand{\msfV}{\mathsf{V}}
\newcommand{\msfW}{\mathsf{W}}
\newcommand{\msfX}{\mathsf{X}}
\newcommand{\msfY}{\mathsf{Y}}
\newcommand{\msfZ}{\mathsf{Z}}

% ====================================
% COMMON ABBREVIATIONS
% ====================================

\newcommand{\ie}{i.e.\ }
\newcommand{\eg}{e.g.\ }
\newcommand{\cf}{cf.\ }
\newcommand{\etc}{etc.\ }
\newcommand{\vs}{vs.\ }
\newcommand{\resp}{resp.\ }
\newcommand{\etal}{et al.\ }

  \graphicspath{{./Figures/}}
  \begin{document}
  \renewcommand{\THERMOEND}{\end{document}}
\fi


\chapter{ThermoS26-05}
\label{chap:ThermoS26-05}

\section{Examples}

\begin{example}
\label{ex:fundamental-relation-homogeneous}
\textup{(3.4)} Suppose that the fundamental relation for a material is given by
\[
\tilde{S}=4 A\, U^{1/4} V^{1/2} N^{1/4}+B N,\qquad \Sigma_U=[0,\infty)^3,
\]
where $A,B>0$ are constants. This function must be homogenous of degree one. Suppose $\lambda>0$. Then
\[
\tilde{S}(\lambda U,\lambda V,\lambda N)=4A\,\lambda U^{1/4}\lambda V^{1/2}\lambda N^{1/4}+B\lambda N
=\lambda\tilde{S}(U,V,N). ///
\]

Recall that
\[
T=\frac{1}{\frac{\partial \tilde{S}}{\partial U}}
=\Big(A\,U^{-3/4}V^{1/2}N^{1/4}\Big)^{-1}
=\frac{U^{3/4}}{A V^{1/2}N^{1/4}},
\]
which is homogenous degree zero.
\[
\frac{P}{T}=\frac{\partial \tilde{S}}{\partial V}
\]
so
\[
P=T\,\frac{\partial \tilde{S}}{\partial V}
=\frac{U^{3/4}}{A V^{1/2} N^{1/4}}\left(\frac{2A\,U^{1/4}N^{1/4}}{V^{1/2}}\right)
=\frac{2U}{V}.
\]
\end{example}

Finally,
\[
\mu=-T\,\frac{\partial \tilde{S}}{\partial N}
=-\frac{U^{3/4}}{A V^{1/2}N^{1/4}}\left(\frac{A\,U^{1/4}V^{1/2}}{N^{3/4}}+B\right)
=-\frac{U}{N}-\frac{B U^{3/4}}{A V^{1/2}N^{1/4}}. ///
\]

\begin{example}
\label{ex:fundamental-relation-u}
\textup{(3.5)} Suppose that the fundamental relation is
\[
\tilde{U}=\left(\frac{S-BN}{4A V^{1/2}N^{1/4}}\right)^4.
\]
Recall that
\[
T=\frac{\partial \tilde{U}}{\partial S}
=4\left(\frac{S-BN}{4A V^{1/2}N^{1/4}}\right)^3\frac{1}{4A V^{1/2}N^{1/4}}
=\frac{\tilde{U}^{3/4}}{A V^{1/2}N^{1/4}},
\]
the same as above.
\[
P=-\frac{\partial \tilde{U}}{\partial V}
=-4\left(\frac{S-BN}{4A V^{1/2}N^{1/4}}\right)^3\left(\frac{S-BN}{4A N^{1/4}}\right)\left(-\frac{1}{2}\right)\frac{1}{V^{3/2}}
\]
so
\[
=2\left(\frac{S-BN}{4A V^{1/2}N^{1/4}}\right)^4\frac{1}{V}
=\frac{2\tilde{U}}{V}.
\]
Finally,
\[
\mu=\frac{\partial \tilde{U}}{\partial N}
=4\left(\frac{S-BN}{4A V^{1/2}N^{1/4}}\right)^3\frac{1}{4A V^{1/2}}\frac{N^{1/4}(-B)-(S-BN)\frac{1}{4}N^{-3/4}}{N^{1/2}}
\]
\[
=4\left(\frac{S-BN}{4A V^{1/2}N^{1/4}}\right)^3\frac{1}{4A V^{1/2}}\frac{N^{1/4}(-B)-(S-BN)\frac{1}{4}N^{-3/4}}{N^{1/2}}
\]
\[
=4\tilde{U}^{3/4}\frac{N^{1/4}}{4A V^{1/2}N^{1/2}}\cdot\left(-B-\frac{S-BN}{4N}\right)
\]
\[
=-\frac{B\tilde{U}^{3/4}}{A V^{1/2}N^{1/4}}-\frac{\tilde{U}^{3/4}}{A V^{1/2}N^{1/4}}\frac{(S-BN)}{4N}
=-\frac{B\tilde{U}^{3/4}}{A V^{1/2}N^{1/4}}-\frac{\tilde{U}}{N}. ///
\]
\end{example}

\section{Euler Equation with Respect to Entropy}

\begin{theorem}
\label{thm:euler-entropy}
\textup{(3.5)} Let $\tilde{S}$ be the internal energy of an isolated system. Then
\begin{equation}
\tilde{S}=\frac{1}{T_S}U+\frac{P_S}{T_S}V-\sum_{i=1}^r\frac{\mu_{S,i}}{T_S}N_i,
\tag{3.8}\label{eq:euler-entropy}
\end{equation}
where
\[
T_S=T_S(U,V,\vec{N}),\qquad
P_S=P_S(U,V,\vec{N}),\qquad
\mu_{S,i}=\mu_{S,i}(U,V,\vec{N}),
\]
and
\[
\frac{1}{T_S}=\frac{\partial \tilde{S}}{\partial U},\qquad \frac{P_S}{T_S}=\frac{\partial \tilde{S}}{\partial V},\qquad \frac{\mu_{S,i}}{T_S}=\frac{\partial \tilde{S}}{\partial N_i}.
\]
\end{theorem}

\begin{proof}
We again use the fact that $\tilde{S}$ is homogenous of degree one. For any $\lambda>0$,
\[
\tilde{S}(\lambda U,\lambda V,\lambda\vec{N})=\lambda\tilde{S}(U,V,\vec{N}).
\]
Taking the derivative with respect to $\lambda$, we have
\[
\frac{U}{T_S(\lambda U,\lambda V,\lambda\vec{N})}
+\frac{P_S(\lambda U,\lambda V,\lambda\vec{N})}{T_S(\lambda U,\lambda V,\lambda\vec{N})}V
+\sum_{j=1}^r\frac{\mu_{S,j}(\lambda U,\lambda V,\lambda\vec{N})}{T_S(\lambda U,\lambda V,\lambda\vec{N})}N_j
=\tilde{S}(U,V,\vec{N}),
\]
and setting $\lambda=1$ gives the desired result. ///
\end{proof}

\section{Gibbs-Duhem Relation in the Entropy Form}

The Gibbs-Duhem equation is similarly derived.

\begin{theorem}
\label{thm:gibbs-duhem-entropy}
\textup{(3.6)} Suppose that $\vec{Y}:[0,1]\to\Sigma_U$ is a process path. Then
\begin{equation}
0=\frac{dT_S}{d\gamma}\big(\vec{Y}(\gamma)\big)\,\tilde{S}\big(\vec{Y}(\gamma)\big)
-\frac{dP_S}{d\gamma}\big(\vec{Y}(\gamma)\big)V(\gamma)
+\sum_{j=1}^r\frac{d\mu_S^j}{d\gamma}\big(\vec{Y}(\gamma)\big)N_j(\gamma).
\tag{3.9}\label{eq:gibbs-duhem-entropy}
\end{equation}
This equation is called the Gibbs-Duhem relation in the entropy form.
\end{theorem}

\begin{proof}
Using \eqref{eq:euler-entropy}, we have
\[
T_S\big(\vec{Y}(\gamma)\big)\,\tilde{S}\big(\vec{Y}(\gamma)\big)
=U(\gamma)+P_S\big(\vec{Y}(\gamma)\big)V(\gamma)
-\sum_{j=1}^r \mu_{S,j}\big(\vec{Y}(\gamma)\big)N_j(\gamma).
\]
Taking the $\gamma$-derivative of the last equation, we have
\begin{align}
\frac{dT_S}{d\gamma}\big(\vec{Y}(\gamma)\big)\,\tilde{S}\big(\vec{Y}(\gamma)\big)
+T_S\big(\vec{Y}(\gamma)\big)\frac{d\tilde{S}}{d\gamma}\big(\vec{Y}(\gamma)\big)
&=U'(\gamma)+\frac{dP_S}{d\gamma}\big(\vec{Y}(\gamma)\big)V(\gamma)+P_S\big(\vec{Y}(\gamma)\big)V'(\gamma) \notag \\
&\quad -\sum_{j=1}^r\left\{\frac{d\mu_{S,j}}{d\gamma}\big(\vec{Y}(\gamma)\big)N_j(\gamma)+\mu_{S,j}\big(\vec{Y}(\gamma)\big)N_j'(\gamma)\right\}.
\tag{3.10}\label{eq:gibbs-duhem-entropy-derivative}
\end{align}
Taking the $\gamma$-derivative of $\tilde{S}\big(\vec{Y}(\gamma)\big)$ we have
\begin{equation}
\frac{d\tilde{S}}{d\gamma}\big(\vec{Y}(\gamma)\big)
=\frac{1}{T_S(\vec{Y}(\gamma))}U'(\gamma)+\frac{P_S(\vec{Y}(\gamma))}{T_S(\vec{Y}(\gamma))}V'(\gamma)
-\sum_{j=1}^r\frac{\mu_{S,j}(\vec{Y}(\gamma))}{T_S(\vec{Y}(\gamma))}N_j'(\gamma).
\tag{3.11}\label{eq:entropy-derivative}
\end{equation}
Substituting \eqref{eq:entropy-derivative} into \eqref{eq:gibbs-duhem-entropy-derivative} yields \eqref{eq:gibbs-duhem-entropy}. ///
\end{proof}

Remark: Compare \eqref{eq:gibbs-duhem} and \eqref{eq:gibbs-duhem-entropy}:
\[
0=S(\gamma)\frac{dT_S(\vec{Y}(\gamma))}{d\gamma}-V(\gamma)\frac{dP_S(\vec{Y}(\gamma))}{d\gamma}
+\sum_{i=1}^r N_i(\gamma)\frac{d\mu_S^i(\vec{Y}(\gamma))}{d\gamma},
\]
\[
0=\frac{dT_S}{d\gamma}\big(\vec{Y}(\gamma)\big)\,\tilde{S}\big(\vec{Y}(\gamma)\big)
-\frac{dP_S}{d\gamma}\big(\vec{Y}(\gamma)\big)V(\gamma)
+\sum_{j=1}^r\frac{d\mu_S^j}{d\gamma}\big(\vec{Y}(\gamma)\big)N_j(\gamma),
\]
These are essentially the same expression!

\noindent Math Thermo\\
Class \#04\\
01/29/2026

Suppose that $\tilde{S}$ is the fundamental entropy relation
for an isolated system containing a unary ($r=1$)
material.
\[
\tilde{S}=\tilde{S}(U,V,N).
\]

Suppose that $N>0$ is fixed. Define
\[
 u:=\frac{U}{N},\qquad\text{the molar energy,}
\]
\[
 v:=\frac{V}{N},\qquad\text{the molar volume.}
\]

Then, since, for any $\lambda>0$,
\[
\lambda \tilde{S}(U,V,N)=\tilde{S}(\lambda U,\lambda V,\lambda N),
\]

it follows that
\[
\frac{1}{N}\tilde{S}(U,V,N)=\tilde{S}\left(\frac{U}{N},\frac{V}{N},\frac{N}{N}\right)
=\tilde{S}(u,v,1).
\]

Now, define the molar entropy relation
\[
\tilde{s}=\tilde{s}(u,v):=\tilde{S}(u,v,1).
\]

It then follows that
\[
\tilde{S}(U,V,N)=N\tilde{s}(u,v).
\]

Likewise, if we define
\[
 s=\frac{S}{N}\qquad v=\frac{V}{N}
\]

and
\[
\tilde{u}=\tilde{u}(s,v):=\tilde{U}(s,v,1),
\]

then it follows that
\[
\tilde{U}(S,V,N)=N\tilde{u}(s,v).
\]

\begin{example}
\label{ex:molar-entropy-example}
\textup{(4.1)} Suppose that
\[
\tilde{S}(U,V,N)=\left(\frac{NVUR^2}{v_0\theta}\right)^{1/3},
\]
where $R,v_0,\theta>0$ are constants. Then,
\[
\tilde{s}(u,v)=\tilde{S}(u,v,1)
=\left(\frac{vuR^2}{v_0\theta}\right)^{1/3}.
\]

\[
N\tilde{s}(u,v)=\left(\frac{N^3vuR^2}{v_0\theta}\right)^{1/3}
=\left(\frac{NVUR^2}{v_0\theta}\right)^{1/3}
=\tilde{S}(U,V,N).
\]
\end{example}

$\tilde{U}$ and $\tilde{S}$ are homogeneous of degree one (also
called extensive variables).

$T,P,\mu$ on the other hand are homogeneous of
degree zero (also called intensive variables).

Recall that, for all $\lambda>0$,
\[
T_U(S,V,N)=T_U(\lambda S,\lambda V,\lambda N).
\]

Thus,
\[
T_U(S,V,N)=T_U(s,v,1).
\]

We could, of course, give a new symbol for
$T_U(s,v,1)$, for example
\[
 t_U(s,v):=T_U(s,v,1)\ldots
\]

\section{Molar Euler and Gibbs--Duhem Equation}

For a unary material ($r=1$), recall
\[
\tilde{U}(S,V,N)=T_U(S,V,N)S-P_U(S,V,N)V+\mu_U(S,V,N)N
\]

Suppose $N>0$ is fixed. Then,
\[
\frac{1}{N}\tilde{U}(S,V,N)=T_U(S,V,N)\frac{S}{N}-P_U(S,V,N)\frac{V}{N}+\mu_U(S,V,N)
\]

or, equivalently,
\begin{equation}
\tilde{u}(s,v)=T_U(s,v)s-P_U(s,v)v+\mu_U(s,v).
\tag{4.1}\label{eq:molar-euler}
\end{equation}

This is the molar version of Euler's equation.
How do we get the Gibbs--Duhem equation?

Let $\vec{z}(\gamma)=(s(\gamma),v(\gamma))$ be a path in molar
state space. Then
\[
\frac{d}{d\gamma}\tilde{u}(\vec{z}(\gamma))=
\frac{\partial \tilde{u}}{\partial s}(\vec{z}(\gamma))\,s'(\gamma)+
\frac{\partial \tilde{u}}{\partial v}(\vec{z}(\gamma))\,v'(\gamma).
\]

But

exercise
\[
\frac{\partial \tilde{u}}{\partial s}(\vec{z}(\gamma))=T_U(s(\gamma),v(\gamma))
\]

and
\[
\frac{\partial \tilde{u}}{\partial v}(\vec{z}(\gamma))=-P_U(s(\gamma),v(\gamma)).
\]

Thus, the combined First and Second laws are
\begin{equation}
\frac{d}{d\gamma}\tilde{u}(\vec{z}(\gamma))=T_U(\vec{z}(\gamma))s'(\gamma)-P_U(\vec{z}(\gamma))v'(\gamma).
\tag{4.2}\label{eq:combined-laws}
\end{equation}

On the other hand, differentiate the molar Euler
relation (4.1), we have
\begin{align}
\frac{d}{d\gamma}\tilde{u}(\vec{z}(\gamma))&=\frac{d}{d\gamma}T_U(\vec{z}(\gamma))s(\gamma)+T_U(\vec{z}(\gamma))s'(\gamma)\\
&\quad-\frac{d}{d\gamma}P_U(\vec{z}(\gamma))v(\gamma)-P_U(\vec{z}(\gamma))v'(\gamma)\\
&\quad+\frac{d}{d\gamma}\mu_U(\vec{z}(\gamma)).
\tag{4.3}\label{eq:diff-molar-euler}
\end{align}

Therefore, the molar Gibbs--Duhem equation is
\begin{equation}
0=\frac{d}{d\gamma}T_U(\vec{z}(\gamma))s(\gamma)-\frac{d}{d\gamma}P_U(\vec{z}(\gamma))v(\gamma)+\frac{d}{d\gamma}\mu_U(\vec{z}(\gamma)).
\tag{4.4}\label{eq:molar-gd}
\end{equation}

The molar entropy form of the Euler and Gibbs--
Duhem equations are (as one would expect)
\begin{equation}
\tilde{s}(u,v)=\frac{1}{T_S(u,v)}\,u+\frac{P_S(u,v)}{T_S(u,v)}\,v-\frac{\mu_S(u,v)}{T_S(u,v)}.
\tag{4.6}\label{eq:molar-entropy-euler}
\end{equation}

and
\begin{equation}
0=\frac{d}{d\gamma}\left(\frac{1}{T_S(\vec{z}(\gamma))}\right)u(\gamma)+
\frac{d}{d\gamma}\left(\frac{P_S(\vec{z}(\gamma))}{T_S(\vec{z}(\gamma))}\right)v(\gamma)
-\frac{d}{d\gamma}\left(\frac{\mu_S(\vec{z}(\gamma))}{T_S(\vec{z}(\gamma))}\right).
\tag{4.7}\label{eq:molar-entropy-gd}
\end{equation}

The molar representation of the First and
Second laws is
\begin{equation}
\frac{d}{d\gamma}\tilde{s}(\vec{z}(\gamma))=\frac{1}{T_S(\vec{z}(\gamma))}u'(\gamma)+
\frac{P_S(\vec{z}(\gamma))}{T_S(\vec{z}(\gamma))}v'(\gamma).
\tag{4.8}\label{eq:molar-first-second}
\end{equation}

where $\vec{z}(\gamma)=(u(\gamma),v(\gamma))$ is a process path in
molar state space.

\section{Examples}

Let's look at a couple of examples.

\begin{example}
\label{ex:two-eos}
\textup{(4.2)} Suppose that
\[
P=\frac{2U}{V}
\]
\[
T=\left(\frac{A U^{3/2}}{V N^{1/2}}\right)^{1/2}
\]

(homog.\ deg.\ zero)

We should be able to recover the fundamental
relation using these two equations of state.
Observe that
\[
\frac{1}{T_S}=A^{-1/2}u^{-3/4}v^{1/2}
\]
\[
\frac{P_S}{T_S}=2A^{-1/2}u^{1/4}v^{-1/2}
\]

Using the molar 1st and 2nd laws (4.8)
\begin{align*}
\frac{d}{d\gamma}\tilde{s}(\vec{z}(\gamma))&=\frac{1}{T_S(\vec{z}(\gamma))}u'(\gamma)+\frac{P_S(\vec{z}(\gamma))}{T_S(\vec{z}(\gamma))}v'(\gamma)\\
&=A^{-1/2}\left(u(\gamma)^{-3/4}v(\gamma)^{1/2}u'(\gamma)
+2u(\gamma)^{1/4}v(\gamma)^{-1/2}v'(\gamma)\right)\\
&=4A^{-1/2}\frac{d}{d\gamma}\left((u(\gamma))^{1/4}(v(\gamma))^{1/2}\right).
\end{align*}

Therefore,
\[
\tilde{s}(\gamma)=4A^{-1/2}(u(\gamma))^{1/4}(v(\gamma))^{1/2}+s_0.
\]

Equivalently,
\[
\tilde{S}(\gamma)=4A^{-1/2}U^{1/4}V^{1/2}N^{1/4}+Ns_0.
\]

here $s_0$ is a positive constant. ///

Let's try an alternative method.
\end{example}

\begin{example}
\label{ex:gibbs-duhem-start}
\textup{(4.3)} Same problem as above. This
time we will use the Gibbs--Duhem equation
(4.7) as the starting point:

\begin{align*}
\frac{d}{d\gamma}\left(\frac{\mu_S(\vec{z}(\gamma))}{T_S(\vec{z}(\gamma))}\right)
&=\frac{d}{d\gamma}\left(\frac{1}{T_S(\vec{z}(\gamma))}\right)u(\gamma)
+\frac{d}{d\gamma}\left(\frac{P_S(\vec{z}(\gamma))}{T_S(\vec{z}(\gamma))}\right)v(\gamma)\\
&=A^{-1/2}\frac{d}{d\gamma}\left((u(\gamma))^{-3/4}(v(\gamma))^{1/2}\right)u(\gamma)\\
&\quad+A^{-1/2}\frac{d}{d\gamma}\left(2(u(\gamma))^{1/4}(v(\gamma))^{-1/2}\right)\\
&\quad\phantom{+A^{-1/2}\frac{d}{d\gamma}}v(\gamma)\\
&=A^{-1/2}\left(-\frac{3}{4}(u(\gamma))^{-3/4}(v(\gamma))^{1/2}u'(\gamma)\right.\\
&\quad\left.+(u(\gamma))^{1/4}\frac{1}{2}(v(\gamma))^{-1/2}v'(\gamma)\right)\\
&\quad+A^{-1/2}\left(\frac{1}{2}(u(\gamma))^{-3/4}(v(\gamma))^{1/2}u'(\gamma)\right.\\
&\quad\left.-2(u(\gamma))^{1/4}\frac{1}{2}(v(\gamma))^{-1/2}v'(\gamma)\right)\\
&=A^{-1/2}\left(-\frac{1}{4}(u(\gamma))^{-3/4}(v(\gamma))^{1/2}u'(\gamma)\right.\\
&\quad\left.-\frac{1}{2}(u(\gamma))^{1/4}(v(\gamma))^{-1/2}v'(\gamma)\right)\\
&=-A^{-1/2}\frac{d}{d\gamma}\left((u(\gamma))^{1/4}(v(\gamma))^{1/2}\right).
\end{align*}

Thus,
\[
\frac{\mu_S(\vec{z}(\gamma))}{T_S(\vec{z}(\gamma))}=-A^{-1/2}(u(\gamma))^{1/4}(v(\gamma))^{1/2}-s_0.
\]

or
\[
\frac{\mu_S(u,v)}{T_S(u,v)}=-A^{-1/2}u^{1/4}v^{1/2}-s_0.
\]

Using Euler's equation (4.6), we have
\begin{align*}
\tilde{s}(u,v)&=\frac{1}{T_S(u,v)}u+\frac{P_S(u,v)}{T_S(u,v)}v-\frac{\mu_S(u,v)}{T_S(u,v)}\\
&=A^{-1/2}u^{1/4}v^{1/2}+2A^{-1/2}u^{1/4}v^{1/2}
+ A^{-1/2}u^{1/4}v^{1/2}+s_0\\
&=4A^{-1/2}u^{1/4}v^{1/2}+s_0,
\end{align*}

which is the same as before.

Using 2 equations of state, we can recover the
3rd and then utilizing Euler's equation we
get the fundamental relation. ///
\end{example}

\begin{example}
\label{ex:ideal-gas}
\textup{(4.4)} Ideal Gas law

\[
PV=NRT
\]
\[
U=\frac{3}{2}NRT
\]

With these two equations of state we can find the
fundamental relations.

Observe that
\[
\frac{1}{T_S}=\frac{3R}{2u}\qquad\frac{P_S}{T_S}=\frac{R}{v}.
\]

This suggests that we again use the entropy
equation (in molar form). The Gibbs--Duhem equation is
\begin{align*}
\frac{d}{d\gamma}\left(\frac{\mu_S(\vec{z}(\gamma))}{T_S(\vec{z}(\gamma))}\right)
&=\frac{d}{d\gamma}\left(\frac{1}{T_S(\vec{z}(\gamma))}\right)u(\gamma)
+\frac{d}{d\gamma}\left(\frac{P_S(\vec{z}(\gamma))}{T_S(\vec{z}(\gamma))}\right)v(\gamma)\\
&=\frac{3R}{2}\frac{d}{d\gamma}\left(\frac{1}{u(\gamma)}\right)u(\gamma)
+R\frac{d}{d\gamma}\left(\frac{1}{v(\gamma)}\right)v(\gamma)\\
&=-\frac{3R}{2}\frac{u'(\gamma)}{u(\gamma)}-R\frac{v'(\gamma)}{v(\gamma)}.
\end{align*}

Integrating, we have
\[
\frac{\mu_S(u(\gamma),v(\gamma))}{T_S(u(\gamma),v(\gamma))}-\frac{\mu_0}{T_0}=-\frac{3R}{2}\ln\left(\frac{u(\gamma)}{u_0}\right)
-R\ln\left(\frac{v(\gamma)}{v_0}\right),
\]

where
\[
\frac{\mu_0}{T_0}:=\frac{\mu_S(u_0,v_0)}{T_S(u_0,v_0)}.
\]

Using the molar Euler equation, we have
\begin{align*}
\tilde{s}(u,v)&=\frac{1}{T_S(u,v)}u+\frac{P_S(u,v)}{T_S(u,v)}v-\frac{\mu_S(u,v)}{T_S(u,v)}\\
&=\frac{3R}{2}+R+\frac{3R}{2}\ln\left(\frac{u}{u_0}\right)+R\ln\left(\frac{v}{v_0}\right)-\frac{\mu_0}{T_0}\\
&=s_0+R\ln\left(\left(\frac{u}{u_0}\right)^{3/2}\left(\frac{v}{v_0}\right)\right).
\end{align*}

Thus,
\[
\tilde{S}(U,V,N)=Ns_0+NR\ln\left(\left(\frac{U}{U_0}\right)^{3/2}\left(\frac{V}{V_0}\right)\left(\frac{N}{N_0}\right)^{-5/2}\right),
\]

where
\[
U=uN,\ V=vN,
\]
\[
U_0=u_0N_0,\ V_0=v_0N_0,
\]

and
\[
 s_0=\frac{5}{2}R-\frac{\mu_0}{T_0}. ///
\]
\end{example}

\THERMOEND
