\newif\ifthermosubfile
\ifdefined\THERMONOTESMAIN
  \thermosubfilefalse
\else
  \thermosubfiletrue
\fi

\providecommand{\THERMOEND}{} % safe in both modes

\ifthermosubfile
  \documentclass[11pt]{book}
  \usepackage{cmbright}
\usepackage{mathtools}

\usepackage{pgfplots}
\usepgfplotslibrary{patchplots}
\pgfplotsset{compat=1.15}

\usepackage{color}
\usepackage{xcolor}
\usepackage{subfigure}

\usepackage{amssymb}
\usepackage{amsmath}
\usepackage{amsfonts}
\usepackage{amsthm} 
\usepackage{mdframed}
\usepackage{bbm}
\usepackage{listings}

\usepackage[shortlabels]{enumitem}

\usepackage{tikz}
\usepackage{tikz-3dplot}
\usepackage{xkeyval,tkz-base}

\usetikzlibrary{patterns}
\usetikzlibrary{arrows,arrows.meta,shapes,decorations,automata}
\usetikzlibrary{backgrounds,petri,matrix}
\usetikzlibrary{intersections}
\usetikzlibrary{calc}
\usetikzlibrary{positioning}
\usetikzlibrary{shapes.geometric}
\usetikzlibrary{shapes.arrows}
\usetikzlibrary{matrix}
\usetikzlibrary{positioning}
\usetikzlibrary{3d,calc}

\usepackage{multirow}
\usepackage{array}
\usepackage{algorithm}
\usepackage{algpseudocode}

\usepackage{graphicx}
\usepackage{latexsym}
\usepackage{booktabs}
\usepackage{longtable}

%\usepackage{aligned-overset}

\numberwithin{equation}{section}

\usepackage{lastpage}
\usepackage{fancyhdr}
\pagestyle{fancy}
\setlength{\headheight}{14pt}
\lhead{Mathematical Thermodynamics}
\chead{}
\rhead{Page \thepage}
\cfoot{Page\ \thepage\ of\ \protect\pageref{LastPage}}
\setcounter{section}{-1}

% Theorem environments
\theoremstyle{plain}
\newtheorem{theorem}{Theorem}[section]
\newtheorem{lemma}[theorem]{Lemma}
\newtheorem{proposition}[theorem]{Proposition}
\newtheorem{corollary}[theorem]{Corollary}

\theoremstyle{definition}
\newtheorem{definition}[theorem]{Definition}
\newtheorem{example}[theorem]{Example}
\newtheorem{exercise}[theorem]{Exercise}

\theoremstyle{remark}
\newtheorem{remark}[theorem]{Remark}
\newtheorem{note}[theorem]{Note}

% Hyperref setup
\usepackage{hyperref}
\usepackage{nameref}
\hypersetup{
    hypertexnames=true,
    breaklinks=false,
    colorlinks=true,
    linkcolor=blue,
    citecolor=blue,
    urlcolor=blue
}

\makeatletter
  \newenvironment{dedication}
  {
     \cleardoublepage
     \thispagestyle{empty}
     \vspace*{\stretch{1}}
     \hfill\begin{minipage}[t]{0.66\textwidth}
     \raggedright
  }%
  {
     \end{minipage}
     \vspace*{\stretch{3}}
     \clearpage
  }
  \newenvironment{varsubequations}[1]
  {%
    \addtocounter{equation}{-1}%
    \begin{subequations}
    \renewcommand{\theparentequation}{#1}%
    \def\@currentlabel{#1}%
  }
  {%
    \end{subequations}\ignorespacesafterend
  }
\makeatother

  % ====================================
% THERMODYNAMIC STATE VARIABLES
% ====================================

% Extensive properties
\newcommand{\Int}{U}           % Internal energy
\newcommand{\Ent}{S}           % Entropy
\newcommand{\Vol}{V}           % Volume
\newcommand{\Enth}{H}          % Enthalpy H = U + PV
\newcommand{\Helm}{A}          % Helmholtz free energy A = U - TS
\newcommand{\Gibbs}{G}         % Gibbs free energy G = H - TS
\newcommand{\GrandPot}{\Omega} % Grand potential

% Intensive properties
\newcommand{\Temp}{T}          % Temperature
\newcommand{\Press}{P}         % Pressure
\newcommand{\ChemPot}{\mu}     % Chemical potential
\newcommand{\Mass}{m}          % Mass
\newcommand{\Moles}{n}         % Number of moles

% Specific properties (per unit mass)
\newcommand{\sint}{u}          % Specific internal energy
\newcommand{\sent}{s}          % Specific entropy
\newcommand{\svol}{v}          % Specific volume
\newcommand{\senth}{h}         % Specific enthalpy

% Heat and work
\newcommand{\Heat}{Q}          % Heat transfer
\newcommand{\Work}{W}          % Work
\newcommand{\dQ}{\delta Q}     % Inexact differential of heat
\newcommand{\dW}{\delta W}     % Inexact differential of work

% Thermodynamic processes
\newcommand{\iso}{\text{iso}}
\newcommand{\rev}{\text{rev}}
\newcommand{\irrev}{\text{irrev}}

% ====================================
% MATHEMATICAL OPERATORS
% ====================================

\DeclareMathOperator{\grad}{grad}
\DeclareMathOperator{\dive}{div}
\DeclareMathOperator{\curl}{curl}
\DeclareMathOperator{\Span}{span}
\DeclareMathOperator{\rank}{rank}
\DeclareMathOperator{\trace}{tr}
\DeclareMathOperator{\sgn}{sgn}
\DeclareMathOperator{\diam}{diam}

% Partial derivatives
\newcommand{\pder}[2]{\frac{\partial #1}{\partial #2}}
\newcommand{\pdertwo}[2]{\frac{\partial^2 #1}{\partial #2^2}}
\newcommand{\pdermix}[3]{\frac{\partial^2 #1}{\partial #2 \partial #3}}

% Thermodynamic partial derivatives
\newcommand{\pderT}[3]{\left(\frac{\partial #1}{\partial #2}\right)_{#3}}

% Total derivatives
\newcommand{\dder}[2]{\frac{d #1}{d #2}}
\newcommand{\ddertwo}[2]{\frac{d^2 #1}{d #2^2}}

% Differential operators
\newcommand{\dd}{\mathrm{d}}
\newcommand{\dif}{\mathrm{d}}

% ====================================
% VECTOR AND MATRIX NOTATION
% ====================================

% Bold vectors
\newcommand{\bfv}[1]{\boldsymbol{#1}}
\newcommand{\bfx}{\boldsymbol{x}}
\newcommand{\bfy}{\boldsymbol{y}}
\newcommand{\bfz}{\boldsymbol{z}}
\newcommand{\bfu}{\boldsymbol{u}}
\newcommand{\bfv}{\boldsymbol{v}}
\newcommand{\bfw}{\boldsymbol{w}}
\newcommand{\bff}{\boldsymbol{f}}
\newcommand{\bfg}{\boldsymbol{g}}
\newcommand{\bfF}{\boldsymbol{F}}
\newcommand{\bfn}{\boldsymbol{n}}
\newcommand{\bfe}{\boldsymbol{e}}
\newcommand{\bfzero}{\boldsymbol{0}}

% Matrix spaces
\newcommand{\Rn}{\mathbb{R}^n}
\newcommand{\Rm}{\mathbb{R}^m}
\newcommand{\Rmn}{\mathbb{R}^{m \times n}}

% ====================================
% SETS AND SPACES
% ====================================

\newcommand{\R}{\mathbb{R}}    % Real numbers
\newcommand{\C}{\mathbb{C}}    % Complex numbers
\newcommand{\N}{\mathbb{N}}    % Natural numbers
\newcommand{\Z}{\mathbb{Z}}    % Integers
\newcommand{\Q}{\mathbb{Q}}    % Rationals

% Function spaces
\newcommand{\Lp}[1]{L^{#1}}
\newcommand{\Hone}{H^1}
\newcommand{\Ck}[1]{C^{#1}}

% ====================================
% SPECIAL SYMBOLS
% ====================================

% Indicator/characteristic function
\newcommand{\ind}{\mathbbm{1}}

% Inner product
\newcommand{\inner}[2]{\langle #1, #2 \rangle}

% Norm
\newcommand{\norm}[1]{\left\| #1 \right\|}
\newcommand{\abs}[1]{\left| #1 \right|}

% Set notation
\newcommand{\set}[1]{\left\{ #1 \right\}}
\newcommand{\setdef}[2]{\left\{ #1 \,:\, #2 \right\}}

% Ceiling and floor
\newcommand{\ceil}[1]{\left\lceil #1 \right\rceil}
\newcommand{\floor}[1]{\left\lfloor #1 \right\rfloor}

% ====================================
% MATHCAL AND MATHSF LETTERS
% ====================================

% Calligraphic letters for sets and spaces
\newcommand{\calA}{\mathcal{A}}
\newcommand{\calB}{\mathcal{B}}
\newcommand{\calC}{\mathcal{C}}
\newcommand{\calD}{\mathcal{D}}
\newcommand{\calE}{\mathcal{E}}
\newcommand{\calF}{\mathcal{F}}
\newcommand{\calG}{\mathcal{G}}
\newcommand{\calH}{\mathcal{H}}
\newcommand{\calI}{\mathcal{I}}
\newcommand{\calJ}{\mathcal{J}}
\newcommand{\calK}{\mathcal{K}}
\newcommand{\calL}{\mathcal{L}}
\newcommand{\calM}{\mathcal{M}}
\newcommand{\calN}{\mathcal{N}}
\newcommand{\calO}{\mathcal{O}}
\newcommand{\calP}{\mathcal{P}}
\newcommand{\calQ}{\mathcal{Q}}
\newcommand{\calR}{\mathcal{R}}
\newcommand{\calS}{\mathcal{S}}
\newcommand{\calT}{\mathcal{T}}
\newcommand{\calU}{\mathcal{U}}
\newcommand{\calV}{\mathcal{V}}
\newcommand{\calW}{\mathcal{W}}
\newcommand{\calX}{\mathcal{X}}
\newcommand{\calY}{\mathcal{Y}}
\newcommand{\calZ}{\mathcal{Z}}

% Sans-serif letters for matrices and operators
\newcommand{\msfA}{\mathsf{A}}
\newcommand{\msfB}{\mathsf{B}}
\newcommand{\msfC}{\mathsf{C}}
\newcommand{\msfD}{\mathsf{D}}
\newcommand{\msfE}{\mathsf{E}}
\newcommand{\msfF}{\mathsf{F}}
\newcommand{\msfG}{\mathsf{G}}
\newcommand{\msfH}{\mathsf{H}}
\newcommand{\msfI}{\mathsf{I}}
\newcommand{\msfJ}{\mathsf{J}}
\newcommand{\msfK}{\mathsf{K}}
\newcommand{\msfL}{\mathsf{L}}
\newcommand{\msfM}{\mathsf{M}}
\newcommand{\msfN}{\mathsf{N}}
\newcommand{\msfO}{\mathsf{O}}
\newcommand{\msfP}{\mathsf{P}}
\newcommand{\msfQ}{\mathsf{Q}}
\newcommand{\msfR}{\mathsf{R}}
\newcommand{\msfS}{\mathsf{S}}
\newcommand{\msfT}{\mathsf{T}}
\newcommand{\msfU}{\mathsf{U}}
\newcommand{\msfV}{\mathsf{V}}
\newcommand{\msfW}{\mathsf{W}}
\newcommand{\msfX}{\mathsf{X}}
\newcommand{\msfY}{\mathsf{Y}}
\newcommand{\msfZ}{\mathsf{Z}}

% ====================================
% COMMON ABBREVIATIONS
% ====================================

\newcommand{\ie}{i.e.\ }
\newcommand{\eg}{e.g.\ }
\newcommand{\cf}{cf.\ }
\newcommand{\etc}{etc.\ }
\newcommand{\vs}{vs.\ }
\newcommand{\resp}{resp.\ }
\newcommand{\etal}{et al.\ }

  \graphicspath{{./Figures/}}
  \begin{document}
  \renewcommand{\THERMOEND}{\end{document}}
\fi


\chapter{ThermoS26-03}
\label{chap:ThermoS26-03}

\section{Heat Flow}

We examine the approach to thermal equilibrium using a diathermal wall. Suppose the substance is unary, $r=1$. It must be that the variables
\[
V^\alpha,\ V^\beta,\ N^\alpha,\ N^\beta
\]
are fixed, but energy can be exchanged in the process. Suppose the variable $\gamma$ parameterizes the process.

\begin{figure}[h]
\centering
\includegraphics[width=\textwidth]{Figures/ThermoS26-03-fig-1.pdf}
\caption{Initial and equilibrium configurations for $\gamma=0$ and $\gamma=1$. At $\gamma=0$ the insulating wall is replaced with a diathermal wall; left hot subsystem has $U^\alpha(0)$, $V^\alpha$, $N^\alpha$ and right cold subsystem has $U^\beta(0)$, $V^\beta$, $N^\beta$. At $\gamma=1$ a diathermal wall separates subsystems with $U^\alpha(1)$, $V^\alpha$, $N^\alpha$ and $U^\beta(1)$, $V^\beta$, $N^\beta$ in equilibrium.}
\label{fig:heat-flow-process}
\end{figure}

The initial and equilibrium states satisfy
\begin{align*}
S(0)&=\tilde{S}^\alpha\!\big(U^\alpha(0),V^\alpha,N^\alpha\big)+\tilde{S}^\beta\!\big(U^\beta(0),V^\beta,N^\beta\big)\\
&\le
\tilde{S}^\alpha\!\big(U^\alpha(1),V^\alpha,N^\alpha\big)+\tilde{S}^\beta\!\big(U^\beta(1),V^\beta,N^\beta\big)
=S(1).
\end{align*}



We require that
\[
U^\alpha(\gamma)+U^\beta(\gamma)=U_0.
\]
The entropy is at a maximum at equilibrium. Suppose that, as indicated by Figure~\ref{fig:heat-flow-process},
\[
\tilde{T}^\alpha\big(U^\alpha(0),V^\alpha,N^\alpha\big)>\tilde{T}^\beta\big(U^\beta(0),V^\beta,N^\beta\big)
\]
while at equilibrium
\[
\tilde{T}^\alpha\big(U^\alpha(1),V^\alpha,N^\alpha\big)=\tilde{T}^\beta\big(U^\beta(1),V^\beta,N^\beta\big).
\]

We will show that
\[
U^\alpha(0)>U^\alpha(1)\qquad (\text{energy is lost in }\alpha)
\]
and
\[
U^\beta(0)<U^\beta(1)\qquad (\text{energy is gain in }\beta).
\]
To see this, compute
\begin{align}
\frac{dS}{d\gamma}
&=\frac{\partial \tilde{S}^\alpha}{\partial U^\alpha}\frac{\partial U^\alpha}{\partial \gamma}
+\frac{\partial \tilde{S}^\beta}{\partial U^\beta}\frac{\partial U^\beta}{\partial \gamma} \notag \\
&=\frac{1}{\tilde{T}_S^\alpha(\gamma)}\frac{\partial U^\alpha}{\partial \gamma}
+\frac{1}{\tilde{T}_S^\beta(\gamma)}\frac{\partial U^\beta}{\partial \gamma} \notag \\
&=\left(\frac{1}{\tilde{T}_S^\alpha(\gamma)}-\frac{1}{\tilde{T}_S^\beta(\gamma)}\right)\frac{\partial U^\alpha}{\partial \gamma},
\tag{3.1}\label{eq:heat-ds-gamma}
\end{align}
where
\[
\tilde{T}_S^q(\gamma):=T_S^q\big(U^q(\gamma)\big),\qquad q=\alpha,\beta.
\]

We know that
\[
S(0)\le \tilde{S}(1).
\]
We also know that
\[
\tilde{T}_S^\alpha(0)>\tilde{T}_S^\beta(0)
\qquad\text{and}\qquad
\tilde{T}_S^\alpha(1)=\tilde{T}_S^\beta(1).
\]
Set
\[
R(\gamma):=\frac{1}{\tilde{T}^\alpha(\gamma)}-\frac{1}{\tilde{T}^\beta(\gamma)},\qquad 0\le\gamma\le 1.
\]
We know that $R(0)<0$ and $R(1)=0$.

\begin{figure}[h]
\centering
\includegraphics[width=0.75\textwidth]{Figures/ThermoS26-03-fig-2.pdf}
\caption{Plot of $R(\gamma)$ on $[0,1]$. The red curve is impossible; if it occurred, equilibrium would be reached earlier at $\gamma=\gamma^\ast$. The valid curve remains negative with $R(1)=0$.}
\label{fig:r-plot}
\end{figure}

The red curve in Figure~\ref{fig:r-plot} is not possible; if it were, equilibrium would be reached earlier at $\gamma=\gamma^\ast$. Thus
\[
R(\gamma)<0\qquad \forall\ \gamma\in[0,1).
\]
Integrating \eqref{eq:heat-ds-gamma}, we get
\begin{equation}
0\le \tilde{S}(1)-\tilde{S}(0)=\int_0^1 R(\gamma)\frac{dU^\alpha}{d\gamma}\,d\gamma.
\tag{3.2}\label{eq:heat-entropy-diff}
\end{equation}

For $U^\alpha(\gamma)$ we have two options:
\[
\text{Case (1)}\quad U^\alpha(0)>U^\alpha(1),
\qquad
\text{Case (2)}\quad U^\alpha(0)\le U^\alpha(1).
\]
We are free to pick a parameterization however we want. Let us take a simple linear path
\[
\frac{dU^\alpha}{d\gamma}=U^\alpha(1)-U^\alpha(0)=:C^\alpha.
\]
Then Case (1) implies $C^\alpha<0$, while Case (2) implies $C^\alpha\ge 0$. The integral in \eqref{eq:heat-entropy-diff} is
\[
0\le \tilde{S}(1)-\tilde{S}(0)=C^\alpha\int_0^1 R(\gamma)\,d\gamma.
\]
Therefore the only possible choice is Case (1). Thus
\[
U^\alpha(0)>U^\alpha(1)\qquad \big(U^\beta(0)<U^\beta(1)\big).
\]
This is consistent with our intuition about temperature. If $\tilde{T}^\alpha(0)>\tilde{T}^\beta(0)$ heat energy flows from subsystem $\alpha$ to subsystem $\beta$.

\begin{figure}[h]
\centering
\includegraphics[width=\textwidth]{Figures/ThermoS26-03-fig-3}
\caption{Energy transfer for $\gamma=0$ (hot to cold through replaced wall) and $\gamma=1$ (equilibrium with diathermal wall) showing flow from subsystem $\alpha$ to subsystem $\beta$.}
\label{fig:heat-flow-direction}
\end{figure}

Heat Transfer Principle: Heat energy always flows from hotter to colder system. This is equivalent to the second law of thermodynamics.

\section{The Euler Equation}

Recall that internal energy and the entropy are homogenous of degree one:
\begin{equation}
\tilde{U}(\lambda S,\lambda V,\lambda\vec{N})=\lambda\tilde{U}(S,V,\vec{N}),\qquad \lambda>0,
\tag{3.3}\label{eq:euler-homogeneous}
\end{equation}
where
\[
\vec{N}=
\begin{bmatrix}
N_1\\
N_2\\
\vdots\\
N_r
\end{bmatrix}.
\]

\begin{theorem}
\label{thm:euler-internal-energy}
\textup{(3.1)} Let $\tilde{U}$ be the internal energy of an isolated system. Then
\begin{equation}
\tilde{U}=T S-P V+\mu_1 N_1+\cdots+\mu_r N_r.
\tag{3.4}\label{eq:euler-equation}
\end{equation}
\end{theorem}

\begin{proof}
Since $\tilde{U}$ is homogenous of degree one, differentiating \eqref{eq:euler-homogeneous} with respect to $\lambda$ we obtain
\[
T(\lambda S,\lambda V,\lambda\vec{N})\,S-P(\lambda S,\lambda V,\lambda\vec{N})\,V
+\sum_{i=1}^r\mu_i(\lambda S,\lambda V,\lambda\vec{N})\,N_i=\tilde{U}(S,V,\vec{N}).
\]
Taking $\lambda=1$ gives \eqref{eq:euler-equation}, as desired. ///
\end{proof}

Remark: Equation \eqref{eq:euler-equation} is known as Euler's Equation.

\begin{definition}
\label{def:process-path}
\textup{(3.2)} A process path in state space $\Sigma_S$,
\[
\Sigma_S\subseteq [0,\infty)\times[0,\infty)\times[0,\infty)\times\cdots\times[0,\infty),
\]
is a continuous, piecewise differentiable function $\vec{Y}:[0,1]\to\Sigma_S$, defined by
\[
\vec{Y}(\gamma)=
\begin{bmatrix}
S(\gamma)\\
V(\gamma)\\
N_1(\gamma)\\
\vdots\\
N_r(\gamma)
\end{bmatrix}.
\]
A process path in $\Sigma_U$ is defined similarly.
\end{definition}

Thus, using the chain rule, we have
\begin{align}
\frac{d}{d\gamma}\tilde{U}\big(\vec{Y}(\gamma)\big)
&=T\big(\vec{Y}(\gamma)\big)S'(\gamma)-P\big(\vec{Y}(\gamma)\big)V'(\gamma)
+\sum_{i=1}^r\mu_i\big(\vec{Y}(\gamma)\big)N_i'(\gamma)
\tag{3.5}\label{eq:chain-rule-u}
\end{align}
for a valid process path in state space.

\begin{theorem}
\label{thm:gibbs-duhem}
\textup{(3.3)} Suppose that $\vec{Y}:[0,1]\to\Sigma_S$ is a process path in state space $\Sigma_S$. Then
\begin{equation}
0=S(\gamma)\frac{dT_S(\vec{Y}(\gamma))}{d\gamma}-V(\gamma)\frac{dP_S(\vec{Y}(\gamma))}{d\gamma}
+\sum_{i=1}^r N_i(\gamma)\frac{d\mu_S^i(\vec{Y}(\gamma))}{d\gamma}.
\tag{3.6}\label{eq:gibbs-duhem}
\end{equation}
This equation is called the Gibbs-Duhem relation.
\end{theorem}

\begin{proof}
Begin with the Euler equation and differentiate with respect to the process parameter $\gamma$:
\begin{align}
\frac{d\tilde{U}}{d\gamma}\big(\vec{Y}(\gamma)\big)
&=\frac{dT}{d\gamma}\big(\vec{Y}(\gamma)\big)S(\gamma)+T\big(\vec{Y}(\gamma)\big)S'(\gamma) \notag \\
&\quad -\frac{dP}{d\gamma}\big(\vec{Y}(\gamma)\big)V(\gamma)-P\big(\vec{Y}(\gamma)\big)V'(\gamma) \notag \\
&\quad +\sum_{i=1}^r\left\{\frac{d\mu_i}{d\gamma}\big(\vec{Y}(\gamma)\big)N_i(\gamma)
+\mu_i\big(\vec{Y}(\gamma)\big)N_i'(\gamma)\right\}.
\tag{3.7}\label{eq:euler-derivative}
\end{align}
Substituting \eqref{eq:chain-rule-u} into \eqref{eq:euler-derivative} yields \eqref{eq:gibbs-duhem}. ///
\end{proof}

\section{Examples}

\begin{example}
\label{ex:fundamental-relation-homogeneous}
\textup{(3.4)} Suppose that the fundamental relation for a material is given by
\[
\tilde{S}=4 A\, U^{1/4} V^{1/2} N^{1/4}+B N,\qquad \Sigma_U=[0,\infty)^3,
\]
where $A,B>0$ are constants. This function must be homogenous of degree one. Suppose $\lambda>0$. Then
\[
\tilde{S}(\lambda U,\lambda V,\lambda N)=4A\,\lambda U^{1/4}\lambda V^{1/2}\lambda N^{1/4}+B\lambda N
=\lambda\tilde{S}(U,V,N). ///
\]

Recall that
\[
T=\frac{1}{\frac{\partial \tilde{S}}{\partial U}}
=\Big(A\,U^{-3/4}V^{1/2}N^{1/4}\Big)^{-1}
=\frac{U^{3/4}}{A V^{1/2}N^{1/4}},
\]
which is homogenous degree zero.
\[
\frac{P}{T}=\frac{\partial \tilde{S}}{\partial V}
\]
so
\[
P=T\,\frac{\partial \tilde{S}}{\partial V}
=\frac{U^{3/4}}{A V^{1/2} N^{1/4}}\left(\frac{2A\,U^{1/4}N^{1/4}}{V^{1/2}}\right)
=\frac{2U}{V}.
\]
\end{example}

Finally,
\[
\mu=-T\,\frac{\partial \tilde{S}}{\partial N}
=-\frac{U^{3/4}}{A V^{1/2}N^{1/4}}\left(\frac{A\,U^{1/4}V^{1/2}}{N^{3/4}}+B\right)
=-\frac{U}{N}-\frac{B U^{3/4}}{A V^{1/2}N^{1/4}}. ///
\]

\begin{example}
\label{ex:fundamental-relation-u}
\textup{(3.5)} Suppose that the fundamental relation is
\[
\tilde{U}=\left(\frac{S-BN}{4A V^{1/2}N^{1/4}}\right)^4.
\]
Recall that
\[
T=\frac{\partial \tilde{U}}{\partial S}
=4\left(\frac{S-BN}{4A V^{1/2}N^{1/4}}\right)^3\frac{1}{4A V^{1/2}N^{1/4}}
=\frac{\tilde{U}^{3/4}}{A V^{1/2}N^{1/4}},
\]
the same as above.
\[
P=-\frac{\partial \tilde{U}}{\partial V}
=-4\left(\frac{S-BN}{4A V^{1/2}N^{1/4}}\right)^3\left(\frac{S-BN}{4A N^{1/4}}\right)\left(-\frac{1}{2}\right)\frac{1}{V^{3/2}}
\]
so
\[
=2\left(\frac{S-BN}{4A V^{1/2}N^{1/4}}\right)^4\frac{1}{V}
=\frac{2\tilde{U}}{V}.
\]
Finally,
\[
\mu=\frac{\partial \tilde{U}}{\partial N}
=4\left(\frac{S-BN}{4A V^{1/2}N^{1/4}}\right)^3\frac{1}{4A V^{1/2}}\frac{N^{1/4}(-B)-(S-BN)\frac{1}{4}N^{-3/4}}{N^{1/2}}
\]
\[
=4\left(\frac{S-BN}{4A V^{1/2}N^{1/4}}\right)^3\frac{1}{4A V^{1/2}}\frac{N^{1/4}(-B)-(S-BN)\frac{1}{4}N^{-3/4}}{N^{1/2}}
\]
\[
=4\tilde{U}^{3/4}\frac{N^{1/4}}{4A V^{1/2}N^{1/2}}\cdot\left(-B-\frac{S-BN}{4N}\right)
\]
\[
=-\frac{B\tilde{U}^{3/4}}{A V^{1/2}N^{1/4}}-\frac{\tilde{U}^{3/4}}{A V^{1/2}N^{1/4}}\frac{(S-BN)}{4N}
=-\frac{B\tilde{U}^{3/4}}{A V^{1/2}N^{1/4}}-\frac{\tilde{U}}{N}. ///
\]
\end{example}

\section{Euler Equation with Respect to Entropy}

\begin{theorem}
\label{thm:euler-entropy}
\textup{(3.5)} Let $\tilde{S}$ be the internal energy of an isolated system. Then
\begin{equation}
\tilde{S}=\frac{1}{T_S}U+\frac{P_S}{T_S}V-\sum_{i=1}^r\frac{\mu_{S,i}}{T_S}N_i,
\tag{3.8}\label{eq:euler-entropy}
\end{equation}
where
\[
T_S=T_S(U,V,\vec{N}),\qquad
P_S=P_S(U,V,\vec{N}),\qquad
\mu_{S,i}=\mu_{S,i}(U,V,\vec{N}),
\]
and
\[
\frac{1}{T_S}=\frac{\partial \tilde{S}}{\partial U},\qquad \frac{P_S}{T_S}=\frac{\partial \tilde{S}}{\partial V},\qquad \frac{\mu_{S,i}}{T_S}=\frac{\partial \tilde{S}}{\partial N_i}.
\]
\end{theorem}

\begin{proof}
We again use the fact that $\tilde{S}$ is homogenous of degree one. For any $\lambda>0$,
\[
\tilde{S}(\lambda U,\lambda V,\lambda\vec{N})=\lambda\tilde{S}(U,V,\vec{N}).
\]
Taking the derivative with respect to $\lambda$, we have
\[
\frac{U}{T_S(\lambda U,\lambda V,\lambda\vec{N})}
+\frac{P_S(\lambda U,\lambda V,\lambda\vec{N})}{T_S(\lambda U,\lambda V,\lambda\vec{N})}V
+\sum_{j=1}^r\frac{\mu_{S,j}(\lambda U,\lambda V,\lambda\vec{N})}{T_S(\lambda U,\lambda V,\lambda\vec{N})}N_j
=\tilde{S}(U,V,\vec{N}),
\]
and setting $\lambda=1$ gives the desired result. ///
\end{proof}

\section{Gibbs-Duhem Relation in the Entropy Form}

The Gibbs-Duhem equation is similarly derived.

\begin{theorem}
\label{thm:gibbs-duhem-entropy}
\textup{(3.6)} Suppose that $\vec{Y}:[0,1]\to\Sigma_U$ is a process path. Then
\begin{equation}
0=\frac{dT_S}{d\gamma}\big(\vec{Y}(\gamma)\big)\,\tilde{S}\big(\vec{Y}(\gamma)\big)
-\frac{dP_S}{d\gamma}\big(\vec{Y}(\gamma)\big)V(\gamma)
+\sum_{j=1}^r\frac{d\mu_S^j}{d\gamma}\big(\vec{Y}(\gamma)\big)N_j(\gamma).
\tag{3.9}\label{eq:gibbs-duhem-entropy}
\end{equation}
This equation is called the Gibbs-Duhem relation in the entropy form.
\end{theorem}

\begin{proof}
Using \eqref{eq:euler-entropy}, we have
\[
T_S\big(\vec{Y}(\gamma)\big)\,\tilde{S}\big(\vec{Y}(\gamma)\big)
=U(\gamma)+P_S\big(\vec{Y}(\gamma)\big)V(\gamma)
-\sum_{j=1}^r \mu_{S,j}\big(\vec{Y}(\gamma)\big)N_j(\gamma).
\]
Taking the $\gamma$-derivative of the last equation, we have
\begin{align}
\frac{dT_S}{d\gamma}\big(\vec{Y}(\gamma)\big)\,\tilde{S}\big(\vec{Y}(\gamma)\big)
+T_S\big(\vec{Y}(\gamma)\big)\frac{d\tilde{S}}{d\gamma}\big(\vec{Y}(\gamma)\big)
&=U'(\gamma)+\frac{dP_S}{d\gamma}\big(\vec{Y}(\gamma)\big)V(\gamma)+P_S\big(\vec{Y}(\gamma)\big)V'(\gamma) \notag \\
&\quad -\sum_{j=1}^r\left\{\frac{d\mu_{S,j}}{d\gamma}\big(\vec{Y}(\gamma)\big)N_j(\gamma)+\mu_{S,j}\big(\vec{Y}(\gamma)\big)N_j'(\gamma)\right\}.
\tag{3.10}\label{eq:gibbs-duhem-entropy-derivative}
\end{align}
Taking the $\gamma$-derivative of $\tilde{S}\big(\vec{Y}(\gamma)\big)$ we have
\begin{equation}
\frac{d\tilde{S}}{d\gamma}\big(\vec{Y}(\gamma)\big)
=\frac{1}{T_S(\vec{Y}(\gamma))}U'(\gamma)+\frac{P_S(\vec{Y}(\gamma))}{T_S(\vec{Y}(\gamma))}V'(\gamma)
-\sum_{j=1}^r\frac{\mu_{S,j}(\vec{Y}(\gamma))}{T_S(\vec{Y}(\gamma))}N_j'(\gamma).
\tag{3.11}\label{eq:entropy-derivative}
\end{equation}
Substituting \eqref{eq:entropy-derivative} into \eqref{eq:gibbs-duhem-entropy-derivative} yields \eqref{eq:gibbs-duhem-entropy}. ///
\end{proof}

Remark: Compare \eqref{eq:gibbs-duhem} and \eqref{eq:gibbs-duhem-entropy}:
\[
0=S(\gamma)\frac{dT_S(\vec{Y}(\gamma))}{d\gamma}-V(\gamma)\frac{dP_S(\vec{Y}(\gamma))}{d\gamma}
+\sum_{i=1}^r N_i(\gamma)\frac{d\mu_S^i(\vec{Y}(\gamma))}{d\gamma},
\]
\[
0=\frac{dT_S}{d\gamma}\big(\vec{Y}(\gamma)\big)\,\tilde{S}\big(\vec{Y}(\gamma)\big)
-\frac{dP_S}{d\gamma}\big(\vec{Y}(\gamma)\big)V(\gamma)
+\sum_{j=1}^r\frac{d\mu_S^j}{d\gamma}\big(\vec{Y}(\gamma)\big)N_j(\gamma),
\]
These are essentially the same expression!

\THERMOEND
