\newif\ifthermosubfile
\ifdefined\THERMONOTESMAIN
  \thermosubfilefalse
\else
  \thermosubfiletrue
\fi

\providecommand{\THERMOEND}{} % safe in both modes

\ifthermosubfile
  \documentclass[11pt]{book}
  \usepackage{cmbright}
\usepackage{mathtools}

\usepackage{pgfplots}
\usepgfplotslibrary{patchplots}
\pgfplotsset{compat=1.15}

\usepackage{color}
\usepackage{xcolor}
\usepackage{subfigure}

\usepackage{amssymb}
\usepackage{amsmath}
\usepackage{amsfonts}
\usepackage{amsthm} 
\usepackage{mdframed}
\usepackage{bbm}
\usepackage{listings}

\usepackage[shortlabels]{enumitem}

\usepackage{tikz}
\usepackage{tikz-3dplot}
\usepackage{xkeyval,tkz-base}

\usetikzlibrary{patterns}
\usetikzlibrary{arrows,arrows.meta,shapes,decorations,automata}
\usetikzlibrary{backgrounds,petri,matrix}
\usetikzlibrary{intersections}
\usetikzlibrary{calc}
\usetikzlibrary{positioning}
\usetikzlibrary{shapes.geometric}
\usetikzlibrary{shapes.arrows}
\usetikzlibrary{matrix}
\usetikzlibrary{positioning}
\usetikzlibrary{3d,calc}

\usepackage{multirow}
\usepackage{array}
\usepackage{algorithm}
\usepackage{algpseudocode}

\usepackage{graphicx}
\usepackage{latexsym}
\usepackage{booktabs}
\usepackage{longtable}

%\usepackage{aligned-overset}

\numberwithin{equation}{section}

\usepackage{lastpage}
\usepackage{fancyhdr}
\pagestyle{fancy}
\setlength{\headheight}{14pt}
\lhead{Mathematical Thermodynamics}
\chead{}
\rhead{Page \thepage}
\cfoot{Page\ \thepage\ of\ \protect\pageref{LastPage}}
\setcounter{section}{-1}

% Theorem environments
\theoremstyle{plain}
\newtheorem{theorem}{Theorem}[section]
\newtheorem{lemma}[theorem]{Lemma}
\newtheorem{proposition}[theorem]{Proposition}
\newtheorem{corollary}[theorem]{Corollary}

\theoremstyle{definition}
\newtheorem{definition}[theorem]{Definition}
\newtheorem{example}[theorem]{Example}
\newtheorem{exercise}[theorem]{Exercise}

\theoremstyle{remark}
\newtheorem{remark}[theorem]{Remark}
\newtheorem{note}[theorem]{Note}

% Hyperref setup
\usepackage{hyperref}
\usepackage{nameref}
\hypersetup{
    hypertexnames=true,
    breaklinks=false,
    colorlinks=true,
    linkcolor=blue,
    citecolor=blue,
    urlcolor=blue
}

\makeatletter
  \newenvironment{dedication}
  {
     \cleardoublepage
     \thispagestyle{empty}
     \vspace*{\stretch{1}}
     \hfill\begin{minipage}[t]{0.66\textwidth}
     \raggedright
  }%
  {
     \end{minipage}
     \vspace*{\stretch{3}}
     \clearpage
  }
  \newenvironment{varsubequations}[1]
  {%
    \addtocounter{equation}{-1}%
    \begin{subequations}
    \renewcommand{\theparentequation}{#1}%
    \def\@currentlabel{#1}%
  }
  {%
    \end{subequations}\ignorespacesafterend
  }
\makeatother

  % ====================================
% THERMODYNAMIC STATE VARIABLES
% ====================================

% Extensive properties
\newcommand{\Int}{U}           % Internal energy
\newcommand{\Ent}{S}           % Entropy
\newcommand{\Vol}{V}           % Volume
\newcommand{\Enth}{H}          % Enthalpy H = U + PV
\newcommand{\Helm}{A}          % Helmholtz free energy A = U - TS
\newcommand{\Gibbs}{G}         % Gibbs free energy G = H - TS
\newcommand{\GrandPot}{\Omega} % Grand potential

% Intensive properties
\newcommand{\Temp}{T}          % Temperature
\newcommand{\Press}{P}         % Pressure
\newcommand{\ChemPot}{\mu}     % Chemical potential
\newcommand{\Mass}{m}          % Mass
\newcommand{\Moles}{n}         % Number of moles

% Specific properties (per unit mass)
\newcommand{\sint}{u}          % Specific internal energy
\newcommand{\sent}{s}          % Specific entropy
\newcommand{\svol}{v}          % Specific volume
\newcommand{\senth}{h}         % Specific enthalpy

% Heat and work
\newcommand{\Heat}{Q}          % Heat transfer
\newcommand{\Work}{W}          % Work
\newcommand{\dQ}{\delta Q}     % Inexact differential of heat
\newcommand{\dW}{\delta W}     % Inexact differential of work

% Thermodynamic processes
\newcommand{\iso}{\text{iso}}
\newcommand{\rev}{\text{rev}}
\newcommand{\irrev}{\text{irrev}}

% ====================================
% MATHEMATICAL OPERATORS
% ====================================

\DeclareMathOperator{\grad}{grad}
\DeclareMathOperator{\dive}{div}
\DeclareMathOperator{\curl}{curl}
\DeclareMathOperator{\Span}{span}
\DeclareMathOperator{\rank}{rank}
\DeclareMathOperator{\trace}{tr}
\DeclareMathOperator{\sgn}{sgn}
\DeclareMathOperator{\diam}{diam}

% Partial derivatives
\newcommand{\pder}[2]{\frac{\partial #1}{\partial #2}}
\newcommand{\pdertwo}[2]{\frac{\partial^2 #1}{\partial #2^2}}
\newcommand{\pdermix}[3]{\frac{\partial^2 #1}{\partial #2 \partial #3}}

% Thermodynamic partial derivatives
\newcommand{\pderT}[3]{\left(\frac{\partial #1}{\partial #2}\right)_{#3}}

% Total derivatives
\newcommand{\dder}[2]{\frac{d #1}{d #2}}
\newcommand{\ddertwo}[2]{\frac{d^2 #1}{d #2^2}}

% Differential operators
\newcommand{\dd}{\mathrm{d}}
\newcommand{\dif}{\mathrm{d}}

% ====================================
% VECTOR AND MATRIX NOTATION
% ====================================

% Bold vectors
\newcommand{\bfv}[1]{\boldsymbol{#1}}
\newcommand{\bfx}{\boldsymbol{x}}
\newcommand{\bfy}{\boldsymbol{y}}
\newcommand{\bfz}{\boldsymbol{z}}
\newcommand{\bfu}{\boldsymbol{u}}
\newcommand{\bfv}{\boldsymbol{v}}
\newcommand{\bfw}{\boldsymbol{w}}
\newcommand{\bff}{\boldsymbol{f}}
\newcommand{\bfg}{\boldsymbol{g}}
\newcommand{\bfF}{\boldsymbol{F}}
\newcommand{\bfn}{\boldsymbol{n}}
\newcommand{\bfe}{\boldsymbol{e}}
\newcommand{\bfzero}{\boldsymbol{0}}

% Matrix spaces
\newcommand{\Rn}{\mathbb{R}^n}
\newcommand{\Rm}{\mathbb{R}^m}
\newcommand{\Rmn}{\mathbb{R}^{m \times n}}

% ====================================
% SETS AND SPACES
% ====================================

\newcommand{\R}{\mathbb{R}}    % Real numbers
\newcommand{\C}{\mathbb{C}}    % Complex numbers
\newcommand{\N}{\mathbb{N}}    % Natural numbers
\newcommand{\Z}{\mathbb{Z}}    % Integers
\newcommand{\Q}{\mathbb{Q}}    % Rationals

% Function spaces
\newcommand{\Lp}[1]{L^{#1}}
\newcommand{\Hone}{H^1}
\newcommand{\Ck}[1]{C^{#1}}

% ====================================
% SPECIAL SYMBOLS
% ====================================

% Indicator/characteristic function
\newcommand{\ind}{\mathbbm{1}}

% Inner product
\newcommand{\inner}[2]{\langle #1, #2 \rangle}

% Norm
\newcommand{\norm}[1]{\left\| #1 \right\|}
\newcommand{\abs}[1]{\left| #1 \right|}

% Set notation
\newcommand{\set}[1]{\left\{ #1 \right\}}
\newcommand{\setdef}[2]{\left\{ #1 \,:\, #2 \right\}}

% Ceiling and floor
\newcommand{\ceil}[1]{\left\lceil #1 \right\rceil}
\newcommand{\floor}[1]{\left\lfloor #1 \right\rfloor}

% ====================================
% MATHCAL AND MATHSF LETTERS
% ====================================

% Calligraphic letters for sets and spaces
\newcommand{\calA}{\mathcal{A}}
\newcommand{\calB}{\mathcal{B}}
\newcommand{\calC}{\mathcal{C}}
\newcommand{\calD}{\mathcal{D}}
\newcommand{\calE}{\mathcal{E}}
\newcommand{\calF}{\mathcal{F}}
\newcommand{\calG}{\mathcal{G}}
\newcommand{\calH}{\mathcal{H}}
\newcommand{\calI}{\mathcal{I}}
\newcommand{\calJ}{\mathcal{J}}
\newcommand{\calK}{\mathcal{K}}
\newcommand{\calL}{\mathcal{L}}
\newcommand{\calM}{\mathcal{M}}
\newcommand{\calN}{\mathcal{N}}
\newcommand{\calO}{\mathcal{O}}
\newcommand{\calP}{\mathcal{P}}
\newcommand{\calQ}{\mathcal{Q}}
\newcommand{\calR}{\mathcal{R}}
\newcommand{\calS}{\mathcal{S}}
\newcommand{\calT}{\mathcal{T}}
\newcommand{\calU}{\mathcal{U}}
\newcommand{\calV}{\mathcal{V}}
\newcommand{\calW}{\mathcal{W}}
\newcommand{\calX}{\mathcal{X}}
\newcommand{\calY}{\mathcal{Y}}
\newcommand{\calZ}{\mathcal{Z}}

% Sans-serif letters for matrices and operators
\newcommand{\msfA}{\mathsf{A}}
\newcommand{\msfB}{\mathsf{B}}
\newcommand{\msfC}{\mathsf{C}}
\newcommand{\msfD}{\mathsf{D}}
\newcommand{\msfE}{\mathsf{E}}
\newcommand{\msfF}{\mathsf{F}}
\newcommand{\msfG}{\mathsf{G}}
\newcommand{\msfH}{\mathsf{H}}
\newcommand{\msfI}{\mathsf{I}}
\newcommand{\msfJ}{\mathsf{J}}
\newcommand{\msfK}{\mathsf{K}}
\newcommand{\msfL}{\mathsf{L}}
\newcommand{\msfM}{\mathsf{M}}
\newcommand{\msfN}{\mathsf{N}}
\newcommand{\msfO}{\mathsf{O}}
\newcommand{\msfP}{\mathsf{P}}
\newcommand{\msfQ}{\mathsf{Q}}
\newcommand{\msfR}{\mathsf{R}}
\newcommand{\msfS}{\mathsf{S}}
\newcommand{\msfT}{\mathsf{T}}
\newcommand{\msfU}{\mathsf{U}}
\newcommand{\msfV}{\mathsf{V}}
\newcommand{\msfW}{\mathsf{W}}
\newcommand{\msfX}{\mathsf{X}}
\newcommand{\msfY}{\mathsf{Y}}
\newcommand{\msfZ}{\mathsf{Z}}

% ====================================
% COMMON ABBREVIATIONS
% ====================================

\newcommand{\ie}{i.e.\ }
\newcommand{\eg}{e.g.\ }
\newcommand{\cf}{cf.\ }
\newcommand{\etc}{etc.\ }
\newcommand{\vs}{vs.\ }
\newcommand{\resp}{resp.\ }
\newcommand{\etal}{et al.\ }

  \graphicspath{{./Figures/}}
  \begin{document}
  \renewcommand{\THERMOEND}{\end{document}}
\fi


\chapter{ThermoS26-01}
\label{chap:ThermoS26-01}

\section*{Non-Equilibrium Thermodynamics}
\addcontentsline{toc}{section}{Non-Equilibrium Thermodynamics}

\subsection*{Course Plan}
\begin{enumerate}
\item Equilibrium Thermodynamics
\item Statistical Mechanics
\item Kinetic Theory of Gases
\item Conservation Laws in Continuous Systems
\item Entropy Production
\item Onsager's Principle
\item Applications
\end{enumerate}

\section{Equilibrium Thermodynamics}

\begin{definition}[Isolated System]
\label{defn:isolated-system}
An isolated system is a collection of matter that shares no information with the outside world.
\end{definition}

\begin{figure}[h]
\centering
\includegraphics[width=\textwidth]{Figures/ThermoS26-01-fig-1}
\caption{Isolated system with rigid walls. The system is characterized by internal energy $U$, volume $V$, and mole numbers $N_1, \ldots, N_r$.}
\label{fig:isolated-system}
\end{figure}

\noindent\textbf{Postulate I:} There are particular states, called equilibrium states, of an isolated system that, macroscopically, are characterized by
\begin{itemize}
\item $U$, the internal energy of the system, $U \geq 0$,
\item $V$, the volume of the system, $V \geq 0$,
\item $N_1, \ldots, N_r$, the mole numbers of the $r$ chemical components of the system, $N_i \geq 0$.
\end{itemize}

The rigid walls that isolate the system from the rest of the universe allow no change in volume, no exchange of matter, and no exchange of energy.

\begin{definition}[Composite System]
\label{defn:composite-system}
A composite system is the union of two isolated systems that can exchange volume, matter (chemical components) and/or energy; however, no volume, matter, or energy is exchanged with outside.
\end{definition}

A \textbf{diathermal wall} in a composite system is one which separates two otherwise isolated systems and allows for the exchange of energy, but not matter or volume.

\begin{figure}[h]
\centering
\includegraphics[width=\textwidth]{Figures/ThermoS26-01-fig-2}
\caption{Composite system with diathermal wall separating subsystems $\alpha$ and $\beta$.}
\label{fig:diathermal-wall}
\end{figure}

A \textbf{diathermal piston} in a composite system is one which separates two otherwise isolated systems and allows for the exchange of energy and volume but not matter.

\noindent\textbf{Postulate II:} There exists a function $\tilde{S}$ for an isolated system called its entropy, defined for the system at equilibrium and depending on $U, V, N_1, \ldots, N_r$, \ie,
\begin{equation}
\label{eq:entropy-function}
\tilde{S} = \tilde{S}(U, V, N_1, \ldots, N_r).
\end{equation}

The entropy exists for a composite system and is the sum of the entropy functions, $\tilde{S}^\alpha$ and $\tilde{S}^\beta$, for the respective subsystems, \ie,
\begin{equation}
\label{eq:entropy-additive}
\tilde{S} = \tilde{S}^\alpha + \tilde{S}^\beta = \tilde{S}^\alpha(U^\alpha, V^\alpha, N_1^\alpha, \ldots, N_r^\alpha) + \tilde{S}^\beta(U^\beta, V^\beta, N_1^\beta, \ldots, N_r^\beta).
\end{equation}

The entropy of an isolated system is homogeneous of degree 1, meaning
\begin{equation}
\label{eq:entropy-homogeneous}
\tilde{S}(\lambda U, \lambda V, \lambda N_1, \ldots, \lambda N_r) = \lambda \tilde{S}(U, V, N_1, \ldots, N_r)
\end{equation}
for any $\lambda > 0$.

\noindent\textbf{Postulate III:} The entropy of an isolated system is a concave, twice continuously differentiable, non-negative function over its (convex) domain
\begin{equation}
\label{eq:domain-sigma-u}
\Sigma_u \subseteq [0,\infty)^{r+2} = \underbrace{[0, \infty)}_{(U)} \times \underbrace{[0, \infty)}_{(V)} \times \underbrace{[0, \infty)}_{(N_1)} \times \cdots \times \underbrace{[0, \infty)}_{(N_r)}.
\end{equation}

The entropy is a monotonically increasing function of $U$. In particular,
\begin{equation}
\label{eq:entropy-monotone}
\pderT{\tilde{S}}{U}{V, N_1, \ldots, N_r} > 0, \quad \forall (U, V, \vec{N}) \in \Sigma_u^o,
\end{equation}
where $\Sigma_u^o$ is the interior of the set $\Sigma_u$.

\begin{theorem}
\label{thm:energy-representation}
Suppose that Postulates I--III hold for an isolated system. Then, there is a function
\begin{equation}
\label{eq:energy-function}
\tilde{U} = \tilde{U}(S, V, N_1, \ldots, N_r)
\end{equation}
and a convex domain of definition
\begin{equation}
\label{eq:domain-sigma-s}
\Sigma_s \subseteq [0,\infty)^{r+2} = \underbrace{[0, \infty)}_{(S)} \times \underbrace{[0, \infty)}_{(V)} \times \underbrace{[0, \infty)}_{(N_1)} \times \cdots \times \underbrace{[0, \infty)}_{(N_r)}
\end{equation}
that satisfies
\begin{equation}
\label{eq:energy-entropy-inverse-1}
\tilde{U}(\tilde{S}(U, V, \vec{N}), V, \vec{N}) = U, \quad \forall (U, V, \vec{N}) \in \Sigma_u
\end{equation}
and
\begin{equation}
\label{eq:energy-entropy-inverse-2}
\tilde{S}(\tilde{U}(S, V, \vec{N}), V, \vec{N}) = S, \quad \forall (S, V, \vec{N}) \in \Sigma_s.
\end{equation}

Moreover, $\tilde{U}$ is a twice continuously differentiable convex function with the property
\begin{equation}
\label{eq:energy-monotone}
\pderT{\tilde{U}}{S}{V, N_1, \ldots, N_r} > 0, \quad \forall (S, V, \vec{N}) \in \Sigma_s.
\end{equation}
\end{theorem}

\begin{example}
\label{ex:unary-system}
Suppose that, for a unary material ($r=1$),
\begin{equation}
\label{eq:example-entropy}
\tilde{S}(U, V, N) = \left(\frac{NVUR^2}{v_o \Theta}\right)^{1/3}, \quad \Sigma_u = [0, \infty)^3,
\end{equation}
where $R$, $v_o$, $\Theta > 0$ are constants. Consider the function
\begin{equation}
\label{eq:example-energy}
\tilde{U}(S, V, N) = \frac{S^3 v_o \Theta}{NVR^2}, \quad \Sigma_s = \underbrace{[0, \infty)}_{(S)} \times \underbrace{(0, \infty)}_{(V)} \times \underbrace{(0, \infty)}_{(N)}.
\end{equation}

Then
\begin{equation}
\label{eq:example-inverse-1}
\tilde{S}(\tilde{U}(S, V, N), V, N) = S, \quad \forall (S, V, N) \in \Sigma_s
\end{equation}
and
\begin{equation}
\label{eq:example-inverse-2}
\tilde{U}(\tilde{S}(U, V, N), V, N) = U, \quad \forall (U, V, N) \in \Sigma_u.
\end{equation}
\end{example}

\begin{definition}[Temperature]
\label{defn:temperature}
Suppose that Postulates I--III hold. Then the temperature of an equilibrium isolated system is defined as
\begin{equation}
\label{eq:temperature-def}
T_u(S, V, N_1, \ldots, N_r) \equiv \pderT{U}{S}{V, N_1, \ldots, N_r}.
\end{equation}
\end{definition}

\noindent\textbf{Postulate IV:} Equilibrium of a composite system is that state\\ $((U^\alpha, V^\alpha, \vec{N}^\alpha), (U^\beta, V^\beta, \vec{N}^\beta)) \in \Sigma^{\alpha} \times \Sigma^{\beta}$ such that
\begin{equation}
\label{eq:entropy-maximum}
S = S^\alpha + S^\beta
\end{equation}
is at its maximum possible value.

\begin{definition}[Pressure and Chemical Potential]
\label{defn:pressure-chemical-potential}
The function
\begin{equation}
\label{eq:pressure-def}
P_{\tilde{U}}(S, V, N_1, \ldots, N_r) := -\pderT{\tilde{U}}{V}{S, N_1, \ldots, N_r}
\end{equation}
is called the pressure of a system. The function
\begin{equation}
\label{eq:chemical-potential-def}
\mu_{u,i}(S, V, N_1, \ldots, N_r) := \pderT{\tilde{U}}{N_i}{S, V, N_j, j \neq i}
\end{equation}
is called the $i^{\text{th}}$ chemical potential.
\end{definition}

\begin{theorem}
\label{thm:entropy-partials}
Let $\tilde{S}$ and $\tilde{U}$ be the entropy and internal energy functions of an isolated system. Then,
\begin{equation}
\label{eq:entropy-partial-U}
\pder{\tilde{S}}{U} = \frac{1}{T_s(U, V, \vec{N})},
\end{equation}
where
\begin{equation}
\label{eq:temperature-consistency-1}
T_u(S, V, \vec{N}) = T_s(\tilde{U}(S, V, \vec{N}), V, \vec{N})
\end{equation}
and
\begin{equation}
\label{eq:temperature-consistency-2}
T_s(U, V, \vec{N}) = T_u(\tilde{S}(U, V, \vec{N}), V, \vec{N}).
\end{equation}

Furthermore,
\begin{equation}
\label{eq:entropy-partial-V}
\pder{\tilde{S}}{V} = \frac{P_s(U, V, \vec{N})}{T_s(U, V, \vec{N})},
\end{equation}
where
\begin{equation}
\label{eq:pressure-consistency-1}
P_{\tilde{U}}(S, V, \vec{N}) = P_s(\tilde{U}(S, V, \vec{N}), V, \vec{N}),
\end{equation}
and
\begin{equation}
\label{eq:pressure-consistency-2}
P_s(U, V, \vec{N}) = P_u(\tilde{S}(U, V, \vec{N}), V, \vec{N}).
\end{equation}

And, finally,
\begin{equation}
\label{eq:entropy-partial-N}
\pder{\tilde{S}}{N_j} = -\frac{\mu_{s,j}(U, V, \vec{N})}{T_s(U, V, \vec{N})},
\end{equation}
where
\begin{equation}
\label{eq:chemical-potential-consistency-1}
\mu_{u,j}(S, V, \vec{N}) = \mu_{s,j}(\tilde{U}(S, V, \vec{N}), V, \vec{N})
\end{equation}
and
\begin{equation}
\label{eq:chemical-potential-consistency-2}
\mu_{s,j}(U, V, \vec{N}) = \mu_{u,j}(\tilde{S}(U, V, \vec{N}), V, \vec{N}).
\end{equation}
\end{theorem}

\begin{proof}
Proof Exercises.
\end{proof}

\begin{remark}
We will usually abuse notation and just write
\begin{equation}
\label{eq:notation-abuse}
T_u = T_s, \quad P_{\tilde{U}} = P_s, \quad \mu_{u,j} = \mu_{s,i},
\end{equation}
when the usage may be inferred from the context.
\end{remark}

\noindent\textbf{Postulate V:} The entropy of an isolated system is zero when the temperature is zero; that is,
\begin{equation}
\label{eq:third-law}
T_s(U, V, N_1, \ldots, N_r) = 0 \implies \tilde{S}(U, V, N_1, \ldots, N_r) = 0.
\end{equation}

\begin{example}
\label{ex:absolute-zero}
Consider a unary ($r=1$) isolated system in equilibrium with the fundamental relation
\begin{equation}
\label{eq:example-fundamental}
\tilde{S}(U, V, N) = \left(\frac{NVUR^2}{v_o \Theta}\right)^{1/3}.
\end{equation}

We can set
\begin{equation}
\Sigma_u = [0, \infty)^3.
\end{equation}

$\tilde{S} = 0$ for all $(U, V, N) \in \partial \Sigma_u$, where
\begin{equation}
\label{eq:boundary-sigma-u}
\partial \Sigma_u = \{(U, V, N) \in \Sigma_u \mid U=0, \text{ or } V=0, \text{ or } N=0\}.
\end{equation}

Recall,
\begin{align}
\label{eq:temperature-example}
T_s(U, V, N) &= \frac{3 v_o \Theta}{NVR^2} \tilde{S}(U, V, N)^2 \notag \\
&= \frac{3 v_o \Theta}{NVR^2} \left(\frac{NVUR^2}{v_o \Theta}\right)^{2/3}.
\end{align}

$T_s$ is not defined for all of $\Sigma_u$! It is defined for all
\begin{equation}
(U, V, N) \in \text{dom}(T_s) \subset \Sigma_u.
\end{equation}

Define
\begin{equation}
\label{eq:absolute-zero-set}
AZ := \{(U, V, N) \in \text{dom}(T_s) \mid T_s(U, V, N) = 0\}.
\end{equation}

Then
\begin{equation}
AZ \subset \partial \Sigma_u. \quad (\text{Postulate V})
\end{equation}

I leave it as a homework exercise to determine $\text{dom}(T_s)$ and $AZ$.
\end{example}

\THERMOEND
