\ifdefined\THERMONOTESMAIN
    \providecommand{\THERMOEND}{}
\else
\documentclass[11pt]{book}

\usepackage{cmbright}
\usepackage{mathtools}

\usepackage{pgfplots}
\usepgfplotslibrary{patchplots}
\pgfplotsset{compat=1.15}

\usepackage{color}
\usepackage{xcolor}
\usepackage{subfigure}

\usepackage{amssymb}
\usepackage{amsmath}
\usepackage{amsfonts}
\usepackage{amsthm} 
\usepackage{mdframed}
\usepackage{bbm}
\usepackage{listings}

\usepackage[shortlabels]{enumitem}

\usepackage{tikz}
\usepackage{tikz-3dplot}
\usepackage{xkeyval,tkz-base}

\usetikzlibrary{patterns}
\usetikzlibrary{arrows,arrows.meta,shapes,decorations,automata}
\usetikzlibrary{backgrounds,petri,matrix}
\usetikzlibrary{intersections}
\usetikzlibrary{calc}
\usetikzlibrary{positioning}
\usetikzlibrary{shapes.geometric}
\usetikzlibrary{shapes.arrows}
\usetikzlibrary{matrix}
\usetikzlibrary{positioning}
\usetikzlibrary{3d,calc}

\usepackage{multirow}
\usepackage{array}
\usepackage{algorithm}
\usepackage{algpseudocode}

\usepackage{graphicx}
\usepackage{latexsym}
\usepackage{booktabs}
\usepackage{longtable}

%\usepackage{aligned-overset}

\numberwithin{equation}{section}

\usepackage{lastpage}
\usepackage{fancyhdr}
\pagestyle{fancy}
\setlength{\headheight}{14pt}
\lhead{Mathematical Thermodynamics}
\chead{}
\rhead{Page \thepage}
\cfoot{Page\ \thepage\ of\ \protect\pageref{LastPage}}
\setcounter{section}{-1}

% Theorem environments
\theoremstyle{plain}
\newtheorem{theorem}{Theorem}[section]
\newtheorem{lemma}[theorem]{Lemma}
\newtheorem{proposition}[theorem]{Proposition}
\newtheorem{corollary}[theorem]{Corollary}

\theoremstyle{definition}
\newtheorem{definition}[theorem]{Definition}
\newtheorem{example}[theorem]{Example}
\newtheorem{exercise}[theorem]{Exercise}

\theoremstyle{remark}
\newtheorem{remark}[theorem]{Remark}
\newtheorem{note}[theorem]{Note}

% Hyperref setup
\usepackage{hyperref}
\usepackage{nameref}
\hypersetup{
    hypertexnames=true,
    breaklinks=false,
    colorlinks=true,
    linkcolor=blue,
    citecolor=blue,
    urlcolor=blue
}

\makeatletter
  \newenvironment{dedication}
  {
     \cleardoublepage
     \thispagestyle{empty}
     \vspace*{\stretch{1}}
     \hfill\begin{minipage}[t]{0.66\textwidth}
     \raggedright
  }%
  {
     \end{minipage}
     \vspace*{\stretch{3}}
     \clearpage
  }
  \newenvironment{varsubequations}[1]
  {%
    \addtocounter{equation}{-1}%
    \begin{subequations}
    \renewcommand{\theparentequation}{#1}%
    \def\@currentlabel{#1}%
  }
  {%
    \end{subequations}\ignorespacesafterend
  }
\makeatother

% ====================================
% THERMODYNAMIC STATE VARIABLES
% ====================================

% Extensive properties
\newcommand{\Int}{U}           % Internal energy
\newcommand{\Ent}{S}           % Entropy
\newcommand{\Vol}{V}           % Volume
\newcommand{\Enth}{H}          % Enthalpy H = U + PV
\newcommand{\Helm}{A}          % Helmholtz free energy A = U - TS
\newcommand{\Gibbs}{G}         % Gibbs free energy G = H - TS
\newcommand{\GrandPot}{\Omega} % Grand potential

% Intensive properties
\newcommand{\Temp}{T}          % Temperature
\newcommand{\Press}{P}         % Pressure
\newcommand{\ChemPot}{\mu}     % Chemical potential
\newcommand{\Mass}{m}          % Mass
\newcommand{\Moles}{n}         % Number of moles

% Specific properties (per unit mass)
\newcommand{\sint}{u}          % Specific internal energy
\newcommand{\sent}{s}          % Specific entropy
\newcommand{\svol}{v}          % Specific volume
\newcommand{\senth}{h}         % Specific enthalpy

% Heat and work
\newcommand{\Heat}{Q}          % Heat transfer
\newcommand{\Work}{W}          % Work
\newcommand{\dQ}{\delta Q}     % Inexact differential of heat
\newcommand{\dW}{\delta W}     % Inexact differential of work

% Thermodynamic processes
\newcommand{\iso}{\text{iso}}
\newcommand{\rev}{\text{rev}}
\newcommand{\irrev}{\text{irrev}}

% ====================================
% MATHEMATICAL OPERATORS
% ====================================

\DeclareMathOperator{\grad}{grad}
\DeclareMathOperator{\dive}{div}
\DeclareMathOperator{\curl}{curl}
\DeclareMathOperator{\Span}{span}
\DeclareMathOperator{\rank}{rank}
\DeclareMathOperator{\trace}{tr}
\DeclareMathOperator{\sgn}{sgn}
\DeclareMathOperator{\diam}{diam}

% Partial derivatives
\newcommand{\pder}[2]{\frac{\partial #1}{\partial #2}}
\newcommand{\pdertwo}[2]{\frac{\partial^2 #1}{\partial #2^2}}
\newcommand{\pdermix}[3]{\frac{\partial^2 #1}{\partial #2 \partial #3}}

% Thermodynamic partial derivatives
\newcommand{\pderT}[3]{\left(\frac{\partial #1}{\partial #2}\right)_{#3}}

% Total derivatives
\newcommand{\dder}[2]{\frac{d #1}{d #2}}
\newcommand{\ddertwo}[2]{\frac{d^2 #1}{d #2^2}}

% Differential operators
\newcommand{\dd}{\mathrm{d}}
\newcommand{\dif}{\mathrm{d}}

% ====================================
% VECTOR AND MATRIX NOTATION
% ====================================

% Bold vectors
\newcommand{\bfv}[1]{\boldsymbol{#1}}
\newcommand{\bfx}{\boldsymbol{x}}
\newcommand{\bfy}{\boldsymbol{y}}
\newcommand{\bfz}{\boldsymbol{z}}
\newcommand{\bfu}{\boldsymbol{u}}
\newcommand{\bfv}{\boldsymbol{v}}
\newcommand{\bfw}{\boldsymbol{w}}
\newcommand{\bff}{\boldsymbol{f}}
\newcommand{\bfg}{\boldsymbol{g}}
\newcommand{\bfF}{\boldsymbol{F}}
\newcommand{\bfn}{\boldsymbol{n}}
\newcommand{\bfe}{\boldsymbol{e}}
\newcommand{\bfzero}{\boldsymbol{0}}

% Matrix spaces
\newcommand{\Rn}{\mathbb{R}^n}
\newcommand{\Rm}{\mathbb{R}^m}
\newcommand{\Rmn}{\mathbb{R}^{m \times n}}

% ====================================
% SETS AND SPACES
% ====================================

\newcommand{\R}{\mathbb{R}}    % Real numbers
\newcommand{\C}{\mathbb{C}}    % Complex numbers
\newcommand{\N}{\mathbb{N}}    % Natural numbers
\newcommand{\Z}{\mathbb{Z}}    % Integers
\newcommand{\Q}{\mathbb{Q}}    % Rationals

% Function spaces
\newcommand{\Lp}[1]{L^{#1}}
\newcommand{\Hone}{H^1}
\newcommand{\Ck}[1]{C^{#1}}

% ====================================
% SPECIAL SYMBOLS
% ====================================

% Indicator/characteristic function
\newcommand{\ind}{\mathbbm{1}}

% Inner product
\newcommand{\inner}[2]{\langle #1, #2 \rangle}

% Norm
\newcommand{\norm}[1]{\left\| #1 \right\|}
\newcommand{\abs}[1]{\left| #1 \right|}

% Set notation
\newcommand{\set}[1]{\left\{ #1 \right\}}
\newcommand{\setdef}[2]{\left\{ #1 \,:\, #2 \right\}}

% Ceiling and floor
\newcommand{\ceil}[1]{\left\lceil #1 \right\rceil}
\newcommand{\floor}[1]{\left\lfloor #1 \right\rfloor}

% ====================================
% MATHCAL AND MATHSF LETTERS
% ====================================

% Calligraphic letters for sets and spaces
\newcommand{\calA}{\mathcal{A}}
\newcommand{\calB}{\mathcal{B}}
\newcommand{\calC}{\mathcal{C}}
\newcommand{\calD}{\mathcal{D}}
\newcommand{\calE}{\mathcal{E}}
\newcommand{\calF}{\mathcal{F}}
\newcommand{\calG}{\mathcal{G}}
\newcommand{\calH}{\mathcal{H}}
\newcommand{\calI}{\mathcal{I}}
\newcommand{\calJ}{\mathcal{J}}
\newcommand{\calK}{\mathcal{K}}
\newcommand{\calL}{\mathcal{L}}
\newcommand{\calM}{\mathcal{M}}
\newcommand{\calN}{\mathcal{N}}
\newcommand{\calO}{\mathcal{O}}
\newcommand{\calP}{\mathcal{P}}
\newcommand{\calQ}{\mathcal{Q}}
\newcommand{\calR}{\mathcal{R}}
\newcommand{\calS}{\mathcal{S}}
\newcommand{\calT}{\mathcal{T}}
\newcommand{\calU}{\mathcal{U}}
\newcommand{\calV}{\mathcal{V}}
\newcommand{\calW}{\mathcal{W}}
\newcommand{\calX}{\mathcal{X}}
\newcommand{\calY}{\mathcal{Y}}
\newcommand{\calZ}{\mathcal{Z}}

% Sans-serif letters for matrices and operators
\newcommand{\msfA}{\mathsf{A}}
\newcommand{\msfB}{\mathsf{B}}
\newcommand{\msfC}{\mathsf{C}}
\newcommand{\msfD}{\mathsf{D}}
\newcommand{\msfE}{\mathsf{E}}
\newcommand{\msfF}{\mathsf{F}}
\newcommand{\msfG}{\mathsf{G}}
\newcommand{\msfH}{\mathsf{H}}
\newcommand{\msfI}{\mathsf{I}}
\newcommand{\msfJ}{\mathsf{J}}
\newcommand{\msfK}{\mathsf{K}}
\newcommand{\msfL}{\mathsf{L}}
\newcommand{\msfM}{\mathsf{M}}
\newcommand{\msfN}{\mathsf{N}}
\newcommand{\msfO}{\mathsf{O}}
\newcommand{\msfP}{\mathsf{P}}
\newcommand{\msfQ}{\mathsf{Q}}
\newcommand{\msfR}{\mathsf{R}}
\newcommand{\msfS}{\mathsf{S}}
\newcommand{\msfT}{\mathsf{T}}
\newcommand{\msfU}{\mathsf{U}}
\newcommand{\msfV}{\mathsf{V}}
\newcommand{\msfW}{\mathsf{W}}
\newcommand{\msfX}{\mathsf{X}}
\newcommand{\msfY}{\mathsf{Y}}
\newcommand{\msfZ}{\mathsf{Z}}

% ====================================
% COMMON ABBREVIATIONS
% ====================================

\newcommand{\ie}{i.e.\ }
\newcommand{\eg}{e.g.\ }
\newcommand{\cf}{cf.\ }
\newcommand{\etc}{etc.\ }
\newcommand{\vs}{vs.\ }
\newcommand{\resp}{resp.\ }
\newcommand{\etal}{et al.\ }


    \providecommand{\THERMOEND}{\end{document}}
    \begin{document}
\fi

\chapter{Non-Equilibrium Thermodynamics}
\label{chap:thermo-s26-01}

\begin{center}
Math-Thermo\\
Chap 01\\
01/20/2026
\end{center}

\graphicspath{{Figures/}}

\section{Course Plan}
\begin{enumerate}
    \item Equilibrium Thermodynamics
    \item Statistical Mechanics
    \item Kinetic Theory of Gases
    \item Conservation Laws in Continuous Systems
    \item Entropy Production
    \item Onsager's Principle
    \item Applications
\end{enumerate}

\section{Equilibrium Thermodynamics}

\begin{definition}
\textbf{(1.1)} An isolated system is a collection of
matter that shares no information with the outside world.
\end{definition}

\paragraph{Postulate I.}
There are particular states, called equilibrium states,
of an isolated system that, macroscopically, are characterized by

\begin{align*}
U, &\quad \text{the internal energy of the system,} \quad U > 0,\\
V, &\quad \text{the volume of the system,} \quad V > 0,\\
N_1,\ldots,N_r, &\quad \text{the mole numbers of the $r$ chemical components of the system,} \quad N_i \ge 0.
\end{align*}

\begin{figure}[h]
    \centering
    \includegraphics[width=0.8\textwidth]{ThermoS26-01-fig-1.pdf}
    \caption{Isolated system with rigid walls.}
\end{figure}

The rigid walls that isolate the system from the rest of the universe allow no change in
volume, no exchange of matter, and no exchange of energy.

\begin{definition}
\textbf{(1.2)} A composite system is the union
of two isolated systems that can exchange volume,
matter (chemical components), and/or energy; however
no volume, matter, or energy is exchanged with outside.
\end{definition}

A diathermal wall in a composite system is one
which separates two otherwise isolated systems
and allows for the exchange of energy, but not
matter or volume.

\begin{figure}[h]
    \centering
    \includegraphics[width=0.8\textwidth]{ThermoS26-01-fig-2.pdf}
    \caption{Composite system with a diathermal wall.}
\end{figure}

A diathermal partition in a composite system is one
which separates two otherwise isolated systems
and allows for the exchange of energy and volume
but not matter.

\paragraph{Postulate II.}
There exists a function $\tilde{S}$ for an
isolated system called its entropy, defined
for the system at equilibrium and dependent on
$U, V, N_1, \ldots, N_r$, i.e.,

\[
\tilde{S} = \tilde{S}(U, V, N_1, \ldots, N_r).
\]

The entropy states for a composite system and
is the sum of the entropy functions, $\tilde{S}^\alpha$ and
$\tilde{S}^\beta$, for the respective subsystems, i.e.,

\begin{equation}
\tag{1.1}
\tilde{S} = \tilde{S}^\alpha + \tilde{S}^\beta = \tilde{S}^\alpha(U^\alpha, V^\alpha, N_1^\alpha, \ldots, N_r^\alpha)
+ \tilde{S}^\beta(U^\beta, V^\beta, N_1^\beta, \ldots, N_r^\beta)
\end{equation}

The entropy of an isolated system is homogeneous
of degree 1, meaning

\[
\tilde{S}(\lambda U, \lambda V, \lambda N_1, \ldots, \lambda N_r) = \lambda \tilde{S}(U, V, N_1, \ldots, N_r)
\]

for any $\lambda > 0$.

\paragraph{Postulate III.}
The entropy of an isolated
system is a concave, twice continuously differentiable,
positive function over its (convex) domain,

\[
\Sigma_0 \subseteq [0,\infty)^{r+2} = [0,\infty) \times [0,\infty) \times [0,\infty) \times \cdots \times [0,\infty)
\]
\[
(U) \quad (V) \quad (N_1) \quad \cdots \quad (N_r)
\]

The entropy is a monotonically increasing
function of $U$. In particular,

\[
\left(\frac{\partial \tilde{S}}{\partial U}\right)_{V,N_1,\ldots,N_r} > 0, \quad \forall (U,V,\vec{N}) \in \Sigma_0.
\]

\begin{theorem}
\textbf{(1.3)} Suppose that Postulates I--III hold
for an isolated system. Then, there is a
function

\[
\tilde{U} = \tilde{U}(S, V, N_1, \ldots, N_r)
\]

and a convex domain of definition

\[
\Sigma_s \subseteq [0,\infty)^{r+2} = [0,\infty) \times [0,\infty) \times [0,\infty) \times \cdots \times [0,\infty)
\]
\[
(S) \quad (V) \quad (N_1) \quad \cdots \quad (N_r)
\]

that satisfies

\[
\tilde{U}(\tilde{S}(U,V,\vec{N}), V, \vec{N}) = U, \quad \forall (U,V,\vec{N}) \in \Sigma_0
\]

and

\[
\tilde{S}(\tilde{U}(S,V,\vec{N}), V, \vec{N}) = S, \quad \forall (S,V,\vec{N}) \in \Sigma_s.
\]

Moreover, $\tilde{U}$ is a twice continuously differentiable convex
function with the property

\[
\left(\frac{\partial \tilde{U}}{\partial S}\right)_{V,N_1,\ldots,N_r} > 0, \quad \forall (S,V,\vec{N}) \in \Sigma_s.
\]
\end{theorem}

\begin{example}
\textbf{(1.4)} Suppose that for a unary material ($r=1$),

\[
\tilde{S}(U,V,N) = \left(\frac{N V U R^2}{\nu_0 \Theta}\right)^{1/3}, \quad \Sigma_0 = [0,\infty)^3,
\]

where $R$, $\nu_0$, $\Theta > 0$ are constants. Consider
the function,

\[
\tilde{U}(S,V,N) = \frac{S^3 \nu_0 \Theta}{N V R^2}, \quad \Sigma_s = ?
\]

Then

\[
\tilde{S}(\tilde{U}(S,V,N), V, N) = S, \quad \forall (S,V,N) \in \Sigma_s
\]

and

\[
\tilde{U}(\tilde{S}(U,V,N), V, N) = U, \quad \forall (U,V,N) \in \Sigma_0.
\]
\end{example}

\begin{definition}
\textbf{(1.5)} Suppose that Postulates I--III hold. Then
the temperature of an equilibrium isolated system is
defined as

\begin{equation}
\tag{1.2}
T(U,S,N_1,\ldots,N_r) \equiv \left(\frac{\partial \tilde{U}}{\partial S}\right)_{V,N_1,\ldots,N_r}.
\end{equation}
\end{definition}

\paragraph{Postulate IV.}
Equilibrium of a composite system
is that state $(U^\alpha, V^\alpha, \vec{N}^\alpha), (U^\beta, V^\beta, \vec{N}^\beta) \in \Sigma_s^\alpha \times \Sigma_s^\beta$
such that

\[
S = S^\alpha + S^\beta
\]

is at its maximum possible value.

\begin{definition}
\textbf{(1.6)} The functions

\begin{equation}
\tag{1.3}
P_0(S,V,N_1,\ldots,N_r) := -\left(\frac{\partial \tilde{U}}{\partial V}\right)_{S,N_1,\ldots,N_r}
\end{equation}

is called the pressure of a system. The function

\begin{equation}
\tag{1.4}
\mu_{0,i}(S,V,N_1,\ldots,N_r) := \left(\frac{\partial \tilde{U}}{\partial N_i}\right)_{S,V,N_j, j\ne i}
\end{equation}

is called the $i^{\text{th}}$ chemical potential.
\end{definition}

\begin{theorem}
\textbf{(1.7)} Let $\tilde{S}$ and $\tilde{U}$ be the entropy and
internal energy functions of an isolated system.
Then,

\[
\frac{\partial \tilde{S}}{\partial U} = \frac{1}{T_s(U,V,\vec{N})},
\]

where

\[
T_0(S,V,\vec{N}) = T_s(\tilde{U}(S,V,\vec{N}), V, \vec{N})
\]

and

\[
T_s(U,V,\vec{N}) = T_0(\tilde{S}(U,V,\vec{N}), V, \vec{N}).
\]

Furthermore,

\[
\frac{\partial \tilde{S}}{\partial V} = \frac{P_s(U,V,\vec{N})}{T_s(U,V,\vec{N})},
\]

where

\[
P_0(S,V,\vec{N}) = P_s(\tilde{U}(S,V,\vec{N}), V, \vec{N}),
\]

and

\[
P_s(U,V,\vec{N}) = P_0(\tilde{S}(U,V,\vec{N}), V, \vec{N}).
\]

And finally,

\[
\frac{\partial \tilde{S}}{\partial N_i} = -\frac{\mu_{s,i}(U,V,\vec{N})}{T_s(U,V,\vec{N})}.
\]

where

\[
\mu_{0,i}(S,V,\vec{N}) = \mu_{s,i}(\tilde{U}(S,V,\vec{N}), V, \vec{N})
\]

and

\[
\mu_{s,i}(U,V,\vec{N}) = \mu_{0,i}(\tilde{S}(U,V,\vec{N}), V, \vec{N}).
\]
\end{theorem}

\begin{proof}
Exercise !!!
\end{proof}

\begin{remark}
We will usually abuse notation
and just write

\[
T_0 = T_s, \quad P_0 = P_s, \quad \mu_{0,i} = \mu_{s,i},
\]

when the usage may inferred from the context.
\end{remark}

\paragraph{Postulate V.}
The entropy of an isolated system
is zero when the temperature is zero; that is,

\[
T_s(U,V,N_1,\ldots,N_r) = 0 \Longrightarrow \tilde{S}(U,V,N_1,\ldots,N_r) = 0.
\]

\begin{example}
\textbf{(1.8)} Consider a unary ($r=1$) isolated
system in equilibrium with the fundamental
relation

\[
\tilde{S}(U,V,N) = \left(\frac{N V U R^2}{\nu_0 \Theta}\right)^{1/3}.
\]

We can set

\[
\Sigma_0 = [0,\infty)^3.
\]

$\tilde{S} = 0$ for all $(U,V,N) \in \partial \Sigma_0$, where

\[
\partial \Sigma_0 = \{(U,V,N) \in \Sigma_0 : U=0, \text{ or } V=0, \text{ or } N=0\}
\]

Recall,

\[
T_s(U,V,N) = \frac{3 \nu_0 \Theta}{N V R^2} \tilde{S}(U,V,N)
\]

\[
= \frac{3 \nu_0 \Theta}{N V R^2} \left(\frac{N V U R^2}{\nu_0 \Theta}\right)^{1/3}.
\]

$T_s$ is not defined for all of $\Sigma_0$! It is
defined for all

\[
(U,V,N) \in \mathrm{dom}(T_s) \subset \Sigma_0
\]

Define

\[
AZ := \{(U,V,N) \in \mathrm{dom}(T_s) \mid T_s(U,V,N) = 0\}.
\]

Then

\[
AZ \subset \partial \Sigma_0.
\]

(Postulate V)

I leave it as a homework exercise to determine
$\mathrm{dom}(T_s)$ and $AZ$.
\end{example}

\begin{remark}
From this point forward, we will assume
that all postulates hold.
\end{remark}

\begin{theorem}
\textbf{(1.9)} Suppose that in a composite system
$\alpha$ and $\beta$ are separated by a diathermal wall.
Then, the equilibrium of the composite system
may be characterized by

\begin{equation}
\tag{1.5}
U^\alpha + U^\beta = U_0
\end{equation}

and

\begin{equation}
\tag{1.6}
T^\alpha(U^\alpha, V^\alpha, \vec{N}^\alpha) = T^\beta(U^\beta, V^\beta, \vec{N}^\beta).
\end{equation}
\end{theorem}

\begin{proof}
Internal energy may be exchanged between
systems $\alpha$ and $\beta$ but cannot be exchanged with
the outside world. Thus (1.5) must hold because
of energy conservation. At equilibrium we must have

\[
S(U^\alpha) = S^\alpha(U^\alpha, V^\alpha, \vec{N}^\alpha) + S^\beta(U_0 - U^\alpha, V^\beta, \vec{N}^\beta)
\]

and

\[
\frac{\partial S}{\partial U^\alpha} = 0.
\]

Note, all other variables besides $U^\alpha$ are fixed.

\[
0 = \frac{\partial S^\alpha}{\partial U^\alpha} + \frac{\partial S^\beta}{\partial U^\beta}\frac{\partial}{\partial U^\alpha}(U_0 - U^\alpha)
\]

\[
= \left[T^\alpha(U^\alpha,V^\alpha,\vec{N}^\alpha)\right]^{-1} + \left[T^\beta(U_0 - U^\alpha, V^\beta, \vec{N}^\beta)\right]^{-1}(-1)
\]

Thus,

\[
T^\alpha(U^\alpha,V^\alpha,\vec{N}^\alpha) = T^\beta(U^\beta,V^\beta,\vec{N}^\beta)
\]

with

\[
U^\beta = U_0 - U^\alpha.
\]

How do we know that solutions exist and
are unique?

\begin{figure}[h]
    \centering
    \includegraphics[width=0.8\textwidth]{ThermoS26-01-fig-3.pdf}
    \caption{Concave entropy curves $\tilde{S}^\alpha(U^\alpha)$ and $\tilde{S}^\beta(U_0-U^\alpha)$.}
\end{figure}

This proof can also be carried out by using
Lagrange multipliers. Set

\[
J := S^\alpha(U^\alpha,V^\alpha,\vec{N}^\alpha) + S^\beta(U^\beta,V^\beta,\vec{N}^\beta)
+ \lambda(U^\alpha + U^\beta - U_0)
\]

Then, equilibrium is characterized by

\[
0 = \frac{\partial J}{\partial U^\alpha} = \left[T^\alpha(U^\alpha,V^\alpha,\vec{N}^\alpha)\right]^{-1} + \lambda,
\]

\[
0 = \frac{\partial J}{\partial U^\beta} = \left[T^\beta(U^\beta,V^\beta,\vec{N}^\beta)\right]^{-1} + \lambda,
\]

\[
0 = \frac{\partial J}{\partial \lambda} = U^\alpha + U^\beta - U_0,
\]

which yields the same result. ///
\end{proof}

\begin{figure}[h]
    \centering
    \includegraphics[width=0.75\textwidth]{ThermoS26-01-fig-4.pdf}
    \caption{Graph of $\tilde{S}(U) = \tilde{S}^\alpha(U) + \tilde{S}^\beta(U_0-U)$ with maximum at $U_*$.}
\end{figure}

The Lagrange Multiplier techniques can be
visualized as follows:

\begin{figure}[h]
    \centering
    \includegraphics[width=0.75\textwidth]{ThermoS26-01-fig-5.pdf}
    \caption{Contour lines of $\tilde{S}$ and the constraint line $U^\alpha+U^\beta=U_0$.}
\end{figure}

\begin{theorem}
\textbf{(1.9)} Suppose that in a composite system
$\alpha$ and $\beta$ are separated by a diathermal partition.
Then, the equilibrium of the composite system
may be characterized by

\begin{equation}
\tag{1.7}
U^\alpha + U^\beta = U_0,
\end{equation}

\begin{equation}
\tag{1.8}
V^\alpha + V^\beta = V_0,
\end{equation}

and

\begin{equation}
\tag{1.9}
T^\alpha(U^\alpha,V^\alpha,\vec{N}^\alpha) = T^\beta(U^\beta,V^\beta,\vec{N}^\beta), \quad \text{(thermal equil.)}
\end{equation}

\begin{equation}
\tag{1.10}
P^\alpha(U^\alpha,V^\alpha,\vec{N}^\alpha) = P^\beta(U^\beta,V^\beta,\vec{N}^\beta). \quad \text{(mechanical equil.)}
\end{equation}
\end{theorem}

\begin{proof}
For this, we use the method of Lagrange
multipliers. Note that $\vec{N}^\alpha$ and $\vec{N}^\beta$ are fixed. Define

\[
J := S^\alpha(U^\alpha,V^\alpha,\vec{N}^\alpha) + S^\beta(U^\beta,V^\beta,\vec{N}^\beta)
+ \lambda_u(U^\alpha + U^\beta - U_0)
+ \lambda_v(V^\alpha + V^\beta - V_0).
\]

The conditions for equilibrium are

\[
0 = \frac{\partial J}{\partial U^\alpha} = \left[T^\alpha(U^\alpha,V^\alpha,\vec{N}^\alpha)\right]^{-1} + \lambda_u,
\]

\[
0 = \frac{\partial J}{\partial U^\beta} = \left[T^\beta(U^\beta,V^\beta,\vec{N}^\beta)\right]^{-1} + \lambda_u,
\]

\[
0 = \frac{\partial J}{\partial V^\alpha} = -\frac{P^\alpha(U^\alpha,V^\alpha,\vec{N}^\alpha)}{T^\alpha(U^\alpha,V^\alpha,\vec{N}^\alpha)} + \lambda_v,
\]

\[
0 = \frac{\partial J}{\partial V^\beta} = -\frac{P^\beta(U^\beta,V^\beta,\vec{N}^\beta)}{T^\beta(U^\beta,V^\beta,\vec{N}^\beta)} + \lambda_v.
\]

and

\[
0 = \frac{\partial J}{\partial \lambda_u} = U^\alpha + U^\beta - U_0,
\]

\[
0 = \frac{\partial J}{\partial \lambda_v} = V^\alpha + V^\beta - V_0.
\]

The result is clear. ///
\end{proof}

Here, we have used the fact that

\[
\frac{\partial S^\alpha}{\partial U^\alpha} = \frac{1}{T^\alpha}
\]

\[
\frac{\partial S^\alpha}{\partial V^\alpha} = \frac{P^\alpha}{T^\alpha}
\]

and

\[
\frac{\partial S^\alpha}{\partial N_i^\alpha} = -\frac{\mu_i^\alpha}{T^\alpha}.
\]

As a short hand, we will write

\[
\mathrm{d}U^\alpha = T^\alpha \,\mathrm{d}S^\alpha - P^\alpha \,\mathrm{d}V^\alpha + \sum_{i=1}^r \mu_i^\alpha \,\mathrm{d}N_i^\alpha
\]

and

\[
\mathrm{d}S^\alpha = \frac{1}{T^\alpha} \,\mathrm{d}U^\alpha + \frac{P^\alpha}{T^\alpha} \,\mathrm{d}V^\alpha - \sum_{i=1}^r \frac{\mu_i^\alpha}{T^\alpha} \,\mathrm{d}N_i^\alpha.
\]

Using our abusive notation, we have

\[
\mathrm{d}S^\alpha = \frac{1}{T^\alpha} \,\mathrm{d}U^\alpha + \frac{P^\alpha}{T^\alpha} \,\mathrm{d}V^\alpha - \sum_{i=1}^r \frac{\mu_i^\alpha}{T^\alpha} \,\mathrm{d}N_i^\alpha,
\]

for example.

\begin{theorem}
\textbf{(1.10)} Suppose that a composite system
is comprised of two otherwise isolated systems
with no barrier between the systems. Suppose
that

\[
S^\alpha = S^\alpha(U^\alpha,V^\alpha,\vec{N}^\alpha)
\]

and

\[
S^\beta = S^\beta(U^\beta,V^\beta,\vec{N}^\beta)
\]

are the fundamental entropy relations for
two systems. Then the equilibrium state is
defined by the relations

\begin{equation}
\tag{1.11}
U^\alpha + U^\beta = U_0,
\end{equation}

\begin{equation}
\tag{1.12}
V^\alpha + V^\beta = V_0,
\end{equation}

and

\begin{equation}
\tag{1.13}
N_i^\alpha + N_i^\beta = N_{0,i}
\end{equation}

where $U_0, V_0, N_{0,i} > 0$, and

\begin{equation}
\tag{1.14}
T^\alpha(U^\alpha,V^\alpha,\vec{N}^\alpha) = T^\beta(U^\beta,V^\beta,\vec{N}^\beta), \quad \text{(thermal equil.)}
\end{equation}

\begin{equation}
\tag{1.15}
P^\alpha(U^\alpha,V^\alpha,\vec{N}^\alpha) = P^\beta(U^\beta,V^\beta,\vec{N}^\beta), \quad \text{(mech. equil.)}
\end{equation}

\begin{equation}
\tag{1.16}
\mu_i^\alpha(U^\alpha,V^\alpha,\vec{N}^\alpha) = \mu_i^\beta(U^\beta,V^\beta,\vec{N}^\beta), \quad \text{(chem. equil.)}
\end{equation}

Proof: The procedure is the same. One can
use the method of Lagrange multipliers to do the
calculations. In particular, define

\[
J := S^\alpha(U^\alpha,V^\alpha,\vec{N}^\alpha) + S^\beta(U^\beta,V^\beta,\vec{N}^\beta) + \lambda_u(U^\alpha + U^\beta - U_0)
+ \lambda_v(V^\alpha + V^\beta - V_0) + \sum_{i=1}^r \lambda_i (N_i^\alpha + N_i^\beta - N_{0,i}).
\]
///
\end{theorem}

\THERMOEND
