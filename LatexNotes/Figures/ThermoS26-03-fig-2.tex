\documentclass[tikz,border=10pt]{standalone}
\usepackage{pgfplots}
\usetikzlibrary{decorations.pathmorphing, arrows.meta, positioning, calc}

\begin{document}
\begin{tikzpicture}
    \begin{axis}[
        % --- Axis Setup ---
        axis lines = center,
        xlabel = {$\tau$},
        ylabel = {},
        xlabel style = {below right, font=\Large},
        ylabel style = {above left},
        axis line style = {-Stealth, thick},
        % Domain and Range
        xmin = -0.1, xmax = 1.4,
        ymin = -2.5, ymax = 1.8,
        % Hide standard ticks
        ticks = none,
        clip = false,
        width = 12cm,
        height = 7cm
    ]

        % Coordinates
        \coordinate (start) at (axis cs:0, -2);
        \coordinate (end) at (axis cs:1, 0);
        % Intersection point (approximate based on curve controls)
        \coordinate (cross) at (axis cs:0.18, 0); 

        % --- 1. The Curves ---

        % Black Curve: Sagging approach
        \draw[thick, black] (start) 
            .. controls (axis cs:0.4, -1.5) and (axis cs:0.8, -0.8) .. (end);

        % Red Curve: Custom Bezier Path
        % Adjusted slightly to ensure it crosses exactly at the tick mark (0.18)
        \draw[thick, red] (start) 
            .. controls (axis cs:0.1, -0.5) and (axis cs:0.12, 1.2) .. (axis cs:0.4, 1.2)
            .. controls (axis cs:0.65, 1.2) and (axis cs:0.8, 0.1) .. (end);

        % --- 2. Ticks and Labels ---

        % tau* tick mark
        % Drawn explicitly thick and black
        \draw[very thick, black] ($(cross)+(-6,10)$) -- ($(cross)+(-6,-10)$);
        
        % Label closer to the tick mark
        \node[above=2pt, font=\large] at ($(cross)+(-5,-60)$) {$\tau^*$};

        % tau = 1 point
        \node[circle, fill=black, inner sep=2pt] at (end) {};
        \node[above=5pt, font=\large] at (end) {$\tau=1$};

        % Start point
        \node[circle, fill=red, draw=black, inner sep=2pt] at (start) {};

        % --- 3. Equation ---
        \node[anchor=north west, font=\Large] at (axis cs: 0.35, -1.5) {
            $R(\tau) = \frac{1}{\tilde{T}^\alpha(\tau)} - \frac{1}{\tilde{T}^\beta(\tau)}$
        };

    \end{axis}
\end{tikzpicture}
\end{document}